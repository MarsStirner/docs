\subsection{Ведение договоров и прейскурантов} \label{dog}
\subsubsection{Регистрация базового договора на оказание медицинских услуг} \label{dog_add}

\textbf{Необходимые начальные условия:} Должны быть выполнены первоначальные настройки справочников в системе

\textbf{Роли пользователя:} Эономист, Администратор.

\setcounter{nnn}{0}
\begin{longtable}{|p{1cm}|p{7.5cm}|p{8cm}|}
\caption{Регистрация договора \label{dog_add_tbl}}\\
\hline \rule{0pt}{15pt}  \centering \textbf{№ пп} & \centering \textbf{Действие} & \hfil \textbf{Ожидаемый результат} \\ \hline
\endfirsthead
\hline \rule{0pt}{15pt} \centering \textbf{№ пп} & \centering \textbf{Действие} & \hfil \textbf{Ожидаемый результат} \\ \hline
\endhead
\nn & В главном меню выбрать пункт \mm{Расчет \str Договоры}. & Открывается форма \kw{Договоры}. \\ \hline
\nn & Нажать клавишу F9 на клавиатуре или кнопку \kw{Вставка F9} в нижней части формы. & Открывается форма \kw{Договор}. \\ \hline
\nn & В поле \dm{Тип финансирования} выбрать <<Республиканский бюджет БП 036>>, в перечисленные ниже поля ввести указанные значения: \newline \dm{Номер}: <<115>>,  \newline \dm{Группа}: <<Республианский бюджет>>,  \newline \dm{Дата}: 15 июня текущего года,   \newline \dm{Основание}: <<Республиканский бюджет>>,  \newline \dm{Период с}: 1 января следующего года,  \newline \dm{по}: 31 декабря следующего года.  & Данные внесены в соответствующие поля. \\ \hline
\nn & В поле \dm{Получатель} выбрать <<ГКП на ПВХ <<Степногорская центральная городская больница>>>>. & В поля, содержащие реквизиты получателя, автоматически подставились значения из справочника организаций. \\ \hline
\nn & В поле \dm{Плательщик} выбрать <<ГУ <<Управление здравоохранением Акмолинской области>>>>. & В поля, содержащие реквизиты плательшика, автоматически подставились значения из справочника организаций. \\ \hline
\nn & Дважды щелкнуть левой кнопкой мыши по первой строке таблицы, расположенной в нижней части формы и выбрать из справочника <<Поликлиника (бюджет)>>. & Запись добавлена в таблицу. \\ \hline
\nn & Дважды щелкнуть левой кнопкой мыши по следующей строке таблицы и выбрать из справочника <<Стационарное лечение (бюджет)>>. & Запись добавлена в таблицу. \\ \hline
\nn & Перейти на следующую строку таблицы. & Курсор установлен на пустой строке таблицы. \\ \hline
\nn & Нажать кнопку \kw{OK}. & Текущая форма закрывается. Данные договора сохраняются в БД. Осуществляется возврат к форме \kw{Договоры}. \\ \hline
\end{longtable}

\subsubsection{Регистрация прейскуранта на оказание медицинских услуг} \label{price_add}

\textbf{Необходимые начальные условия:} Должна быть выполнена регистрация договора согласно п. \ref{dog_add}

\textbf{Роли пользователя:} Экономист, Администратор.

\setcounter{nnn}{0}
\begin{longtable}{|p{1cm}|p{7.5cm}|p{8cm}|}
\caption{Регистрация прейскуранта \label{price_add_tbl}}\\
\hline \rule{0pt}{15pt}  \centering \textbf{№ пп} & \centering \textbf{Действие} & \hfil \textbf{Ожидаемый результат} \\ \hline
\endfirsthead
\hline \rule{0pt}{15pt} \centering \textbf{№ пп} & \centering \textbf{Действие} & \hfil \textbf{Ожидаемый результат} \\ \hline
\endhead
\nn & В главном меню выбрать пункт \mm{Расчет \str Договоры}. & Открывается форма \kw{Договоры}. \\ \hline
\nn & Дважды щелкнуть левой кнопкой мыши по строке <<Республиканский бюджет БП 036, Республиканский бюджет, 115>> & Открывается форма \kw{Договор}, содержащая данные, внесенные при выполнении п. \ref{dog_add} \\ \hline
\nn & В нижней части формы выбрать вкладку \kw{Тариф}. & Осуществлен переход на вкладку \kw{Тариф}.\\ \hline
\nn & Дважды щелкнув левой кнопкой мыши в соответствующей ячейке, выбрать из справочников следующие значения: \newline \dm{Событие}: <<Стационарное лечение (бюджет)>>,\newline  \dm{Тарифицируется}: <<мероприятие по количеству>>, \newline \dm{Услуга}: <<10.35 Койко-день для стационарного больного>>, \newline \dm{Ед.усл.}: <<Усл>>. \newline В ячейку \dm{Цена} ввести с клавиатуры <<7000>>. & Заполнена строка в таблице \kw{Тарификация}. \\ \hline
\nn & Аналогично предыдущей строке, заполнить вторую строку, указав следующие значения: \newline \dm{Событие}: <<Поликлиника (бюджет)>>,\newline  \dm{Тарифицируется}: <<мероприятие по количеству>>, \newline \dm{Услуга}: <<А01.0001.000 Прием: терапевт>>, \newline \dm{Ед.усл.}: <<Усл>>. \newline В ячейку \dm{Цена} ввести с клавиатуры <<250>>. & Заполнена строка в таблице \kw{Тарификация}. \\ \hline
\nn & Аналогично добавить строки по другим услугам.  & Заполнена таблица \kw{Тарификация}.\\ \hline
\nn & Перейти на следующую строку таблицы. & Курсор установлен на пустой строке таблицы. \\ \hline
\nn & Нажать кнопку \kw{OK}. & Текущая форма закрывается. Данные сохраняются в БД. Осуществляется возврат к форме \kw{Договоры}. \\ \hline
\end{longtable}