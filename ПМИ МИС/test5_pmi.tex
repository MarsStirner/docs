\subsection{Получение списков и отчетов в поликлинике} \label{rep_pol}
\subsubsection{Выполнение назначенных процедур} \label{proclist_pol}

\textbf{Необходимые начальные условия:} Должно быть зарегистрировано поликлиническое обращение пациента согласно п. \ref{obsl1_pol} Должно быть зарегистрировано направление на физиотерапевтическое лечение согласно п. \ref{fiz_pol}

\textbf{Роли пользователя:} Медсестра, Врач поликлиники, Администратор.

\setcounter{nnn}{0}
\begin{longtable}{|p{1cm}|p{7.5cm}|p{8cm}|}
\caption{Выполнение назначенных процедур \label{proclist_ pol_tbl}}\\
\hline \rule{0pt}{15pt}  \centering \textbf{№ пп} & \centering \textbf{Действие} & \hfil \textbf{Ожидаемый результат} \\ \hline
\endfirsthead
\hline \rule{0pt}{15pt} \centering \textbf{№ пп} & \centering \textbf{Действие} & \hfil \textbf{Ожидаемый результат} \\ \hline
\endhead
\nn & В главном меню выбрать пункт \mm{Работа \str Обслуживание пациентов}. & Открывается форма, содержащая картотеку пациентов. \\ \hline
\nn & Перейти на вкладку \kw{Обслуживание}. & Осуществлен переход на вкладку \kw{Обслуживаниение}. \\ \hline
\nn & Внутри вкладки \kw{Обслуживание} перейти на вкладку \kw{Лечение}. & Осуществлен переход на вкладку \kw{Лечение} на вкладке \kw{Обслуживание}. \\ \hline
\nn & На панели \kw{Фильтр} в правой части экрана установить флажок \dm{Тип} и в ставшем активным поле выбрать из списка <<3\_10\_3|Электрофорез>>, установить флажок \dm{Состояние} и в ставшем активным поле выбрать из списка <<Начато>>. В поле \dm{Показывать мероприятия} выбрать значение <<все>>. & Заданы параметры поиска назначенных процедур. \\ \hline
\nn & Нажать кнопку \kw{Применить} на панели \kw{Фильтр}. & Отображается список направлений пациентов на лечение, найденных по заданным параметрам поиска. \\ \hline
\nn & Дважды щелкнуть левой кнопкой мыши по записи <<Иванов Иван Васильевич ,12.12.1961, М ...>>. & Открывается форма \kw{Иванов Иван Васильевич - Электрофорез}. \\ \hline
\nn & В поле \dm{Состояние} выбрать из списка значение <<Завершено>>.& В поле \dm{Закончено} автоматически установлена текущая дата. \\ \hline
\nn & Нажать кнопку \kw{Сохранить}. & Текущая форма закрывается. Данные сохраняются в БД. Осуществляется возврат на вкладку \kw{Обслуживание}. Запись <<Иванов Иван Васильевич ,12.12.1961, М ...>> исчезла из списка. \\ \hline
\end{longtable}

\subsubsection{Получение отчетов по поликлинике} \label{rep1_pol}

\textbf{Необходимые начальные условия:} Должно быть зарегистрировано поликлиническое обращение пациента согласно п. \ref{obsl1_pol} Должен быть зарегистрирован осмотр врача согласно п. \ref{osmotr_pol}

\textbf{Роли пользователя:} Медстатистик, Врач поликлиники, Администратор.

\setcounter{nnn}{0}
\begin{longtable}{|p{1cm}|p{7.5cm}|p{8cm}|}
\caption{Получение отчетов по поликлинике \label{rep1_ pol_tbl}}\\
\hline \rule{0pt}{15pt}  \centering \textbf{№ пп} & \centering \textbf{Действие} & \hfil \textbf{Ожидаемый результат} \\ \hline
\endfirsthead
\hline \rule{0pt}{15pt} \centering \textbf{№ пп} & \centering \textbf{Действие} & \hfil \textbf{Ожидаемый результат} \\ \hline
\endhead
\nn & В главном меню выбрать пункт \mm{Анализ \str Посещаемость \str Ф.30}. & Открывается форма \kw{Форма 30}. \\ \hline
\nn & В полях \dm{Дата начала периода} и \dm{Дата окончания периода} указать первый и последний день текущего месяца. & Заданы параметры формирования отчета. \\ \hline
\nn & Нажать кнопку \kw{OK}. & Открывается форма предварительного просмотра печати \kw{Форма 30}. \\ \hline
\nn & Нажать кнопку \kw{Печатать} и направить отчет на нужный принтер. & Отчет распечатан. \\ \hline
\nn & Нажать кнопку \kw{Закрыть}. & Окно предварительного просмотра печати закрывается. \\ \hline
\nn & В главном меню выбрать пункт \mm{Анализ \str Посещаемость \str Ф.39}. & Открывается форма \kw{Форма 39}. \\ \hline
\nn & В полях \dm{Дата начала периода} и \dm{Дата окончания периода} указать первый и последний день текущего месяца. В поле \dm{Врач} выбрать из списка <<2299 Дорофеев Р.О., Лечебное дело>>. & Заданы параметры формирования отчета. \\ \hline
\nn & Нажать кнопку \kw{OK}. & Открывается форма предварительного просмотра печати \kw{Форма 39}. \\ \hline
\nn & Нажать кнопку \kw{Печатать} и направить отчет на нужный принтер. & Отчет распечатан. \\ \hline
\nn & Нажать кнопку \kw{Закрыть}. & Окно предварительного просмотра печати закрывается. \\ \hline
\nn & В главном меню выбрать пункт \mm{Анализ \str Заболеваемость \str Ф.12 \str Взрослые}. & Открывается форма \kw{Форма 12(взрослые)}. \\ \hline
\nn & В полях \dm{Дата начала периода} и \dm{Дата окончания периода} указать первый и последний день текущего месяца. & Заданы параметры формирования отчета. \\ \hline
\nn & Нажать кнопку \kw{OK}. & Открывается форма предварительного просмотра печати \kw{Форма 12(взрослые)}. \\ \hline
\nn & Нажать кнопку \kw{Печатать} и направить отчет на нужный принтер. & Отчет распечатан. \\ \hline
\nn & Нажать кнопку \kw{Закрыть}. & Окно предварительного просмотра печати закрывается. \\ \hline
\nn & В главном меню выбрать пункт \mm{Анализ \str Аналитические отчеты \str Форма 131 \str Сводка обращений по врачам}. & Открывается форма \kw{Сводка по Ф.131 по врачам}. \\ \hline
\nn & В полях \dm{Дата начала периода} и \dm{Дата окончания периода} указать первый и последний день текущего месяца. & Заданы параметры формирования отчета. \\ \hline
\nn & Нажать кнопку \kw{OK}. & Открывается форма предварительного просмотра печати \kw{Сводка по Ф. 131 по врачам}. \\ \hline
\nn & Нажать кнопку \kw{Печатать} и направить отчет на нужный принтер. & Отчет распечатан. \\ \hline
\nn & Нажать кнопку \kw{Закрыть}. & Окно предварительного просмотра печати закрывается. \\ \hline
\end{longtable}
