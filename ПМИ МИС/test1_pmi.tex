\newpage
\section{Проверка соответствия функциональных характеристик ПО}
\subsection{Работа с картотекой пациентов}
\subsubsection{Регистрация нового пациента} \label{new_client}

\textbf{Необходимые начальные условия:} Должны быть выполнены первоначальные настройки справочников в системе.

\textbf{Роли пользователя:} Регистратор, администратор.

\begin{longtable}{|p{1cm}|p{7.5cm}|p{8cm}|}
\caption{Регистрация нового пациента \label{new_client_tbl}}\\
\hline \rule{0pt}{15pt}  \centering \textbf{№ пп} & \centering \textbf{Действие} & \hfil \textbf{Ожидаемый результат} \\ \hline
\endfirsthead
\hline \rule{0pt}{15pt} \centering \textbf{№ пп} & \centering \textbf{Действие} & \hfil \textbf{Ожидаемый результат} \\ \hline
\endhead
\nn & Выбрать пункт меню \mm{Работа \str Обслуживание пациентов}. & 	Открывается форма, содержащая картотеку пациентов. \\ \hline
\nn & Нажать кнопку \kw{Регистрация(F9)}. &	Открывается форма \kw{Регистрационная карточка}  для регистрации нового пациента. \\ \hline
\nn & В открывшейся форме заполнить следующие поля: \newline
\dm{Фамилия}: <<Иванов>>, \newline \dm{Имя}: <<Иван>>, \newline \dm{Отчество}: <<Васильевич>>, \newline \dm{Дата рождения}: <<12.12.1960>>, \newline \dm{Пол}: <<М>>, \newline \dm{ИИН}: <<447812345511>>. & Данные внесены в соответствующие поля. \\ \hline
\nn&  В разделе \kw{Адрес регистрации} в поле \dm{Тип населенного пункта} выбрать из списка <<Город>>, в последующих полях указать адрес регистрации пациента <<г. Степногорск, мкр. 2, дом 55>>, заполнив соответствующие поля. & Внесена информация об адресе регистрации пациента. \\ \hline
\nn& Нажать кнопку с изображением стрелки вниз напротив заголовка подраздела \kw{Адрес проживания}. 	& Данные адреса прописки скопированы в соответствующие поля подраздела \kw{Адрес проживания}. \\ \hline
\nn & Выбрать из списка тип документа, удостоверяющего личность - <<Паспорт>>, в поле \dm{Серия} указать <<5555>>, в поле \dm{Номер} - <<345675>>.  & Данные внесены в соответствующие поля. \\ \hline
\nn  & Перейти на вкладку \kw{Соц. статус}. &	Открывается вкладка \kw{Соц. статус}. \\ \hline
\nn \label{ss_tbl} &  Двойным щелчком левой кнопки мыши активировать поле \dm{Класс} в таблице и выбрать из списка значение <<Социальный статус>>,  в той же строке двойным щелчком мыши активировать список в поле \dm{Тип} и выбрать значение <<Работающий>>, дважды щелкнуть левой кнопкой мыши в поле \dm{Дата начала} активной строки.  &	В таблицу статусов добавлена строка о социальном статусе <<Работающий>>. В качестве даты начала действия статуса установлена текущая дата. \\ \hline
\nn & Перейти на следующую строку таблицы и аналогично п. \ref{ss_tbl} заполнить поля: \newline
\dm{Класс}: <<Инвалидность>>, \newline
\dm{Тип}: <<Инвалиды III группы>>, \newline
\dm{Дата начала}: <<01.05.2013>>.	& В таблицу статусов добавлены данные об инвалидности.\\ \hline
\nn & Выделить строку <<Инвалидность>> в таблице \kw{Соц.статус}, щелкнув по ней левой кнопкой мыши. В нижней части формы в поле \dm{Документ, подтверждающий соц.статус} выбрать значение <<Справка МСЭ>> из справочника и ввести следующие значения в поля: \newline
\dm{Серия}: <<A1>>, \newline
\dm{Номер}: <<1554>>, \newline
\dm{Дата}: <<30.04.2013>>, \newline
\dm{Выдан}: <<Степногорская ЦГБ>>. & Данные внесены в соответствующие поля. \\ \hline 
\nn & Перейти на вкладку \kw{Прикрепление}. & Открывается вкладка \kw{Прикрепление}. \\ \hline
\nn & Двойным щелчком левой кнопки мыши активировать справочник в поле \dm{Тип} и выбрать значение <<территориальное>>, дважды щелкнуть левой кнопкой мыши в поле \dm{ЛПУ} активной строки и выбрать значение соответствующее Степногорской ЦГБ. & Внесены данные о прикреплении к ЛПУ, в поле \dm{Дата прикрепления} активной строки автоматически внесена текущая дата. \\ \hline 
\nn & Перейти на вкладку \kw{Занятость}. & Открывается вкладка \kw{Занятость}. \\ \hline
\nn & В верхнем поле ввести значение <<Школа №1>>, в поле \dm{Должность} указать <<Учитель>>.	& Введены данные о месте работы пациента. \\ \hline
\nn & В таблице \kw{Вредность} активировать поле \dm{Вредность} новой строки и выбрать из справочника <<Работа в образовательных организациях всех видов>>, в поле \dm{Стаж} указать значение <<20>>. & Данные о вредностях зарегистрированы. \\ \hline
\nn & В таблице \kw{Фактор} двойным щелчком левой кнопки мыши активировать список значений и выбрать <<4.3 Перенапряжение голосового аппарата, обусловлнное профессионаьной деятельностью>>. & Данные о вредных факторах зарегистрированы. \\ \hline
\nn & Перейти на вкладку \kw{Особенности}. & Открывается вкладка \kw{Особенности}.\\ \hline
\nn & В таблице \kw{Группа крови и резус-фактор} дважды щелкнуть левой клавишей мыши в ячейке \dm{Группа крови} и выбрать значение <<1+ | 0(I)Rh+>>, дважды щелкнуть левой кнопкой мыши в ячейке \dm{Дата установления} и указать дату <<01.05.2014>>. & Данные о группе крови пациента внесены. В поле \dm{Врач} записи о группе крови автоматически подставлено имя текущего пользователя. \\ \hline
\nn & В таблице \kw{Аллергия} двойным щелчком левой кнопки мыши активизировать ячейку \dm{Наименование вещества} и ввести в нее значение <<тополиный пух>>, в поле \dm{Степень} дважды щелкнуть левой кнопкой мыши и выбрать из списка <<высокая>>, дважды щелкнуть левой кнопкой мыши в поле \dm{Дата установления}. & Данные об особенностях пациента зарегистрированы. В поле \dm{Дата установления} указана текущая дата. \\ \hline
\nn & Перейти на вкладку \kw{Прочее}. & Открывается вкладка \kw{Прочее}. \\ \hline
\nn & В таблице \kw{Контакты} дважды щелкнуть левой кнопкой мыщи в ячейке \dm{Тип} и выбрать значение <<домашний телефон>>, в ячейке \dm{Номер} ввести <<42-10-10>>, в ячейке \dm{Примечание} указать <<Дочь Маргарита Владимировна>>. В следующей строке в ячейке \dm{Тип} выбрать значение <<мобильный телефон>>, в ячейке \dm{Номер} ввести <<89053220202>>. & Зарегистрированы контактные данные пациента. \\ \hline
\nn & Нажать кнопку \kw{Подтвердить}. & Текущая форма закрывается. Регистрационная карточка пациента сохраняется в БД. \\ \hline
\end{longtable}

\subsubsection{Поиск пациентов в картотеке} \label{find_client}

\textbf{Необходимые начальные условия:} Должен быть выполнен п. \ref{new_client}

\textbf{Роли пользователя:} Регистратор, администратор.

\setcounter{nnn}{0}
\begin{longtable}{|p{1cm}|p{7.5cm}|p{8cm}|}
\caption{Поиск пациентов \label{find_client_tbl}}\\
\hline \rule{0pt}{15pt}  \centering \textbf{№ пп} & \centering \textbf{Действие} & \hfil \textbf{Ожидаемый результат} \\ \hline
\endfirsthead
\hline \rule{0pt}{15pt} \centering \textbf{№ пп} & \centering \textbf{Действие} & \hfil \textbf{Ожидаемый результат} \\ \hline
\endhead
\nn & На панели \kw{Фильтр} установить флажок в поле \dm{Фамилия} и ввести в ставшее доступным поле значение <<Иван>>. & Указана часть фамилии пациента для поиска. \\ \hline
\nn & Нажать клавишу Enter в поле \dm{Фамилия} либо кнопку \kw{Применить}, расположенную в нижней части панели фильтрации. & Выполнена фильтрация списка пациентов. Отобраны пациенты, фамилия которых начинается на <<Иван>>. В списке присутствует ранее зарегистрированный пациент Иванов Иван Васильевич, 12.12.1960. \\ \hline
\nn & Установить флажок в поле \dm{Д. рожд.} и ввести в ставшее доступным поле значение <<12.12.1960>>. & В параметры поиска добавлена дата рождения пациента. \\ \hline
\nn & Не покидая поля \dm{Д.рожд.} нажать клавишу Enter на клавиатуре либо нажать кнопку \kw{Применить}, расположенную в нижней части панели фильтрации. & Выполнена фильтрация списка пациентов. Отобраны пациенты, фамилия которых начинается на <<Иван>> и дата рождения = 12.12.1960. В списке присутствует ранее зарегистрированный пациент Иванов Иван Васильевич, 12.12.1960. \\ \hline
\nn & На панели \kw{Фильтр} изменить значение в поле \dm{Д.рожд.} на <<12.12.1961>>. & Изменена дата рождения пациента в параметрах поиска. \\ \hline
\nn & Нажать клавишу Enter в поле \dm{Д. рожд.} либо кнопку \kw{Применить}, расположенную в нижней части панели фильтрации. & Выполнена фильтрация списка пациентов. Отобраны пациенты, фамилия которых начинается на <<Иван>> и дата рождения = 12.12.1961. Ранее зарегистрированный пациент Иванов Иван Васильевич, 12.12.1960 отсутствует в списке. Если не найдено ни одного пациента, появляется диалоговое окно с предложением зарегистрировать нового пациента.\\ \hline
\nn & Нажать кнопку \kw{Сбросить}, расположенную в нижней части панели фильтрации. & Все поля на панели фильтрации очищены. Отображается польный список пациентов.\\ \hline
\nn & Установить флажок в поле \dm{Документ} и ввести в последнее из ставших доступными полей значение <<345675>>. & Внесен номер документа для поиска. \\ \hline
\nn & В поле для ввода номера документа нажать клавишу Enter на клавиатуре либо нажать кнопку \kw{Применить}, расположенную в нижней части панели фильтрации. & Выполнена фильтрация списка пациентов. Отобраны пациенты, для которых зарегистрированы документы с номером <<345675>> любой серии, любого типа. В списке присутствует ранее зарегистрированный пациент Иванов Иван Васильевич, 12.12.1960. \\ \hline
\nn & Установить флажок в поле \dm{Код} и ввести в ставшее доступным поле значение <<8>>. & Введен код документа для поиска. Все остальные поля поиска на панели фильтрации автоматически ыли очищены. \\ \hline
\nn & Нажать клавишу Enter в поле \dm{Код} или кнопку \kw{Применить}, расположенную в нижней части панели фильтрации.
 & Выполнен поиск пациента с кодом <<8>>. Если пациент найден, то он отобразится в списке пациентов единственной записью. Если пациент с таким кодом отсутствует в БД, то появится диалоговое окно с предложением зарегистрировать нового пациента. \\ \hline
\end{longtable}

\subsubsection{Редактирование карточки пациента} \label{edt_client}

\textbf{Необходимые начальные условия:} Должен быть выполнен п. \ref{new_client}

\textbf{Роли пользователя:} Регистратор, администратор.

\setcounter{nnn}{0}
\begin{longtable}{|p{1cm}|p{7.5cm}|p{8cm}|}
\caption{Редактирование карточки пациента \label{new_client_tbl}}\\
\hline \rule{0pt}{15pt}  \centering \textbf{№ пп} & \centering \textbf{Действие} & \hfil \textbf{Ожидаемый результат} \\ \hline
\endfirsthead
\hline \rule{0pt}{15pt} \centering \textbf{№ пп} & \centering \textbf{Действие} & \hfil \textbf{Ожидаемый результат} \\ \hline
\endhead
\nn & Найти пациента Иванов Иван Васильевич, 12.12.1960 (см. п. \ref{find_client}). &  Пациент отображается в списке пациентов. \\ \hline
\nn & Установить курсор на запись <<Иванов Иван Васильевич, 12.12.1960>> и нажать клавишу F4 на клавиатуре или кнопку \kw{Редактировать (F4)} в нижней части формы. & Открывается форма \kw{Регистрационная карточка}, содержащая ранее внесенные данные пациента (см. п. \ref{new_client}) \\ \hline
\nn & В поле \dm{Дата рождения} изменить значение на <<12.12.1961>>.& Изменена дата рождения пациента. \\ \hline
\nn & В подразделе \kw{Документ} выбрать тип <<Паспорт РК>>, в поле \dm{Серия} указать <<А1>>, в поле \dm{Номер} -- значение <<111222>>. & Изменены данные о документе, удостоверяющем личность пациента. \\ \hline
\nn & Перейти на вкладку \kw{Документы} и просмотреть состав документов пациента. & В списке на вкладке \kw{Документы} присутствуют данные о ранее зарегистрированном удостоверении личности и текущем документе. \\ \hline
\nn & Перейти на вкладку \kw{Занятость}. & Осуществлен переход на вкладку \kw{Занятость}. \\ \hline
\nn & В таблице \kw{Факторы} щелкнуть правой кнопкой мыши по записи <<4.3. Перенапряжение голосового аппарата, ...>> и в появившемся контекстном меню выбрать пункт \kw{Удалить текущую строку}. & Удалена запись о вредных факторах пациента.\\ \hline
\nn & Перейти на вкладку \kw{Особенности}. & Осуществлен переход на вкладку \kw{Особенности}. \\ \hline
\nn & В таблице \kw{Аллергия} в новой строке в ячейке \dm{Наименование вещества} ввести <<Цветочная пыльца>>, в ячейке \dm{Степень} активировать список двойным щелчком левой кнопки мыши и выбрать значение <<Средняя>>. & Добавлена запись об аалергических реакциях пациента.\\ \hline
\nn & Перейти на вкладку \kw{Прочее}. & Осуществлен переход на вкладку \kw{Прочее}. \\ \hline
\nn & В строке <<мобильный телефон>> ввести в ячейку \dm{Номер} значение <<89260989988>>.& Изменен номер мобильного телефона пациента.\\ \hline
\nn & Нажать кнопку \kw{Подтвердить} в правом нижнем углу формы. & Изменения персональных данных пациента сохраняются в БД. Текущая форма закрывается. \\ \hline
\nn & Снова нажать клавишу F4 на клавиатуре или кнопку \kw{Редактировать (F4)} в нижней части формы. & Открывается форма \kw{Регистрационная карточка}, содержащая ранее внесенные данные пациента. \\ \hline
\nn & Уедиться, что все внесенные ранее изменения отражены в регистрационной карточке пациента. & Ранее внесенные изменения отражены в регистрационной карточке пациента.\\ \hline
\nn & Нажать кнопку \kw{Закрыть}. & Текущая форма закрывается.\\ \hline
\end{longtable}