\subsection{Регистрация поступления пациента в стационар} \label{rep_st}
\subsubsection{Регистрация поступления пациента в приемное отделение} \label{reg_po_st}

\textbf{Необходимые начальные условия:} Должны быть выполнены первоначальные настройки справочников в системе.

\textbf{Роли пользователя:} Медсестра приемного отделения, Администратор.

\setcounter{nnn}{0}
\begin{longtable}{|p{1cm}|p{7.5cm}|p{8cm}|}
\caption{Регистрация поступления пациента в приемное отделение \label{reg_po_st_tbl}}\\
\hline \rule{0pt}{15pt}  \centering \textbf{№ пп} & \centering \textbf{Действие} & \hfil \textbf{Ожидаемый результат} \\ \hline
\endfirsthead
\hline \rule{0pt}{15pt} \centering \textbf{№ пп} & \centering \textbf{Действие} & \hfil \textbf{Ожидаемый результат} \\ \hline
\endhead
\nn & В главном меню выбрать пункт \mm{Работа \str Обслуживание пациентов}. & Открывается форма, содержащая картотеку пациентов. \\ \hline
\nn & На панели \kw{Фильтр} в правой части экрана установить флажок \dm{Фамилия} и в ставшее активным поле ввести <<Калинина>>, установить флажок \dm{Д.рожд.} и в ставшее активным поле ввести <<20.06.1980>>. & Заданы параметры поиска пациента. \\ \hline
\nn & Нажать кнопку \kw{Применить} на панели \kw{Фильтр}. & Отображается список пациентов, найденных по заданным параметрам поиска. \\ \hline
\nn & Если пациент не найден, необходимо зарегистрировать его аналогично п. \ref{new_client} При регистрации указать следующие данные: \newline \dm{Фамилия}: <<Калинина>>, \newline \dm{Имя}: <<Динара>>, \newline \dm{Отчество}: <<Павловна>>, \newline \dm{Дата рождения}: <<20.06.1980>>, \newline \dm{Пол}: <<Ж>>, \newline \dm{Тип населенного пункта}: <<Город>>, \newline \dm{Адрес регистрации}: <<г.Стапногорск, 4 мкр., д. 48>>, \newline \dm{Адрес проживания}: аналогичен адресу регистрации, \newline \dm{Документ}: <<Удостоверение личности>>, \dm{Серия}: <<7711>>, \dm{номер}: <<45897823>>, \newline \dm{Социальный статус}: <<Работающий>>, \dm{Занятость}: <<Хлебозавод №1>>. & Данные пациента сохранены. Осуществлен возврат к форме, содержащей картотеку пациентов. \\ \hline 
\nn & Установить курсор на записи о пациенте <<Калинина Динава Павловна, 20.06.1980>>. & Курсор установлен на указанном пациенте. В верхней части отображается краткая информация о выбранном пациенте. \\ \hline
\nn & Перейти на вкладку \kw{Обращение}. & Осуществлен переход на вкладку \kw{Обращение}. \\ \hline
\nn \label{n3} & Нажать клавишу F9 на клавиатуре либо кнопку \kw{Новый (F9)} в нижней части формы. & Открывается форма \kw{Новое обращение} для пациента <<Калинина Динара Павловна, 20.06.1980>>. В перечисленных полях указаны следющие значения: \newline \dm{Организация}: <<ГКП на ПХВ <<Степногорская центральная городская больница>>>>, \newline \dm{Тип обращения}: <<Стационар>>,  \newline \dm{Дата начала}: текущие дата и время, \newline \dm{Дата выполнения}: не заполнено, \newline \dm{Порядок}: <<Плановый>>, \newline \dm{Первичность}: <<Первичный>>. \\ \hline
\nn \label{n4} & В перечисленных полях выбрать из списка следующие значения: \newline \dm{Лечащий врач}: <<Арапова М.В., Акушер-гинеколог>>, \newline \dm{Источник финансирования}: <<Платные услуги>>. & В перечисленные ниже поля автоматически подставлены следующие значения: \newline \dm{Договор}: номер договора на платные услуги, \newline \dm{Тип события}: <<Стационарное лечение (платные услуги)>>. \\ \hline
\nn & Нажать кнопку \kw{Создать}. & Открывается форма \kw{Планирование: Пациент: Калинина Динара Павловна...}.\\ \hline
\nn & Убедиться, что в строке <<Поступление>> в нижней части формы установлен флажок в ячейке \dm{Включить} (при отсутствии - установить). & Флажок \dm{Включить} установлен в строке <<Поступление>>.\\ \hline
\nn & Нажать кнопку \kw{ОК}. & Текущая форма закрывается. Открывается форма \kw{Стационарное лечение (платные услуги)}. Все поля, заполненные в пп. \ref{n3}--\ref{n4} содержат соответствующие значения. \\ \hline
\nn & В левой части формы выбрать раздел \kw{Движение пациента}, щелкнув по нему левой кнопкой мыши. & Осуществляется переход в раздел \kw{Движение пациента} стационарного обращения. \\ \hline
\nn & Дважды щелкнуть левой кнопкой мыши по строке <<Поступление>> в правой части формы. & Открывается форма \kw{Калинина Динара Павловна - Поступление}. В поле \dm{Назначено} верхней части формы указаны текущие дата и время, поле \dm{Состояние} установлено <<Закончено>>, в поле \dm{Исполнитель} указан текущий пользователь. \\ \hline
\nn & В ячейке \dm{Значение} табличной части формы указать следующие данные в соответствующих строках: \newline \dm{ИБ переоткрыта}: <<нет>> (выбирается из списка), \newline \dm{Отделение поступления}: <<Приемное отделение>> (выбирается из дерева структуры ЛПУ), \newline \dm{Кем доставлен}: <<Самостоятельно>> (выбирается из списка), \newline \dm{Вид транспортировки}: <<Может идти>> (выбирается из списка), \newline \dm{Состояние при поступлении}: <<удовлетворительное>> (выбирается из списка), \newline \dm{Диагноз приемного отделения} (можно ввести с клавиатуры <<O13>> или выбрать из справочника, раскрывая соответствующие ветви нажатием кнопки <<$+$>>), \newline \dm{Доставлен в стционар от начала заболевания}: <<в течении 7-24 часов>> (выбирается из списка), \newline \dm{Госпитализирован по поводу данного заболевания в текущем году}: <<первично>> (выбирается из списка), & Данные о поступлении внесены в соответствующие ячейки. \\ \hline
 & \dm{Экстренность}: <<В плановом порядке>> (выбирается из списка), \newline \dm{Направлен в отделение}: <<Акушерско-гинекологическое отделение>> (выбирается из дерева структуры ЛПУ), \newline \dm{Профиль койки}: <<Акушерский>> (выбирается из списка), \newline \dm{Тип согласования}: <<С заведующим отделением>> (выбирается из списка), \newline \dm{Цель госпитализации}: <<лечение>> (выбирается из списка), \newline \dm{Код и название направившего ЛПО}: <<ГКП на ПВХ <<Степногорская городская поликлиника>>>> (выбирается из списка), \newline \dm{Диагноз направившего учреждения}: <<О13>>. \newline Остальные ячейки заполнить произвольными значениями или оставить незаполненными. & \\ \hline
\nn & Нажать кнопку \kw{Сохранить}. & Текущая форма закрывается. Данные о поступлении в приемное отделение сохраняются в БД. Осуществляется возврат на форму \kw{Стационарное лечение (платные услуги)}. \\ \hline
\nn & Нажать кнопку \kw{Сохранить}. & Появляется диалоговое окно, содержащее сообщение <<Данные успешно сохранены>>. \\ \hline
\nn & Нажать кнопку \kw{OK}. & Диалоговое окно закрывается. \\ \hline
\nn & Нажать кнопку \kw{Закрыть без сохранения}. & Текущая форма закрывается. Осуществляется возврат на форму, содержащую картотеку пациентов. \\ \hline
\end{longtable}

\subsubsection{Регистрация отказа от госпитализации} \label{cancel_po_st}

\textbf{Необходимые начальные условия:} Должны быть выполнены первоначальные настройки справочников в системе.

\textbf{Роли пользователя:} Медсестра приемного отделения, Администратор.

\setcounter{nnn}{0}
\begin{longtable}{|p{1cm}|p{7.5cm}|p{8cm}|}
\caption{Регистрация отказа от госпитализации\label{cancel_po_st_tbl}}\\
\hline \rule{0pt}{15pt}  \centering \textbf{№ пп} & \centering \textbf{Действие} & \hfil \textbf{Ожидаемый результат} \\ \hline
\endfirsthead
\hline \rule{0pt}{15pt} \centering \textbf{№ пп} & \centering \textbf{Действие} & \hfil \textbf{Ожидаемый результат} \\ \hline
\endhead
\nn & В главном меню выбрать пункт \mm{Работа \str Обслуживание пациентов}. & Открывается форма, содержащая картотеку пациентов. \\ \hline
\nn & На панели \kw{Фильтр} в правой части экрана установить флажок \dm{Фамилия} и в ставшее активным поле ввести <<Петров>>, установить флажок \dm{Д.рожд.} и в ставшее активным поле ввести <<11.09.1950>>. & Заданы параметры поиска пациента. \\ \hline
\nn & Нажать кнопку \kw{Применить} на панели \kw{Фильтр}. & Отображается список пациентов, найденных по заданным параметрам поиска. \\ \hline
\nn & Если пациент не найден, необходимо зарегистрировать его аналогично п. \ref{new_client} При регистрации указать следующие данные: \newline \dm{Фамилия}: <<Петров>>, \newline \dm{Имя}: <<Сергей>>, \newline \dm{Отчество}: <<Андреевич>>, \newline \dm{Дата рождения}: <<11.09.1950>>, \newline \dm{Пол}: <<М>>, \newline \dm{Тип населенного пункта}: <<Город>>, \newline \dm{Адрес регистрации}: <<г.Астана>>, \newline \dm{Адрес проживания}: аналогичен адресу регистрации, \newline \dm{Документ}: <<Паспорт РК>>, \dm{Серия}: <<5503>>, \dm{номер}: <<124845>>, \newline \dm{Социальный статус}: <<Пенсионер>>. & Данные пациента сохранены. Осуществлен возврат к форме, содержащей картотеку пациентов. \\ \hline 
\nn & Установить курсор на записи о пациенте <<Петров Сергей Андреевич, 11.09.1950>>. & Курсор установлен на указанном пациенте. В верхней части отображается краткая информация о выбранном пациенте. \\ \hline
\nn & Перейти на вкладку \kw{Обращение}. & Осуществлен переход на вкладку \kw{Обращение}. \\ \hline
\nn \label{n7} & Нажать клавишу F9 на клавиатуре либо кнопку \kw{Новый (F9)} в нижней части формы. & Открывается форма \kw{Новое обращение} для пациента <<Петров Сергей Андреевич, 11.09.1950>>. В перечисленных полях указаны следющие значения: \newline \dm{Организация}: <<ГКП на ПХВ <<Степногорская центральная городская больница>>>>, \newline \dm{Тип обращения}: <<Стационар>>,  \newline \dm{Дата начала}: текущие дата и время, \newline \dm{Дата выполнения}: не заполнено, \newline \dm{Порядок}: <<Плановый>>, \newline \dm{Первичность}: <<Первичный>>. \\ \hline
\nn \label{n8} & В перечисленных полях выбрать из списка следующие значения: \newline \dm{Лечащий врач}: <<Абдрахманов К.К., Хирург>>, \newline \dm{Источник финансирования}: <<бюджет>>, \newline \dm{Порядок}: <<Экстренный>>. & В перечисленные ниже поля автоматически подставлены следующие значения: \newline \dm{Договор}: номер договора оказания мед.услуг в рамках бюджета, \newline \dm{Тип события}: <<Стационарное лечение (бюджет)>>. \\ \hline
\nn & Нажать кнопку \kw{Создать}. & Открывается форма \kw{Планирование: Пациент: Петров Сергей Андреевич...}.\\ \hline
\nn & Убедиться, что в строке <<Поступление>> в нижней части формы установлен флажок в ячейке \dm{Включить} (при отсутствии - установить). & Флажок \dm{Включить} установлен в строке <<Поступление>>.\\ \hline
\nn & Нажать кнопку \kw{ОК}. & Текущая форма закрывается. Открывается форма \kw{Стационарное лечение (бюджет)}. Все поля, заполненные в пп. \ref{n7}--\ref{n8} содержат соответствующие значения. \\ \hline
\nn & В левой части формы выбрать раздел \kw{Движение пациента}, щелкнув по нему левой кнопкой мыши. & Осуществляется переход в раздел \kw{Движение пациента} стационарного обращения. \\ \hline
\nn & Дважды щелкнуть левой кнопкой мыши по строке <<Поступление>> в правой части формы. & Открывается форма \kw{Петров Сергей Андреевич - Поступление}. В поле \dm{Назначено} верхней части формы указаны текущие дата и время, поле \dm{Состояние} установлено <<Закончено>>, в поле \dm{Исполнитель} указан текущий пользователь. \\ \hline
\nn & В ячейке \dm{Значение} табличной части формы указать следующие данные в соответствующих строках: \newline \dm{ИБ переоткрыта}: <<нет>> (выбирается из списка), \newline \dm{Отделение поступления}: <<Приемное отделение>> (выбирается из дерева структуры ЛПУ), \newline \dm{Кем доставлен}: <<СМП>> (выбирается из списка), \newline \dm{Вид транспортировки}: <<Может идти>> (выбирается из списка), \newline \dm{Травма}: <<ДТП>> (выбирается из списка), \newline \dm{Диагноз приемного отделения} (можно ввести с клавиатуры <<S00.7>> или выбрать из справочника, раскрывая соответствующие ветви нажатием кнопки <<$+$>>), \newline \dm{Доставлен в стционар от начала заболевания}: <<в первые 6 часов>> (выбирается из списка), \newline \dm{Номер наряда СМП}: <<77>> (выбирается из списка), \newline \dm{Направлен в отделение}: <<Приемное отделение>> (выбирается из дерева структуры ЛПУ), & Данные о поступлении внесены в соответствующие ячейки. \\ \hline
 & \dm{Причина отказа в госпитализации}: <<отказ со стороны больного>> (выбирается из списка). \newline Остальные ячейки заполнить произвольными значениями или оставить незаполненными. & \\ \hline
\nn & Нажать кнопку \kw{Сохранить}. & Текущая форма закрывается. Данные о поступлении в приемное отделение сохраняются в БД. Осуществляется возврат на форму \kw{Стационарное лечение (бюджет)}. \\ \hline
\nn & Нажать кнопку \kw{Сохранить}. & Появляется диалоговое окно, содержащее сообщение <<Данные успешно сохранены>>. \\ \hline
\nn & Нажать кнопку \kw{OK}. & Диалоговое окно закрывается. \\ \hline
\nn & Нажать кнопку \kw{Закрыть без сохранения}. & Текущая форма закрывается. Осуществляется возврат на форму, содержащую картотеку пациентов. \\ \hline
\end{longtable}

\subsubsection{Регистрация первичного осмотра пациента в приемном отделении} \label{osmotr_po_st}

\textbf{Необходимые начальные условия:} Должна быть выполнена регистрация пациента в приемном отделении согласно п. \ref{reg_po_st}

\textbf{Роли пользователя:} Дежурный врач, Врач отделения, Администратор.

\setcounter{nnn}{0}
\begin{longtable}{|p{1cm}|p{7.5cm}|p{8cm}|}
\caption{Регистрация первичного осмотра пациента в приемном отделении \label{osmotr_po_st_tbl}}\\
\hline \rule{0pt}{15pt}  \centering \textbf{№ пп} & \centering \textbf{Действие} & \hfil \textbf{Ожидаемый результат} \\ \hline
\endfirsthead
\hline \rule{0pt}{15pt} \centering \textbf{№ пп} & \centering \textbf{Действие} & \hfil \textbf{Ожидаемый результат} \\ \hline
\endhead
\nn & В главном меню выбрать пункт \mm{Работа \str Обслуживание пациентов}. & Открывается форма, содержащая картотеку пациентов. \\ \hline
\nn & На панели \kw{Фильтр} в правой части экрана установить флажок \dm{Фамилия} и в ставшее активным поле ввести <<Калинина>>, установить флажок \dm{Д.рожд.} и в ставшее активным поле ввести <<20.06.1980>>. & Заданы параметры поиска пациента. \\ \hline
\nn & Нажать кнопку \kw{Применить} на панели \kw{Фильтр}. & Отображается список пациентов, найденных по заданным параметрам поиска. \\ \hline
\nn & Установить курсор на записи о пациенте <<Калинина Динара Павловна, 20.06.1980>>. & Курсор установлен на указанном пациенте. В верхней части отображается краткая информация о выбранном пациенте. \\ \hline
\nn & Перейти на вкладку \kw{Обращение}. & Осуществлен переход на вкладку \kw{Обращение}. \\ \hline
\nn & Установить курсор на верхней записи <<Стационарное лечение ...>>. & Курсор установлен на указанной записи. \\ \hline
\nn & Нажать кнопку \kw{Редактировать(F4)} в нижней части формы или клавишу F4 на клавиатуре. & Открывается форма \kw{Стационарное лечение (платные услуги)}.\\ \hline
\nn & В левой части формы выбрать раздел \kw{Медицинские документы}, щелкнув по нему левой кнопкой мыши. & Осуществляется переход в раздел \kw{Медицинские документы} стационарного обращения. \\ \hline
\nn & Нажать кнопку \kw{Создать} в правой верхней части формы. & Открывается форма \kw{Создание действий}. \\ \hline
\nn & Перейти на вкладку \kw{Дерево}. & Осуществлен переход на вкладку \kw{Дерево} на форме \kw{Создание действий}. \\ \hline
\nn & Нажать знак <<$+$>> слева от названия группы: <<1\_3: Стационар>>. & Раскрыт список видов документов для стционара. \\ \hline 
\nn & Выбрать из списка <<1\_3\_00: Осмотр врача приемного отделения>> и дважды щелкнуть по нему левой кнопкой мыши либо перетащить запись мышкой в таблицу \kw{Выбранные действия} в правой части формы. & Действие появилось в таблице \kw{Выбранные действия}. \\ \hline
\nn & Нажать кнопку \kw{ОК}. & Текущая форма закрывается. Открывается форма \kw{Калинина Динара Павловна - Осмотр врача приемного отделения}. \\ \hline
\nn & Последовательно заполнить ячейки \dm{Значение} табличной части формы способом описанным в пп. \ref{n1} -- \ref{n2} раздела \ref{osmotr_pol} & Заполнена таблица в левой части формы. \\ \hline 
\nn & Нажать кнопку \kw{OK}. & Текущая форма закрывается. Осуществляется возврат на форму \kw{Стационарное лечение (платные услуги)}. В списке медицинских документов появляется новая запись.\\ \hline
\nn & Нажать кнопку \kw{Сохранить}. & Появляется диалоговое окно, содержащее сообщение <<Данные успешно сохранены>>. \\ \hline
\nn & Нажать кнопку \kw{OK}. & Диалоговое окно закрывается. \\ \hline
\nn & Нажать кнопку \kw{Закрыть без сохранения}. & Текущая форма закрывается. Осуществляется возврат на форму, содержащую картотеку пациентов. \\ \hline
\end{longtable}

\subsubsection{Печать медицинских документов пациента при поступлении} \label{prn_po_st}

\textbf{Необходимые начальные условия:} Должны быть выполнены пп. \ref{reg_po_st} и \ref{osmotr_po_st} 

\textbf{Роли пользователя:} Медсестра приемного отделения, Администратор.

\setcounter{nnn}{0}
\begin{longtable}{|p{1cm}|p{7.5cm}|p{8cm}|}
\caption{Печать медицинских документов пациента при поступлении \label{prn_ po_st_tbl}}\\
\hline \rule{0pt}{15pt}  \centering \textbf{№ пп} & \centering \textbf{Действие} & \hfil \textbf{Ожидаемый результат} \\ \hline
\endfirsthead
\hline \rule{0pt}{15pt} \centering \textbf{№ пп} & \centering \textbf{Действие} & \hfil \textbf{Ожидаемый результат} \\ \hline
\endhead
\nn & В главном меню выбрать пункт \mm{Работа \str Обслуживание пациентов}. & Открывается форма, содержащая картотеку пациентов. \\ \hline
\nn & На панели \kw{Фильтр} в правой части экрана установить флажок \dm{Фамилия} и в ставшее активным поле ввести <<Калинина>>, установить флажок \dm{Д.рожд.} и в ставшее активным поле ввести <<20.06.1980>>. & Заданы параметры поиска пациента. \\ \hline
\nn & Нажать кнопку \kw{Применить} на панели \kw{Фильтр}. & Отображается список пациентов, найденных по заданным параметрам поиска. \\ \hline
\nn & Установить курсор на записи о пациенте <<Калинина Динара Павловна, 20.06.1980>>. & Курсор установлен на указанном пациенте. В верхней части отображается краткая информация о выбранном пациенте. \\ \hline
\nn & Перейти на вкладку \kw{Обращение}. & Осуществлен переход на вкладку \kw{Обращение}. \\ \hline
\nn & Установить курсор на верхней записи <<Стационарное лечение ...>>. & Курсор установлен на указанной записи. \\ \hline
\nn & Нажать кнопку \kw{Редактировать(F4)} в нижней части формы или клавишу F4 на клавиатуре. & Открывается форма \kw{Стационарное лечение (платные услуги)}.\\ \hline
\nn \label{n5} & Нажать кнопку \kw{Печать} в средней части формы и из появившегося меню выбрать название печатной формы <<Мед. карта стационарного больного>>. & Открывается окно предварительного просмотра печатной формы. \\ \hline
\nn & Нажать кнопку \kw{Печатать} и направить документ на нужный принтер. & Выполнена печать обложки карты стационарного больного. \\ \hline
\nn \label{n6} & Нажать на кнопку \kw{Закрыть}. & Окно пердварительно просмотра печатной формы закрывается. \\ \hline
\nn & Нажать кнопку \kw{Печать} и из появившегося меню выбрать название печатной формы <<Стат. карта выбывшего из стационара-титульная страница>>. & Открывается окно предварительного просмотра печатной формы. \\ \hline
\nn & Нажать кнопку \kw{Печатать} и направить документ на нужный принтер. & Выполнена печать первой страницы стат. карты стационарного больного. \\ \hline
\nn & Нажать на кнопку \kw{Закрыть}. & Окно пердварительно просмотра печатной формы закрывается. \\ \hline
\nn & При необходимости повторить пп. \ref{n5}--\ref{n6} для других печатных форм. & Выполнена печать медицинских документов при поступлении пациента. \\ \hline
\end{longtable}

\subsubsection{Формирование и печать журнала приемного отделения} \label{list_po_st}

\textbf{Необходимые начальные условия:} Должны быть выполнены пп. \ref{reg_po_st} -- \ref{cancel_po_st} 

\textbf{Роли пользователя:} Медсестра приемного отделения, Администратор.

\setcounter{nnn}{0}
\begin{longtable}{|p{1cm}|p{7.5cm}|p{8cm}|}
\caption{Формирование и печать журнала приемного отделения \label{list_ po_st_tbl}}\\
\hline \rule{0pt}{15pt}  \centering \textbf{№ пп} & \centering \textbf{Действие} & \hfil \textbf{Ожидаемый результат} \\ \hline
\endfirsthead
\hline \rule{0pt}{15pt} \centering \textbf{№ пп} & \centering \textbf{Действие} & \hfil \textbf{Ожидаемый результат} \\ \hline
\endhead
\nn & В главном меню выбрать пункт \mm{Работа \str Стационарный монитор}. & Открывается форма \kw{Стационарный монитор}. \\ \hline
\nn & Перейти на вкладку \kw{Поступили}. & Осуществлен переход на вкладку \kw{Поступили}. \\ \hline
\nn & В левой части формы на панели \kw{Фильтр} в поле \dm{Начало} указать текущую дату (либо дату выполнения п. \ref{reg_po_st}, если он выполнялся ранее), в поле \dm{Поступили} выбрать из списка <<в приемное отделение>>. & Установлены параметры фильтрации обращений. \\ \hline
\nn & Нажать кнопку \kw{Применить} в правом нижнем углу панели \kw{Фильтр}. & Выполнена фильтрация обращений. В списке в правой части формы присутствуют запись <<Калинина Динара Павловна, Ж, 20.06.1980>> и <<Петров Сергей Андреевич, M, 11.09.1950>>. \\ \hline
\nn & Нажать кнопку \kw{Печать} и в появившемся меню выбрать пункт <<Журнал>>. & Открывается окно предварительного просмотра печатной формы журнала учета приема больных и отказов в госпитализации. Для пациента <<Петров Сергей Андреевич>> сделана пометка об отказе от госпитализации. \\ \hline
\nn & Нажать кнопку \kw{Печатать} и направить документ на нужный принтер. & Выполнена печать журнала приемного отделения за выбранный период. \\ \hline
\nn & Нажать на кнопку \kw{Закрыть}. & Окно пердварительно просмотра печатной формы закрывается. \\ \hline
\nn & Нажать на кнопку \kw{Закрыть}. & Форма \kw{Стационарный монитор} закрывается. \\ \hline
\end{longtable}



