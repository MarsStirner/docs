\subsection{Печать медицинских документов и формирование отчетов в стационаре} \label{rep_st}
\subsubsection{Печать медицинских документов пациента} \label{prn_st}

\textbf{Необходимые начальные условия:} Должны быть выполнены тесты из подраздела \ref{lecp_st} Должен быть выполнен пункт \ref{vip_epik_st} Должна быть открыта форма \kw{Стационарное лечение (платные услуги)} для данного обращения.

\textbf{Роли пользователя:} Врач отделения, Администратор.

\setcounter{nnn}{0}
\begin{longtable}{|p{1cm}|p{7.5cm}|p{8cm}|}
\caption{Печать медицинских документов пациента \label{prn_st_tbl}}\\
\hline \rule{0pt}{15pt}  \centering \textbf{№ пп} & \centering \textbf{Действие} & \hfil \textbf{Ожидаемый результат} \\ \hline
\endfirsthead
\hline \rule{0pt}{15pt} \centering \textbf{№ пп} & \centering \textbf{Действие} & \hfil \textbf{Ожидаемый результат} \\ \hline
\endhead
\nn & В левой части формы выбрать раздел \kw{Основная информация}, щелкнув по нему левой кнопкой мыши. & Осуществляется переход в раздел \kw{Основная информация} стационарного обращения. \\ \hline
\nn & Нажать кнопку \kw{Печать} в средней части формы и в раскрывшемся меню выбрать пункт <<Все листы динамического наблюдения текущего врача>>. & Открывается форма предварительного просмотра печати \kw{Все листы динамического наблюдения текущего врача}. \\ \hline
\nn & Нажать кнопку \kw{Печатать} и направить документ на нужный принтер. & Выполнена печать медицинского документа. \\ \hline
\nn & Нажать на кнопку \kw{Закрыть}. & Окно пердварительно просмотра печатной формы закрывается. \\ \hline
\nn & Повторить пп. 2--4 для других печатных форм. & Выполнена печать медицинских документов по данному стационарному обращению. \\ \hline
\nn & В левой части формы выбрать раздел \kw{Медицинские документы}, щелкнув по нему левой кнопкой мыши. & Осуществляется переход в раздел \kw{Медицинские документы} обращения. \\ \hline
\nn & Дважды щелкнуть левой кнопкой мыши по записи <<Заключительный эпикриз>>. & Открывается форма \kw{Калинина Динара Павловна - Заключительный эпикриз}. \\ \hline
\nn & Нажать кнопку \kw{Печать} и из появившегося меню выбрать <<Выписной эпикриз>>. & Открывается окно предварительного просмотра печатной формы. \\ \hline
\nn & Нажать кнопку \kw{Печатать} и направить документ на нужный принтер. & Выполнена печать выписного эпикриза пациента. \\ \hline
\nn & Нажать на кнопку \kw{Закрыть}. & Окно пердварительно просмотра печатной формы закрывается. \\ \hline
\nn & Нажать на кнопку \kw{Закрыть}. & Текущая форма закрывается. \\ \hline
\nn & В левой части формы выбрать раздел \kw{Диагностические и лабораторные исследования}, щелкнув по нему левой кнопкой мыши. & Осуществляется переход в раздел \kw{Диагностические и лабораторные исследования} обращения. \\ \hline
\nn & Щелкнуть правой кнопкой мыши по любой из записей в таблице и в контекстном меню выбрать <<Печать>> \str <<Результаты всех анализов пациента>>. & Открывается форма предварительного просмотра печатной формы. \\ \hline
\nn & Нажать кнопку \kw{Печатать} и направить документ на нужный принтер. & Выполнена печать результатов анализов пациента. \\ \hline
\nn & Нажать на кнопку \kw{Закрыть}. & Окно пердварительно просмотра печатной формы закрывается. \\ \hline
\nn & В левой части формы выбрать раздел \kw{Лечение}, щелкнув по нему левой кнопкой мыши. & Осуществляется переход в раздел \kw{Лечение} обращения. \\ \hline
\nn & Нажать кнопку \kw{Лист назначений} в правой части формы. & Открывается форма \kw{Лист назначений}.  \\ \hline
\nn & Нажать кнопку \kw{Печать}. & Открывается форма предварительного просмотра печатной формы \kw{Лист назначений}. \\ \hline
\nn & Нажать кнопку \kw{Печатать} и направить документ на нужный принтер. & Выполнена печать листа назначений пациента. \\ \hline
\nn & Нажать на кнопку \kw{Закрыть}. & Окно пердварительно просмотра печатной формы закрывается. \\ \hline
\nn & Нажать на кнопку \kw{Закрыть}. & Текущая форма закрывается. \\ \hline
\end{longtable}

\subsubsection{Получение отчетов по стационару} \label{rep1_st}

\textbf{Необходимые начальные условия:} Должны быть выполнены тесты из подраздела \ref{lecp_st} Должен быть выполнен пункт \ref{dv_vip_st} 

\textbf{Роли пользователя:} Медстатистик, Врач отделения, Администратор.

\setcounter{nnn}{0}
\begin{longtable}{|p{1cm}|p{7.5cm}|p{8cm}|}
\caption{Получение отчетов по стационару \label{rep1_st_tbl}}\\
\hline \rule{0pt}{15pt}  \centering \textbf{№ пп} & \centering \textbf{Действие} & \hfil \textbf{Ожидаемый результат} \\ \hline
\endfirsthead
\hline \rule{0pt}{15pt} \centering \textbf{№ пп} & \centering \textbf{Действие} & \hfil \textbf{Ожидаемый результат} \\ \hline
\endhead
\nn & В главном меню выбрать пункт \mm{Анализ \str Аналитические отчеты \str Форма 016\slashу-02}. & Открывается диалоговое окно \kw{Введите начальную дату}. \\ \hline
\nn & В поле даты ввести или выбрать из календря первый день текущего месяца и нажать кнопку \kw{ОК}. & Открывается диалоговое окно \kw{Введите конечную дату}.\\ \hline
\nn & В поле даты ввести или выбрать из календря последний день текущего месяца и нажать кнопку \kw{ОК}. & После непродолжительного ожидания открывается форма предварительного просмотра \kw{Форма 016\slashу-02}.\\ \hline
\nn & Нажать кнопку \kw{Печатать} и направить отчет на нужный принтер. & Отчет распечатан. \\ \hline
\nn & Нажать кнопку \kw{Закрыть}. & Окно предварительного просмотра печати закрывается. \\ \hline
\nn & В главном меню выбрать пункт \mm{Анализ \str Аналитические отчеты \str Форма 007}. & Открывается диалоговое окно \kw{Задайте подразделение}. \\ \hline
\nn & Из дерева структуры ЛПУ выбрать <<Хирургическое отделение>> и нажать кнопку \kw{OK}. & Открывается диалоговое окно \kw{Задайте дату окончания учетных суток}.\\ \hline
\nn & В поле даты ввести или выбрать из календря следующий день и нажать кнопку \kw{ОК}. & Открывается диалоговое окно \kw{Задайте время больничных суток}.\\ \hline
\nn & Ввести в поле <<8:00>> и нажать кнопку \kw{OK}. & Открывается диалоговое окно \kw{Задайте профили коек}.\\ \hline
\nn & Выбрать из списка <<Хирургический взрослые>> и нажать кнопку \kw{OK}. & После непродолжительного ожидания открывается форма предварительного просмотра листа движения пациентов \kw{Форма 007}.\\ \hline
\nn & Нажать кнопку \kw{Печатать} и направить отчет на нужный принтер. & Отчет распечатан. \\ \hline
\nn & Нажать кнопку \kw{Закрыть}. & Окно предварительного просмотра печати закрывается. \\ \hline
\nn & В главном меню выбрать пункт \mm{Анализ \str Стационар \str Форма 014 \str Состав больных в стационаре, сроки и исходы лечения \str Взрослые (18 лет и старше)}. & Открывается форма \kw{параметры отчета}. \\ \hline
\nn & В полях \dm{Дата начала периода} и \dm{Дата окончания периода} указать первый и последний день текущего месяца. Остальные поля оставить заполненными по умолчанию. & Заданы параметры формирования отчета. \\ \hline
\nn & Нажать кнопку \kw{OK}. & Открывается форма предварительного просмотра печати \kw{Форма 014}. \\ \hline
\nn & Нажать кнопку \kw{Печатать} и направить отчет на нужный принтер. & Отчет распечатан. \\ \hline
\nn & Нажать кнопку \kw{Закрыть}. & Окно предварительного просмотра печати закрывается. \\ \hline
\nn & В главном меню выбрать пункт \mm{Работа \str Стационарный монитор}. & Открывается форма \kw{Стационарный монитор}. \\ \hline
\nn & Перейти на вкладку \kw{Выбыли}. & Осуществлен переход на вкладку \kw{Выбыли}. В правой части формы имеется запись о пациенте <<Калинина Динара Павловна, 20.06.1980>>. \\ \hline
\nn & Нажать кнопку \kw{Печать} в правом нижнем углу формы и в появившемся меню выбрать пункт <<Сводка>>. & Открывается форма предварительного просмотра сводки по выбывшим пациентам за текущую дату. \\ \hline
\nn & Нажать кнопку \kw{Печатать} и направить отчет на нужный принтер. & Отчет распечатан. \\ \hline
\nn & Нажать кнопку \kw{Закрыть}. & Окно предварительного просмотра печати закрывается. \\ \hline
\nn & Нажать кнопку \kw{Закрыть}. & Текущая форма закрывается. \\ \hline
\end{longtable}




