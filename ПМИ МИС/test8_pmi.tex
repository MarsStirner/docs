\subsection{Выписка пациента из стационара} \label{vip_st}
\subsubsection{Регистрация выписки из отделения} \label{dv_vip_st}

\textbf{Необходимые начальные условия:} Должен быть выполнен пункт \ref{reg_otd2_st}

\textbf{Роли пользователя:} Медсестра отделения, Администратор.

\setcounter{nnn}{0}
\begin{longtable}{|p{1cm}|p{7.5cm}|p{8cm}|}
\caption{Регистрация поступления пациента в клиническое отделение \label{dv_vip_st_tbl}}\\
\hline \rule{0pt}{15pt}  \centering \textbf{№ пп} & \centering \textbf{Действие} & \hfil \textbf{Ожидаемый результат} \\ \hline
\endfirsthead
\hline \rule{0pt}{15pt} \centering \textbf{№ пп} & \centering \textbf{Действие} & \hfil \textbf{Ожидаемый результат} \\ \hline
\endhead
\nn & В главном меню выбрать пункт \mm{Работа \str Стационарный монитор}. & Открывается форма \kw{Стационарный монитор}. \\ \hline
\nn & Перейти на вкладку \kw{Присутствуют}. & Осуществлен переход на вкладку \kw{Присутствуют}. \\ \hline
\nn & В правом верхнем углу формы в дереве структуры ЛПУ установить курсор на строке <<Хирургическое отделение>>, раскрывая необходимые ветви дерева нажатим кнопки <<$+$>> слева от ее названия. & Курсор установлен на указанном отделении. В списке в правой части формы присутствует запись <<Калинина Динара Павловна, Ж, 20.06.1980>>. \\ \hline
\nn & Дважды щелкнуть левой кнопкой мыши по записи <<Калинина Динара Павловна, Ж, 20.06.1980>> либо щелкнуть по ней правой кнопкой мыши и в появившемся контекстном меню выбрать пункт \kw{Открыть обращение}. & Открывается форма \kw{Стационарное лечение (платные услуги)} для выбранного пациента. \\ \hline
\nn & В левой части формы выбрать раздел \kw{Движение пациента}, щелкнув по нему левой кнопкой мыши. & Осуществляется переход в раздел \kw{Движение пациента} стационарного обращения. \\ \hline
\nn & Дважды щелкнуть левой кнопкой мыши по записи <<Движение>>, находящейся в статусе <<Начато>>. & Открывается форма \kw{Калинина Динара Павловна - Движение}. \\ \hline
\nn & В ячейке \dm{Значение} табличной части формы в строке <<Время выбытия>> указать текущее время. Остальные пустые поля заполнить произвольным образом. & Внесено время выбытия пациента из отделения. \\ \hline
\nn & В поле \dm{Состояние} в верхней части формы выбрать значение <<Закончено>>. & В поле \dm{Выполнено} автоматически подставлена текущая дата. \\ \hline
\nn & Нажать кнопку \kw{Сохранить}. & Текущая форма закрывается. Данные о закрытии движения сохраняются в БД. Осуществляется возврат на форму \kw{Стационарное лечение (платные услуги)}. \\ \hline
\nn & Нажать кнопку \kw{Выписка}. & Открывается форма \kw{Калинина Динара Павловна - Выписка}. В поле \dm{Назначено} верхней части формы указаны текущие дата и время, поле \dm{Состояние} установлено <<Начато>>, в поле \dm{Исполнитель} указан текущий пользователь.  \\ \hline
\nn & В ячейках \dm{Значение} табличной части формы указать следующие значения:  \newline \dm{Исход госпитализации}: <<выписа>>, \newline \dm{Трудоспособность}: <<восстановлена полностью>>, \newline \dm{Результат госпмтализации}: <<Улучшение>>. Остальные поля оставить незаполненными. & Данные о выписке пациента внесены. \\ \hline
\nn & В поле \dm{Состояние} в верхней части формы выбрать значение <<Закончено>>. & В поле \dm{Выполнено} автоматически подставлена текущая дата. \\ \hline
\nn & Нажать кнопку \kw{Сохранить}. & Текущая форма закрывается. Данные о выписке из отделения сохраняются в БД. Освобождается койка, занимаемая пациентом. Осуществляется возврат на форму \kw{Стационарное лечение (платные услуги)}. \\ \hline
\nn & Нажать кнопку \kw{Сохранить}. & Появляется диалоговое окно, содержащее сообщение <<Данные успешно сохранены>>. \\ \hline
\nn & Нажать кнопку \kw{OK}. & Диалоговое окно закрывается. В табличной части формы появилась новая запись о враче. \\ \hline
\end{longtable}

\subsubsection{Формирование эпикриза (на примере выписного эпикриза)} \label{vip_epik_st}

\textbf{Необходимые начальные условия:} Должны быть выполнены все тесты раздела \ref{lecp_st} Должна быть открыта форма \kw{Стационарное лечение (платные услуги)} для данного обращения.

\textbf{Роли пользователя:} Врач отделения, Администратор.

\setcounter{nnn}{0}
\begin{longtable}{|p{1cm}|p{7.5cm}|p{8cm}|}
\caption{Формирование эпикриза \label{vip_epik_ st_tbl}}\\
\hline \rule{0pt}{15pt}  \centering \textbf{№ пп} & \centering \textbf{Действие} & \hfil \textbf{Ожидаемый результат} \\ \hline
\endfirsthead
\hline \rule{0pt}{15pt} \centering \textbf{№ пп} & \centering \textbf{Действие} & \hfil \textbf{Ожидаемый результат} \\ \hline
\endhead
\nn & В левой части формы выбрать раздел \kw{Медицинские документы}, щелкнув по нему левой кнопкой мыши. & Осуществляется переход в раздел \kw{Медицинские документы} обращения. \\ \hline
\nn & Нажать кнопку \kw{Создать} в правой верхней части формы. & Открывается форма \kw{Создание действий}. \\ \hline
\nn & Перейти на вкладку \kw{Дерево}. & Осуществлен переход на вкладку \kw{Дерево} на форме \kw{Создание действий}. \\ \hline
\nn & Последовательно раскрыть следующие ветви дерева, нажимая знак <<$+$>> слева от названия группы: <<Стационар>> -- <<4500: Эпикризы>>. & Раскрыт список видов эпикризов. \\ \hline 
\nn & Выбрать из списка <<4504 Заключительный эпикриз>> и дважды щелкнуть по нему левой кнопкой мыши либо перетащить запись мышкой в таблицу \kw{Выбранные действия} в правой части формы. & Действие появилось в таблице \kw{Выбранные действия}. \\ \hline
\nn & Нажать кнопку \kw{ОК}. & Текущая форма закрывается. Открывается форма \kw{Калинина Динара Павловна - Заключительный эпикриз}. \\ \hline
\nn & В поле \dm{Выполнено} в верхней части формы указать текущую дату и время <<12:00>>. & Установлена дата и время выдачи заключения. \\ \hline
\nn & В ячейках \dm{Значение} таблицы, расположенной в левой части формы ввести следующие данные для указанных строк: \newline \dm{Клинический диагноз}: <<Вызванная беременность гипертензия без протеинурии>>, \newline \dm{Основной клинический диагноз по МКБ}: <<О13.0>>, \newline \dm{Сопутствующие диагнозы}: <<ИЦН>>, \newline \dm{Сопутствующие диагнозы по МКБ}: <<О34.3>>, \newline \dm{Результат лечения}: <<Положительная динамика>>, \newline \dm{Выписан}: <<С улучшением>>. & \\ \hline
\nn & В ячеке \dm{Значение} строки <<Жалобы>> таблицы щелкнуть правой кнопкой мыши и в контекстном меню выбрать \kw{Заполнить}, в следующем меню \kw{Жалобы из первичного осмотра}. & В ячейку копируются данные из соответствующей ячейки зарегистрированного ранее первичного осмотра. \\ \hline
\nn & В ячеке \dm{Значение} строки <<Анамнез заболевания>> таблицы щелкнуть правой кнопкой мыши и в контекстном меню выбрать \kw{Заполнить}. & В ячейку копируются данные из соответствующей ячейки зарегистрированного ранее первичного осмотра. \\ \hline
\nn & В ячеке \dm{Значение} строки <<Проведенные обследования>> таблицы щелкнуть правой кнопкой мыши и в контекстном меню выбрать \kw{Заполнить}. & Открывается форма \kw{Выбор действий}. \\ \hline
\nn & В верхней части формы снять флажок для строки <<Общий анализ мочи>> и убедиться. & Флажки установлены для исследований <<Общий анализ крови>>, <<Анализ мочи по Нечипоренко>> и <<Протокол УЗИ в I триместре беременности>>.\\ \hline
\nn & Нажать кнопку \kw{OK}. & В ячейку \dm{Значение} строки <<Проведенные обследования>> в табличной форме добавлены результаты выбранных исследований.\\ \hline
\nn & В ячеке \dm{Значение} строки <<Проведенное лечение>> таблицы щелкнуть правой кнопкой мыши и в контекстном меню выбрать \kw{Заполнить}. & Открывается форма \kw{Выбор действий}. \\ \hline
\nn & Убедиться, что установлен флажок для записи <<Операция (акушерство)>> и нажать кнопку \kw{OK}. & В ячейку \dm{Значение} строки <<Проведенное лечение>> в табличной форме добавлен результат операции.\\ \hline
\nn & В ячейку \dm{Значение} строки <<Проведенное лечение>> после таблицы ввести с клавиатуры текст <<Магневист, Глюкоза + вит.С, ЛФК>>. & Заполнены данные о проведенном лечении. \\ \hline
\nn & Нажать кнопку \kw{Сохранить}. & Текущая форма закрывается. Осуществляется возврат на форму\kw{Стационарное лечение (платные услуги)}. В списке медицинских документов появляется новая запись.\\ \hline
\nn & Нажать кнопку \kw{Сохранить}. & Появляется диалоговое окно, содержащее сообщение <<Данные успешно сохранены>>. \\ \hline
\nn & Нажать кнопку \kw{OK}. & Диалоговое окно закрывается. \\ \hline
\end{longtable}

\subsubsection{Регистрация заключительных диагнозов пациента} \label{dzn_st}

\textbf{Необходимые начальные условия:} Должен быть зарегистрирован выписной эпикриз согласно п. \ref{vip_epik_st}. Должна быть открыта форма \kw{Стационарное лечение (платные услуги)} для данного обращения.

\textbf{Роли пользователя:} Врач отделения, Администратор.

\setcounter{nnn}{0}
\begin{longtable}{|p{1cm}|p{7.5cm}|p{8cm}|}
\caption{Регистрация заключительных диагнозов пациента \label{dzn_st_tbl}}\\
\hline \rule{0pt}{15pt}  \centering \textbf{№ пп} & \centering \textbf{Действие} & \hfil \textbf{Ожидаемый результат} \\ \hline
\endfirsthead
\hline \rule{0pt}{15pt} \centering \textbf{№ пп} & \centering \textbf{Действие} & \hfil \textbf{Ожидаемый результат} \\ \hline
\endhead
\nn & В левой части формы выбрать раздел \kw{Основная информация}, щелкнув по нему левой кнопкой мыши. & Осуществляется переход в раздел \kw{Основная информация} обращения. \\ \hline
\nn & Нажать кнопку \kw{Окончательные диагнозы}. & Открывается форма \kw{Таблица диаг\-нозов}. \\ \hline
\nn & В первой строке таблицы дважды щелкнуть левой кнопкой мыши в ячейке \dm{Тип} и выбрать из списка <<заключительный>>, в ячейке \dm{МКБ} ввести <<О13.0>>, в ячейке \dm{Хар} дважды щелкнуть левой кнопкой мыши и выбрать значение <<острое>>, в ячейке \dm{Ст} дважды щелкнуть левой кнопкой мыши и выбрать <<основной>>. & Данные внесены в соответствующие поля. В поле \dm{Врач} автоматически подставлена фамилия текущего лечащего врача. \\ \hline
\nn & В следующей строке таблицы дважды щелкнуть левой кнопкой мыши в ячейке \dm{Тип} и выбрать из списка <<сопутствующий>>, в ячейке \dm{МКБ} ввести <<О34.3>>, в ячейке \dm{Хар} дважды щелкнуть левой кнопкой мыши и выбрать значение <<острое>>, в ячейке \dm{Ст} дважды щелкнуть левой кнопкой мыши и выбрать <<сопутствующий>>. & Данные внесены в соответствующие поля. В поле \dm{Врач} автоматически подставлена фамилия текущего лечащего врача. \\ \hline
\nn & Нажать кнопку \kw{Сохранить}. & Текущая форма закрывается. Осуществляется возврат на форму \kw{Стационарное лечение (платные услуги)}. \\ \hline
\nn & Нажать кнопку \kw{Сохранить}. & Появляется диалоговое окно, содержащее сообщение <<Данные успешно сохранены>>. \\ \hline
\nn & Нажать кнопку \kw{OK}. & Диалоговое окно закрывается. \\ \hline
\end{longtable}

\subsubsection{Закрытие обращения} \label{close_st}

\textbf{Необходимые начальные условия:} Должна быть зарегистрирована выписка пациента согласно п. \ref{vip_dv_st} Должен быть зарегистрирован выписной эпикриз согласно п. \ref{vip_epik_st}. Должны быть зарегистрированы заключительные диагнозы пациента согласно п. \ref{close_st} Должна быть открыта форма \kw{Стационарное лечение (платные услуги)} для данного обращения.

\textbf{Роли пользователя:} Врач отделения, Администратор.

\setcounter{nnn}{0}
\begin{longtable}{|p{1cm}|p{7.5cm}|p{8cm}|}
\caption{Закрытие обращения \label{close_st_tbl}}\\
\hline \rule{0pt}{15pt}  \centering \textbf{№ пп} & \centering \textbf{Действие} & \hfil \textbf{Ожидаемый результат} \\ \hline
\endfirsthead
\hline \rule{0pt}{15pt} \centering \textbf{№ пп} & \centering \textbf{Действие} & \hfil \textbf{Ожидаемый результат} \\ \hline
\endhead
\nn & В левой части формы выбрать раздел \kw{Основная информация}, щелкнув по нему левой кнопкой мыши. & Осуществляется переход в раздел \kw{Основная информация} обращения. \\ \hline
\nn & В поле \dm{Дата выписки} указать текущую дату и время, в поле \dm{Результат госпитализации} выбрать из списка <<Выписан>>, в поле \dm{Исход госпитализации} выбрать <<Улучшение>>. & Указана дата завершения случая госпитализации, результат и исход лечения. \\ \hline
\nn & Нажать кнопку \kw{Закрыть обращение} в левом нижнем углу формы. & Появляется диалоговое окно <<Обращение завершено>>. \\ \hline 
\nn & Нажать кнопку \kw{OK}. & Диалоговое окно и форма \kw{Стационарное лечение (платные услуги)} закрываются. \\ \hline
\end{longtable}

