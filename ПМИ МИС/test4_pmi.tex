\subsection{Ведение ЭПМЗ в поликлинике} \label{epmz_pol}
\subsubsection{Регистрация осмотра$\slash$консультации пациента в поликлинике} \label{osmotr_pol}

\textbf{Необходимые начальные условия:} Должно быть зарегистрировано поликлиническое обращение пациента согласно п. \ref{obsl1_pol}

\textbf{Роли пользователя:} Врач поликлиники, Администратор.

\setcounter{nnn}{0}
\begin{longtable}{|p{1cm}|p{7.5cm}|p{8cm}|}
\caption{Регистрация осмотра пациента в поликлинике \label{osmotr_ pol_tbl}}\\
\hline \rule{0pt}{15pt}  \centering \textbf{№ пп} & \centering \textbf{Действие} & \hfil \textbf{Ожидаемый результат} \\ \hline
\endfirsthead
\hline \rule{0pt}{15pt} \centering \textbf{№ пп} & \centering \textbf{Действие} & \hfil \textbf{Ожидаемый результат} \\ \hline
\endhead
\nn & В главном меню выбрать пункт \mm{Работа \str Обслуживание пациентов}. & Открывается форма, содержащая картотеку пациентов. \\ \hline
\nn & Перейти на вкладку \kw{Обращение}. & Осуществлен переход на вкладку \kw{Обращение}. \\ \hline
\nn & На панели \kw{Фильтр} в правой части экрана установить флажок \dm{Тип обращения} и в ставшем активным поле выбрать из списка <<Поликлиника>>, установить флажок \dm{Специальность} и в ставшем активным поле выбрать из списка <<Лечебное дело>>, установить флажок \dm{Врач} и в ставшем активным поле выбрать <<Дорофеев Р.О.>>. В поле \dm{Показывать осмотры} выбрать из списка значение <<все>>. & Заданы параметры поиска обращения. \\ \hline
\nn & Нажать кнопку \kw{Применить} на панели \kw{Фильтр}. & Отображается список обращений, найденных по заданным параметрам поиска. \\ \hline
\nn & Дважды щелкнуть левой кнопкой мыши по незавершенному обращению пациента <<Иванов Иван Васильевич, 12.12.1961>>. & Открывается форма \kw{Поликлиника(платный)}, содержащая данные обращения. \\ \hline
\nn & В левой части формы выбрать раздел \kw{Медицинские документы}, щелкнув по нему левой кнопкой мыши. & Осуществляется переход в раздел \kw{Медицинские документы} обращения. Список документов пуст. \\ \hline
\nn & Нажать кнопку \kw{Создать} в правой верхней части формы. & Открывается форма \kw{Создание действий}. \\ \hline
\nn & Перейти на вкладку \kw{Список}. & Осуществлен переход на вкладку \kw{Список} на форме \kw{Создание действий}. \\ \hline 
\nn & Выбрать из списка <<1\_1\_1\_09 Офтальмолог первично>> и дважды щелкнуть по нему левой кнопкой мыши либо перетащить запись мышкой в таблицу \kw{Выбранные действия} в правой части формы. & Действие появилось в таблице \kw{Выбранные действия}. \\ \hline
\nn & Нажать кнопку \kw{ОК}. & Текущая форма закрывается. Открывается форма \kw{Иванов Иван Васильевич - Офтальмолог первично}. \\ \hline
\nn \label{n1} & Установить курсор в ячейку \dm{Значение} строки <<Жалобы>> табличной части формы. & Строка <<Жалобы>> расширяется, появляются инструменты форматирования текста по правому краю ячейки. В правой части формы в подразделе \kw{Тезаурус} появляется список наиболее часто употребляемых для описания жалоб фраз. \\ \hline
\nn & В подразделе \kw{Тезаурус} дважды щелкнуть левой кнопкой мыши по строке <<на утомление глаз при зрительной нагрузке>>, затем по строке <<на чувство засоренности глаз>>. & В ячейке \dm{Значение} строки <<Жалобы>> появилась фраза <<на утомление глаз при зрительной нагрузке, на чувство засоренности глаз>>. \\ \hline
\nn & не перемещая курсора ввести с клавиатуры << после работы за компьютером>>. & В ячейке \dm{Значение} строки <<Жалобы>> получилась фраза <<на утомление глаз при зрительной нагрузке, на чувство засоренности глаз после работы за компьютером>>. \\ \hline 
\nn \label{n2}& Перейти в ячейку \dm{Значение} строки <<Анамнез заболевания>> нажатием клавиши Tab на клавиатуре либо щелкнув по ячейке левой кнопкой мыши. & Осуществлен переход к следующей строке таблицы. \\  \hline
\nn & Выполнить действия аналогичные описанным в пп. \ref{n1} -- \ref{n2} для каждой строки таблицы. & Заполнена таблица в левой части формы. \\ \hline
\nn & Нажать кнопку \kw{OK}. & Текущая форма закрывается. Осуществляется возврат на форму\kw{Поликлиника(платный)}. В списке медицинских документов появилась 1 запись.\\ \hline
\nn & Нажать кнопку \kw{Сохранить}. & Появляется диалоговое окно, содержащее сообщение <<Данные успешно сохранены>>. \\ \hline
\nn & Нажать кнопку \kw{OK}. & Диалоговое окно закрывается. \\ \hline
\end{longtable}

\subsubsection{Регистрация манипуляций} \label{oper_pol}

\textbf{Необходимые начальные условия:} Должен быть зарегистрирован осмотр пациента в поликлинике согласно п. \ref{osmotr_pol} Должна быть открыта форма \kw{Поликлиника(платный)} для данного обращения.

\textbf{Роли пользователя:} Врач поликлиники, Администратор.

\setcounter{nnn}{0}
\begin{longtable}{|p{1cm}|p{7.5cm}|p{8cm}|}
\caption{Регистрация манипуляций в поликлинике \label{oper_ pol_tbl}}\\
\hline \rule{0pt}{15pt}  \centering \textbf{№ пп} & \centering \textbf{Действие} & \hfil \textbf{Ожидаемый результат} \\ \hline
\endfirsthead
\hline \rule{0pt}{15pt} \centering \textbf{№ пп} & \centering \textbf{Действие} & \hfil \textbf{Ожидаемый результат} \\ \hline
\endhead
\nn & В левой части формы выбрать раздел \kw{Лечение}, щелкнув по нему левой кнопкой мыши. & Осуществляется переход в раздел \kw{Лечение} обращения. Список медицинских записей пуст. \\ \hline
\nn & Нажать кнопку \kw{Создать} в правой верхней части формы. & Открывается форма \kw{Создание действий}. \\ \hline
\nn & Перейти на вкладку \kw{Список}. & Осуществлен переход на вкладку \kw{Список} на форме \kw{Создание действий}. \\ \hline
\nn & В поле \dm{Поиск} ввести <<глаз>> c клавиатуры и нажать клавишу Enter. & В списке в левой части формы отображаются только записи, содержащие в своем названии подстроку <<глаз>>. \\ \hline 
\nn & Выбрать из списка <<А02.26.001 Исследование переднего сегмента глаза методом бокового освещения>> и дважды щелкнуть по нему левой кнопкой мыши либо перетащить запись мышкой в таблицу \kw{Выбранные действия} в правой части формы. & Запись появилась в таблице \kw{Выбранные действия}. \\ \hline
\nn & Выбрать из списка <<А03.26.003 Осмотр переферии глазного дна трехзеркальной линзой Гольдмана>> и дважды щелкнуть по нему левой кнопкой мыши либо перетащить запись мышкой в таблицу \kw{Выбранные действия} в правой части формы. & Запись появилась в таблице \kw{Выбранные действия}. \\ \hline
\nn & Нажать кнопку \kw{ОК}. & Текущая форма закрывается. Открывается форма \kw{Иванов Иван Васильевич - Исследование переднего сегмента глаза методом бокового освещения}. \\ \hline
\nn & Нажать кнопку \kw{Сохранить}. & Текущая форма закрывается. Данные о манипуляции сохраняются в БД. Открывается форма \kw{Иванов Иван Васильевич - Осмотр переферии глазного дна трехзеркальной линзой Гольдмана}.\\ \hline
\nn & Нажать кнопку \kw{Сохранить}. & Текущая форма закрывается. Данные о манипуляции сохраняются в БД. Осуществляется возврат на форму \kw{Поликлиника(платный)}. В списке отображаются 2 записи.\\ \hline
\nn & Нажать кнопку \kw{Сохранить}. & Появляется диалоговое окно, содержащее сообщение <<Данные успешно сохранены>>. \\ \hline
\nn & Нажать кнопку \kw{OK}. & Диалоговое окно закрывается. \\ \hline
\end{longtable}

\subsubsection{Регистрация направлений на исследования} \label{lab_pol}

\textbf{Необходимые начальные условия:} Должен быть зарегистрирован осмотр пациента в поликлинике согласно п. \ref{osmotr_pol} Должна быть открыта форма \kw{Поликлиника(платный)} для данного обращения.

\textbf{Роли пользователя:} Врач поликлиники, Администратор.

\setcounter{nnn}{0}
\begin{longtable}{|p{1cm}|p{7.5cm}|p{8cm}|}
\caption{Регистрация направлений на исследования \label{lab_ pol_tbl}}\\
\hline \rule{0pt}{15pt}  \centering \textbf{№ пп} & \centering \textbf{Действие} & \hfil \textbf{Ожидаемый результат} \\ \hline
\endfirsthead
\hline \rule{0pt}{15pt} \centering \textbf{№ пп} & \centering \textbf{Действие} & \hfil \textbf{Ожидаемый результат} \\ \hline
\endhead
\nn & В левой части формы выбрать раздел \kw{Диагностические и лабораторные исследования}, щелкнув по нему левой кнопкой мыши. & Осуществляется переход в раздел \kw{Диагностические и лабораторные исследования} обращения. Список медицинских записей пуст. \\ \hline
\nn & Нажать кнопку \kw{Создать} в правой верхней части формы. & Открывается форма \kw{Создание действий}. \\ \hline
\nn & Перейти на вкладку \kw{Дерево}. & Осуществляется переход на вкладку \kw{Дерево} на форме \kw{Создание действий}. \\ \hline
\nn & В левой части формы, нажимая на знак <<$+$>> слева от  названия группы, последовательно раскрыть следующие ветви: <<2\_1: Лабораторная -- 2\_1\_01: Общеклинические анализы>>. & Раскрыто дерево видов общеклинических исследований. \\ \hline 
\nn & Выбрать из списка <<2\_1\_01\_01: Общеклинический анализ крови>> и дважды щелкнуть по нему левой кнопкой мыши либо перетащить запись мышкой в таблицу \kw{Выбранные действия} в правой части формы. & Запись появилась в таблице \kw{Выбранные действия}. \\ \hline
\nn & Раскрыть группу <<2\_2: Функциональная>>, нажав левой кнопкой мыши на знак <<$+$>> слева от названия. & Раскрыто дерево видов функциональных исследований. \\ \hline 
\nn & Выбрать из списка <<2\_4\_1\_07: Электрокардиография (ЭКГ)>> и дважды щелкнуть по нему левой кнопкой мыши либо перетащить запись мышкой в таблицу \kw{Выбранные действия} в правой части формы. & Запись появилась в таблице \kw{Выбранные действия}. \\ \hline
\nn & Нажать кнопку \kw{ОК}. & Текущая форма закрывается. Осуществляется возврат на форму \kw{Поликлиника (платный)}. В списке медицинских записей появилось 2 записи. \\ \hline
\nn & Нажать кнопку \kw{Сохранить}. & Появляется диалоговое окно, содержащее сообщение <<Данные успешно сохранены>>. \\ \hline
\nn & Нажать кнопку \kw{OK}. & Диалоговое окно закрывается. \\ \hline
\end{longtable}

\subsubsection{Изменение набора параметров лабораторного  исследования} \label{labedt_pol}

\textbf{Необходимые начальные условия:} Должны быть зарегистрированы направления пациента согласно п. \ref{lab_pol} Должна быть открыта форма \kw{Поликлиника(платный)} в разделе \kw{Диагностические и инструментальные исследования} для данного обращения.

\textbf{Роли пользователя:} Врач поликлиники, Администратор.

\setcounter{nnn}{0}
\begin{longtable}{|p{1cm}|p{7.5cm}|p{8cm}|}
\caption{Изменение набора параметров лабораторного исследования \label{labedt_ pol_tbl}}\\
\hline \rule{0pt}{15pt}  \centering \textbf{№ пп} & \centering \textbf{Действие} & \hfil \textbf{Ожидаемый результат} \\ \hline
\endfirsthead
\hline \rule{0pt}{15pt} \centering \textbf{№ пп} & \centering \textbf{Действие} & \hfil \textbf{Ожидаемый результат} \\ \hline
\endhead
\nn & Дважды щелкнуть левой кнопкой мыши по записи <<Общий анализ крови>>. & Открывается форма <<Иванов Иван Васильевич - Общий аналз крови>>. \\ \hline
\nn & В левой части формы в столбце \dm{Назначено} снять флажки во всех ячейках, расположенных ниже строки <<СОЭ>>. & Флажки \dm{Назначено} установлены только в строках <<Гемоглобин>>--<<СОЭ>>.\\ \hline
\nn & Нажать кнопку \kw{Сохранить}. & Текущая форма закрывается. Изменения набора параметров исследований сохраняются в БД. \\ \hline
\nn & Нажать кнопку \kw{Сохранить}. & Появляется диалоговое окно, содержащее сообщение <<Данные успешно сохранены>>. \\ \hline
\nn & Нажать кнопку \kw{OK}. & Диалоговое окно закрывается. \\ \hline
\end{longtable}

\subsubsection{Назначение медикаментозной терапии} \label{med_pol}

\textbf{Необходимые начальные условия:} Должен быть зарегистрирован осмотр пациента в поликлинике согласно п. \ref{osmotr_pol} Должна быть открыта форма \kw{Поликлиника(платный)} для данного обращения.

\textbf{Роли пользователя:} Врач поликлиники, Администратор.

\setcounter{nnn}{0}
\begin{longtable}{|p{1cm}|p{7.5cm}|p{8cm}|}
\caption{Назначение медикаментозной терапии \label{med_ pol_tbl}}\\
\hline \rule{0pt}{15pt}  \centering \textbf{№ пп} & \centering \textbf{Действие} & \hfil \textbf{Ожидаемый результат} \\ \hline
\endfirsthead
\hline \rule{0pt}{15pt} \centering \textbf{№ пп} & \centering \textbf{Действие} & \hfil \textbf{Ожидаемый результат} \\ \hline
\endhead
\nn & В левой части формы выбрать раздел \kw{Лечение}, щелкнув по нему левой кнопкой мыши. & Осуществляется переход в раздел \kw{Лечение} обращения. \\ \hline
\nn & Нажать кнопку \kw{Создать} в правой верхней части формы. & Открывается форма \kw{Создание действий}. \\ \hline
\nn & Перейти на вкладку \kw{Дерево}. & Осуществлен переход на вкладку \kw{Дерево} на форме \kw{Создание действий}. \\ \hline
\nn & Раскрыть ветвь дерева <<3\_1: Медикаментозное>>, щелкнув левой кнопкой мыши по знаку <<$+$>> слева от названия. & Раскрывается список видов медикаментозных назначений. \\ \hline 
\nn & Выбрать из списка <<3\_1\_05: Назначение>> и дважды щелкнуть по нему левой кнопкой мыши либо перетащить запись мышкой в таблицу \kw{Выбранные действия} в правой части формы. & Запись появилась в таблице \kw{Выбранные действия}. \\ \hline
\nn & Нажать кнопку \kw{ОК}. & Текущая форма закрывается. Открывается форма \kw{Иванов Иван Васильевич - Назначение}. \\ \hline
\nn & В левой части формы внести следующие значения в столбец \dm{Значение}: \newline \dm{Наименование}: <<Тауфон>>, \newline \dm{Доза}: <<2к>>, \newline \dm{Количество раз в день}: <<2>>, \newline \dm{Способ введения}: <<в каждый глаз>>. & Данные внесены в соответствующие поля. \\ \hline
\nn & Нажать кнопку \kw{Сохранить}. & Текущая форма закрывается. Данные о назначении сохраняются в БД. Осуществляется возврат на форму \kw{Поликлиника(платный)}. В списке добавлена новая запись <<Назначение>>.\\ \hline
\nn & Нажать кнопку \kw{Сохранить}. & Появляется диалоговое окно, содержащее сообщение <<Данные успешно сохранены>>. \\ \hline
\nn & Нажать кнопку \kw{OK}. & Диалоговое окно закрывается. \\ \hline
\end{longtable}

\subsubsection{Назначение лечения} \label{fiz_pol}

\textbf{Необходимые начальные условия:} Должен быть зарегистрирован осмотр пациента в поликлинике согласно п. \ref{osmotr_pol} Должна быть открыта форма \kw{Поликлиника(платный)} для данного обращения.

\textbf{Роли пользователя:} Врач поликлиники, Администратор.

\setcounter{nnn}{0}
\begin{longtable}{|p{1cm}|p{7.5cm}|p{8cm}|}
\caption{Назначение лечения \label{fiz_ pol_tbl}}\\
\hline \rule{0pt}{15pt}  \centering \textbf{№ пп} & \centering \textbf{Действие} & \hfil \textbf{Ожидаемый результат} \\ \hline
\endfirsthead
\hline \rule{0pt}{15pt} \centering \textbf{№ пп} & \centering \textbf{Действие} & \hfil \textbf{Ожидаемый результат} \\ \hline
\endhead
\nn & В левой части формы выбрать раздел \kw{Лечение}, щелкнув по нему левой кнопкой мыши. & Осуществляется переход в раздел \kw{Лечение} обращения. \\ \hline
\nn & Нажать кнопку \kw{Создать} в правой верхней части формы. & Открывается форма \kw{Создание действий}. \\ \hline
\nn & Перейти на вкладку \kw{Дерево}. & Осуществлен переход на вкладку \kw{Дерево} на форме \kw{Создание действий}. \\ \hline
\nn & Раскрыть ветвь дерева <<3\_10: Лечебно-терапевтическое отделение>>, щелкнув левой кнопкой мыши по знаку <<$+$>> слева от названия. & Раскрывается список видов физиотерапевтических процедур. \\ \hline 
\nn & Выбрать из списка <<3\_10\_3: Электрофорез>> и дважды щелкнуть по нему левой кнопкой мыши либо перетащить запись мышкой в таблицу \kw{Выбранные действия} в правой части формы. & Запись появилась в таблице \kw{Выбранные действия}. \\ \hline
\nn & Нажать кнопку \kw{ОК}. & Текущая форма закрывается. Открывается форма \kw{Иванов Иван Васильевич - Электрофорез}. \\ \hline
\nn & В поле \dm{Количество} в верхней части формы ввести значение <<10>>. & Указано количество процедур. \\ \hline
\nn & В таблице в левой части формы в ячейку \dm{Значение} строки <<Лекарственные средства>> ввести <<Эуфиллин>>. & Указано лекарственое средство для процедур. \\ \hline
\nn & Нажать кнопку \kw{Сохранить}. & Текущая форма закрывается. Данные о назначении сохраняются в БД. Осуществляется возврат на форму \kw{Поликлиника(платный)}. В списке появилась новая запись <<Электрофорез>>.\\ \hline
\nn & Нажать кнопку \kw{Сохранить}. & Появляется диалоговое окно, содержащее сообщение <<Данные успешно сохранены>>. \\ \hline
\nn & Нажать кнопку \kw{OK}. & Диалоговое окно закрывается. \\ \hline
\end{longtable}

\subsubsection{Открытие листа временной нетрудоспособности} \label{vutopen_pol}

\textbf{Необходимые начальные условия:} Должен быть зарегистрирован осмотр пациента в поликлинике согласно п. \ref{osmotr_pol} Должна быть открыта форма \kw{Поликлиника(платный)} для данного обращения.

\textbf{Роли пользователя:} Врач поликлиники, Администратор.

\setcounter{nnn}{0}
\begin{longtable}{|p{1cm}|p{7.5cm}|p{8cm}|}
\caption{Открытие листа временной нетрудоспособности \label{vutopen_ pol_tbl}}\\
\hline \rule{0pt}{15pt}  \centering \textbf{№ пп} & \centering \textbf{Действие} & \hfil \textbf{Ожидаемый результат} \\ \hline
\endfirsthead
\hline \rule{0pt}{15pt} \centering \textbf{№ пп} & \centering \textbf{Действие} & \hfil \textbf{Ожидаемый результат} \\ \hline
\endhead
\nn & В левой части формы выбрать раздел \kw{ВУТ}, щелкнув по нему левой кнопкой мыши. & Осуществляется переход в раздел \kw{ВУТ} обращения. \\ \hline
\nn & На вкладке \kw{Листок нетрудоспособности} установить флажок \dm{ВУТ}. & Становятся доступными поля для регистрации листа временной нетрудоспособности. \\ \hline
\nn & В поле \dm{Причина} выбрать из списка <<заболевание>>, в поля \dm{Серия и номер} ввести значения <<B4>> и <<451230>> соответственно, в поле \dm{Место работы} ввести <<МУ Школа №1 г. Степногорска>>. & Данные внесены в соответствующие поля. \\ \hline
\nn & В первой строке табличной части формы дважды щелкнуть левой кнопкой мыши в ячейке \dm{Начало}, в ячейке \dm{Окончание} выбрать дату на 5 дней больше текущей, в ячейке \dm{Результат} двойным щелчком мыши активировать список и выбрать значение <<нетрудоспособен>>, в ячейке \dm{Режим} двойным щелчком левой кнопки мыши активировать список и выбрать значение <<домашний>>. & Данные внесены в соответствующие поля. В ячейке \dm{Начало} установлена текущая дата, в ячейке \dm{Длительность} и поле  \dm{Длительность} в верхней части формы автоматически рассчитано значение <<5>>.\\ \hline
\nn & Нажать кнопку \kw{Сохранить}. & Появляется диалоговое окно, содержащее сообщение <<Данные успешно сохранены>>. \\ \hline
\nn & Нажать кнопку \kw{OK}. & Диалоговое окно закрывается. \\ \hline
\end{longtable}

\subsubsection{Регистрация повторного осмотра} \label{osmotr2_pol}

\textbf{Необходимые начальные условия:} Должен быть зарегистрирован осмотр пациента в поликлинике согласно п. \ref{osmotr_pol} Должна быть открыта форма \kw{Поликлиника(платный)} для данного обращения.

\textbf{Роли пользователя:} Врач поликлиники, Администратор.

\setcounter{nnn}{0}
\begin{longtable}{|p{1cm}|p{7.5cm}|p{8cm}|}
\caption{Регистрация повторного осмотра \label{osmotr2_ pol_tbl}}\\
\hline \rule{0pt}{15pt}  \centering \textbf{№ пп} & \centering \textbf{Действие} & \hfil \textbf{Ожидаемый результат} \\ \hline
\endfirsthead
\hline \rule{0pt}{15pt} \centering \textbf{№ пп} & \centering \textbf{Действие} & \hfil \textbf{Ожидаемый результат} \\ \hline
\endhead
\nn & В левой части формы выбрать раздел \kw{Медицинские документы}, щелкнув по нему левой кнопкой мыши. & Осуществляется переход в раздел \kw{Медицинские документы} обращения. \\ \hline
\nn & Нажать кнопку \kw{Создать} в правой верхней части формы. & Открывается форма \kw{Создание действий}. \\ \hline
\nn & Перейти на вкладку \kw{Дерево}. & Осуществлен переход на вкладку \kw{Дерево} на форме \kw{Создание действий}. \\ \hline
\nn & Последовательно раскрыть следующие ветви дерева, нажимая знак <<$+$>> слева от названия группы: <<1\_1: Поликлинический прием>> -- <<1\_1\_2: Повторный прием>>. & Раскрыт список видов документов при повторном приеме. \\ \hline 
\nn & Выбрать из списка <<1\_1\_2\_09 Офтальмолог повторно>> и дважды щелкнуть по нему левой кнопкой мыши либо перетащить запись мышкой в таблицу \kw{Выбранные действия} в правой части формы. & Действие появилось в таблице \kw{Выбранные действия}. \\ \hline
\nn & Нажать кнопку \kw{ОК}. & Текущая форма закрывается. Открывается форма \kw{Иванов Иван Васильевич - Офтальмолог повторно}. \\ \hline
\nn & Повторить пп. \ref{n1} -- \ref{n2} таблицы \ref{osmotr_ pol_tbl} для каждой строки табличной части текущей формы. & Заполнена таблица в левой части формы. \\ \hline 
\nn & Нажать кнопку \kw{OK}. & Текущая форма закрывается. Осуществляется возврат на форму\kw{Поликлиника(платный)}. В списке медицинских документов появляется новая запись.\\ \hline
\nn & Нажать кнопку \kw{Сохранить}. & Появляется диалоговое окно, содержащее сообщение <<Данные успешно сохранены>>. \\ \hline
\nn & Нажать кнопку \kw{OK}. & Диалоговое окно закрывается. \\ \hline
\end{longtable}

\subsubsection{Регистрация продолжения листа временной нетрудоспособности} \label{vutl_pol}

\textbf{Необходимые начальные условия:} Должен быть зарегистрирован лист временной нетрудоспособности согласно п. \ref{vutopen_pol} Должна быть открыта форма \kw{Поликлиника(платный)} для данного обращения.

\textbf{Роли пользователя:} Врач поликлиники, Администратор.

\setcounter{nnn}{0}
\begin{longtable}{|p{1cm}|p{7.5cm}|p{8cm}|}
\caption{Регистрация продолжения листа временной нетрудоспособности \label{vutl_ pol_tbl}}\\
\hline \rule{0pt}{15pt}  \centering \textbf{№ пп} & \centering \textbf{Действие} & \hfil \textbf{Ожидаемый результат} \\ \hline
\endfirsthead
\hline \rule{0pt}{15pt} \centering \textbf{№ пп} & \centering \textbf{Действие} & \hfil \textbf{Ожидаемый результат} \\ \hline
\endhead
\nn & В левой части формы выбрать раздел \kw{ВУТ}, щелкнув по нему левой кнопкой мыши. & Осуществляется переход в раздел \kw{ВУТ} обращения. \\ \hline
\nn & На вкладке \kw{Листок нетрудоспособности} в пустой строке табличной части формы в ячейке \dm{Начало} установить дату следующую за датой, указанной в поле \dm{Окончание} предыдущей строки, в ячейке \dm{Окончание} выбрать дату на 3 дня больше начальной, в ячейке \dm{Результат} двойным щелчком мыши активировать список и выбрать значение <<трудоспособен>>, в ячейке \dm{Режим} двойным щелчком левой кнопки мыши активировать список и выбрать значение <<домашний>>. & Данные внесены в соответствующие поля. В ячейке в ячейке \dm{Длительность} текущей строки автоматически установлено значение <<3>>, в поле  \dm{Длительность} в верхней части формы автоматически рассчитано значение <<8>>.\\ \hline
\nn & Нажать кнопку \kw{Сохранить}. & Появляется диалоговое окно, содержащее сообщение <<Данные успешно сохранены>>. \\ \hline
\nn & Нажать кнопку \kw{OK}. & Диалоговое окно закрывается. \\ \hline
\end{longtable}

\subsubsection{Регистрация диагнозов пациента} \label{dz_pol}

\textbf{Необходимые начальные условия:} Должен быть зарегистрирован осмотр пациента в поликлинике согласно п. \ref{osmotr_pol} Должна быть открыта форма \kw{Поликлиника(платный)} для данного обращения.

\textbf{Роли пользователя:} Врач поликлиники, Администратор.

\setcounter{nnn}{0}
\begin{longtable}{|p{1cm}|p{7.5cm}|p{8cm}|}
\caption{Регистрация диагнозов пациента \label{dz_ pol_tbl}}\\
\hline \rule{0pt}{15pt}  \centering \textbf{№ пп} & \centering \textbf{Действие} & \hfil \textbf{Ожидаемый результат} \\ \hline
\endfirsthead
\hline \rule{0pt}{15pt} \centering \textbf{№ пп} & \centering \textbf{Действие} & \hfil \textbf{Ожидаемый результат} \\ \hline
\endhead
\nn & В левой части формы выбрать раздел \kw{Основная информация}, щелкнув по нему левой кнопкой мыши. & Осуществляется переход в раздел \kw{Основная информация} обращения. \\ \hline
\nn & Нажать кнопку \kw{Окончательные диагнозы}. & Открывается форма \kw{Таблица диаг\-нозов}. \\ \hline
\nn & В первой строке таблицы дважды щелкнуть левой кнопкой мыши в ячейке \dm{Тип} и выбрать из списка <<заключительный>>, в ячейке \dm{МКБ} ввести <<H44.2>>, в ячейке \dm{Хар} дважды щелкнуть левой кнопкой мыши и выбрать значение <<обострение хронического>>. & Данные внесены в соответствующие поля. В поле \dm{Врач} автоматически подставлено значение <<Дорофеев Р.О.>>. \\ \hline
\nn & Нажать кнопку \kw{Сохранить}. & Текущая форма закрывается. Осуществляется возврат на форму \kw{Поликлиника (платный)}. \\ \hline
\nn & Нажать кнопку \kw{Сохранить}. & Появляется диалоговое окно, содержащее сообщение <<Данные успешно сохранены>>. \\ \hline
\nn & Нажать кнопку \kw{OK}. & Диалоговое окно закрывается. \\ \hline
\end{longtable}

\subsubsection{Печать медицинских документов пациента} \label{prn_pol}

\textbf{Необходимые начальные условия:} Должны быть выполнены пп. \ref{osmotr_pol} -- \ref{dz_pol}  Должна быть открыта форма \kw{Поликлиника(платный)} для данного обращения.

\textbf{Роли пользователя:} Врач поликлиники, Администратор.

\setcounter{nnn}{0}
\begin{longtable}{|p{1cm}|p{7.5cm}|p{8cm}|}
\caption{Печать медицинских документов \label{prn_ pol_tbl}}\\
\hline \rule{0pt}{15pt}  \centering \textbf{№ пп} & \centering \textbf{Действие} & \hfil \textbf{Ожидаемый результат} \\ \hline
\endfirsthead
\hline \rule{0pt}{15pt} \centering \textbf{№ пп} & \centering \textbf{Действие} & \hfil \textbf{Ожидаемый результат} \\ \hline
\endhead
\nn & В левой части формы выбрать раздел \kw{Основная информация}, щелкнув по нему левой кнопкой мыши. & Осуществляется переход в раздел \kw{Основная информация} обращения. \\ \hline
\nn & Нажать кнопку \kw{Печать} и из появившегося меню выбрать название печатной формы. & Открывается окно предварительного просмотра печатной формы. \\ \hline
\nn & Нажать кнопку \kw{Печатать} и направить документ на нужный принтер. & Выполнена печать медицинского документа. \\ \hline
\nn & Нажать на кнопку \kw{Закрыть}. & Окно пердварительно просмотра печатной формы закрывается. \\ \hline
\nn & Повторить пп. 2--4 для других печатных форм. & Выполнена печать медицинских документов по данному обращению. \\ \hline
\nn & В левой части формы выбрать раздел \kw{Медицинские документы}, щелкнув по нему левой кнопкой мыши. & Осуществляется переход в раздел \kw{Медицинские документы} обращения. \\ \hline
\nn & Двыжды щелкнуть левой кнопкой мыши по записи <<Офтальмолог первично>>. & Открывается форма \kw{Иванов Иван Васильевич - Офтальмолог первично}. \\ \hline
\nn & Нажать кнопку \kw{Печать} и из появившегося меню выбрать <<Осмотр в поликлинике>>. & Открывается окно предварительного просмотра печатной формы. \\ \hline
\nn & Нажать кнопку \kw{Печатать} и направить документ на нужный принтер. & Выполнена печать протокола первичного осмотра. \\ \hline
\nn & Нажать на кнопку \kw{Закрыть}. & Окно пердварительно просмотра печатной формы закрывается. \\ \hline
\nn & Нажать на кнопку \kw{Закрыть}. & Текущая форма закрывается. \\ \hline
\nn & В левой части формы выбрать раздел \kw{Диагностические и лабораторные исследования}, щелкнув по нему левой кнопкой мыши. & Осуществляется переход в раздел \kw{Диагностические и лабораторные исследования} обращения. \\ \hline
\nn & Двыжды щелкнуть левой кнопкой мыши по записи <<Общий анализ крови>>. & Открывается форма \kw{Иванов Иван Васильевич - Общий анализ крови}. \\ \hline
\nn & Нажать кнопку \kw{Печать} и из появившегося меню выбрать <<Результаты всех анализов пациента>>. & Открывается окно предварительного просмотра печатной формы. \\ \hline
\nn & Нажать на кнопку \kw{Закрыть}. & Окно пердварительно просмотра печатной формы закрывается. \\ \hline
\nn & Нажать на кнопку \kw{Закрыть}. & Текущая форма закрывается. \\ \hline
\end{longtable}

\subsubsection{Закрытие обращения} \label{close_pol}

\textbf{Необходимые начальные условия:} Должен быть зарегистрирован осмотр пациента в поликлинике согласно п. \ref{osmotr_pol} Должна быть открыта форма \kw{Поликлиника(платный)} для данного обращения.

\textbf{Роли пользователя:} Врач поликлиники, Администратор.

\setcounter{nnn}{0}
\begin{longtable}{|p{1cm}|p{7.5cm}|p{8cm}|}
\caption{Закрытие обращения \label{close_pol_tbl}}\\
\hline \rule{0pt}{15pt}  \centering \textbf{№ пп} & \centering \textbf{Действие} & \hfil \textbf{Ожидаемый результат} \\ \hline
\endfirsthead
\hline \rule{0pt}{15pt} \centering \textbf{№ пп} & \centering \textbf{Действие} & \hfil \textbf{Ожидаемый результат} \\ \hline
\endhead
\nn & В левой части формы выбрать раздел \kw{Основная информация}, щелкнув по нему левой кнопкой мыши. & Осуществляется переход в раздел \kw{Основная информация} обращения. \\ \hline
\nn & В поле \dm{Дата выполнения} указать текущую дату и время, в поле \dm{Результат обращения} выбрать из списка <<Лечение завершено>>, в поле \dm{Исход лечения} выбрать <<Улучшение>>. & Указана дата завершения случая обслуживания в поликлинике, результат и исход лечения. \\ \hline
\nn & Нажать кнопку \kw{Закрыть обращение} в левом нижнем углу формы. & Появляется диалоговое окно с предупреждением о наличии незакрытых действий. \\ \hline
\nn & Нажать кнопку \kw{Игнорировать}. & Появляется диалоговое окно <<Обращение завершено>>. \\ \hline 
\nn & Нажать кнопку \kw{OK}. & Диалоговое окно и форма \kw{Поликлиника(платный)} закрываются. \\ \hline
\end{longtable}

