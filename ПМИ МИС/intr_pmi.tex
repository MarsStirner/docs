\newpage
\section{Назначение и цели документа}
Документ предначен для использования в качестве руководящего документа, регламентирующего проведение испытаний МИС. 

Цели документа:
\begin{itemize}
\item согласование характеристик поставляемого программного обеспечения с требованиями Заказчика;
\item повышение эффективности, упрощение организации испытаний;
\item снижение управленческих рисков;
\item получение нормативной базы для принятия решений по результатам испытаний.
\end{itemize}

\section{Термины, определения и сокращения}
\begin{longtable}{|p{1cm}|p{5cm}|p{10.5cm}|}
\hline \rule{0pt}{15pt} \centering \textbf{№ пп} & \centering \textbf{Обозначение} & \hfil \textbf{Расшифровка} \\ \hline
\endfirsthead
\hline \rule{0pt}{15pt} \centering \textbf{№ пп} & \centering \textbf{Обозначение} & \hfil \textbf{Расшифровка} \\
\endhead
1 & \textbf{АС} & Автоматизированная система \\ \hline
2 & \textbf{БД} & База данных \\ \hline
3 & \textbf{МУ} & Медицинское учреждение \\ \hline
4 & \textbf{ЭМК} & Электронная медицинская карта\\ \hline
5 & \textbf{МИС} & Медицинская информационная система\\ \hline
\end{longtable}

\section{Объект испытаний}
Объектом испытаний является медицинская информационная система персонифицированного учета оказания медицинской помощи (МИС).

В состав МИС входят:
\begin{itemize}
\item Модуль <<Поликлиника>>; 
\item Модуль <<Стационар>>; 
\item Модуль <<Параклиника>>;
\item Модуль <<Учет персонала>>.
\end{itemize}
Модули медицинской информационной системы могут объединяться в соответствии с потребностями и организационной структурой конкретного  медицинского учреждения (МУ). 

МИС предназначена для повышения качества лечебно-диагностического процесса, снижения нагрузки на медицинский персонал, предоставления оперативного доступа к медицинской информации пациента и статистическим данным МУ для принятия управленческих решений путем:
\begin{itemize}
\item автоматизации административной и медицинской деятельности при осуществлении лечебно-профилактических мероприятий;
\item обеспечения эффективного и качественного информационного взаимодействия учреждений и организаций системы здравоохранения;
\item обеспечения возможности интеграции с процессами управления поставками лекарственных средств;
\item обеспечения возможности взаимодействия с системами управления затратами для обработки и анализа затрат по оказанию медицинских услуг, как для пациента, так и для структурной единицы МУ. 
\end{itemize}
 
\section{Общие положения}
\subsection{Перечень руководящих документов}
\begin{enumerate}
\item ГОСТ 19.301-79 <<Программа и методика испытаний. Требования к содержанию и оформлению>>.
\item ГОСТ 19.101-77 <<Виды программ и программных документов>>
\item  РД 50-34.698-90 <<Автоматизированные системы. Требования к содержанию документов>>.
\item Техническое задание.
\end{enumerate}

\subsection{Виды испытаний}
Проверка соответствия функциональных и технических характеристик ПО;
оценка достаточности документации.
 

