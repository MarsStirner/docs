\subsection{Регистрация сотрудников и создание расписания}
\subsubsection{Регистрация нового сотрудника} \label{new_sotr}

\textbf{Необходимые начальные условия:} Должны быть выполнены первоначальные настройки справочников в системе.

\textbf{Роли пользователя:} Администратор.

\setcounter{nnn}{0}
\begin{longtable}{|p{1cm}|p{7.5cm}|p{8cm}|}
\caption{Регистрация нового сотрудника \label{new_sotr_tbl}}\\
\hline \rule{0pt}{15pt}  \centering \textbf{№ пп} & \centering \textbf{Действие} & \hfil \textbf{Ожидаемый результат} \\ \hline
\endfirsthead
\hline \rule{0pt}{15pt} \centering \textbf{№ пп} & \centering \textbf{Действие} & \hfil \textbf{Ожидаемый результат} \\ \hline
\endhead
\nn & В главном меню выбрать пункт \mm{Справочники \str Персонал \str Сотрудники}. & 	Открывается форма, содержащая список зарегистрированных сотрудников ЛПУ. \\ \hline
\nn & Нажить клавишу F9 на клавиатуре или кнопку \kw{Регистрация (F9)} в нижней части окна. & Открывается пустая форма \kw{Сотрудник}.\\ \hline
\nn & В открывшейся форме заполнить следующие поля: \newline \dm{Фамилия}: <<Дорофеев>>, \dm{Имя}:<<Руслан>>, \dm{Отчество}: <<Олегович>>, \dm{Код}: <<2299>>. & Данные внесены в соответствующие поля. \\ \hline
\nn & В поле \dm{Должность} выбрать из списка значение <<Врач-гастроэнтеролог>>, в поле \dm{Специальность} -- <<Гастроэнтеролог (терапия)>>.& Данные внесены в соответствующие поля. \\ \hline
\nn & В поле \dm{Регистрационное имя} ввести <<РДорофеев>>. & Указано имя пользователя для входа в \tmis.\\ \hline
\nn & В таблице \kw{Возможные роли} двойным щелчком левой кнопки мыши активировать список ролей в первой строке таблицы и выбрать <<Врач поликлиники>>. Установить курсор на следующую строку таблицы. & Добавлены настройки ролей данного пользователя. \\ \hline
\nn & Установить флажок в поле \dm{Изменить пароль} и в ставшее доступным поле справа ввести пароль нового пользователя для входа в \tmis. & Указан пароль нового пользователя.\\ \hline
\nn & Перейти на вкладку \kw{Личные}. & Осуществлен переход на вкладку \kw{Личные}. \\ \hline
\nn & Заполнить персональную информацию о сотруднике, внося данные в соответствующие поля. & Заполнены персональные данные о сотруднике. \\ \hline
\nn & Перейти на вкладку \kw{Кадровые перемещения}. & Осуществлен переход на вкладку \kw{Кадровые перемещения}.\\ \hline

\end{longtable}