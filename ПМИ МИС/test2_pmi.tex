\subsection{Регистрация сотрудников и создание расписания}
\subsubsection{Регистрация нового сотрудника} \label{new_sotr}

\textbf{Необходимые начальные условия:} Должны быть выполнены первоначальные настройки справочников в системе.

\textbf{Роли пользователя:} Отдел кадров, Администратор.

\setcounter{nnn}{0}
\begin{longtable}{|p{1cm}|p{7.5cm}|p{8cm}|}
\caption{Регистрация нового сотрудника \label{new_sotr_tbl}}\\
\hline \rule{0pt}{15pt}  \centering \textbf{№ пп} & \centering \textbf{Действие} & \hfil \textbf{Ожидаемый результат} \\ \hline
\endfirsthead
\hline \rule{0pt}{15pt} \centering \textbf{№ пп} & \centering \textbf{Действие} & \hfil \textbf{Ожидаемый результат} \\ \hline
\endhead
\nn & В главном меню выбрать пункт \mm{Справочники \str Персонал \str Сотрудники}. & 	Открывается форма \kw{Сотрудники}, содержащая список зарегистрированных сотрудников ЛПУ. \\ \hline
\nn & Нажить клавишу F9 на клавиатуре или кнопку \kw{Регистрация (F9)} в нижней части окна. & Открывается пустая форма \kw{Сотрудник}.\\ \hline
\nn & В открывшейся форме заполнить следующие поля: \newline \dm{Фамилия}: <<Дорофеев>>, \newline \dm{Имя}:<<Руслан>>, \newline \dm{Отчество}: <<Олегович>>, \newline \dm{Код}: <<2299>>. & Данные внесены в соответствующие поля. \\ \hline
\nn & В поле \dm{Подразделение} выбрать из дерева структуры ЛПУ <<Отделение амбулаторной хирургии и травмотологии>>, в поле \dm{Должность} выбрать из списка значение <<Врач-офтальмолог>>, в поле \dm{Специальность} -- <<Лечебное дело>>.& Данные внесены в соответствующие поля. \\ \hline
\nn & В поле \dm{Регистрационное имя} ввести <<РДорофеев>>. & Указано имя пользователя для входа в \tmis.\\ \hline
\nn & В таблице \kw{Возможные роли} двойным щелчком левой кнопки мыши активировать список ролей в первой строке таблицы и выбрать <<Врач поликлиники>>. Переместить курсор на следующую строку таблицы. & Добавлены настройки ролей данного пользователя. \\ \hline
\nn & Установить флажок в поле \dm{Изменить пароль} и в ставшее доступным поле справа ввести пароль для входа в \tmis~для данного пользователя. & Указан пароль пользователя.\\ \hline
\nn & Перейти на вкладку \kw{Личные}. & Осуществлен переход на вкладку \kw{Личные}. \\ \hline
\nn & Заполнить персональную информацию о сотруднике, внося данные в соответствующие поля. & Заполнены персональные данные о сотруднике. \\ \hline
\nn & Перейти на вкладку \kw{Кадровые перемещения}. & Осуществлен переход на вкладку \kw{Кадровые перемещения}.\\ \hline
\nn & Дважды щелкнуть левой кнопкой мыши в ячейке \dm{Дата} первой строки таблицы, в ячейке \dm{Тип перемещения} активировать список двойным щелчком левой кнопки мыши и выбрать значение <<Прием на работу>>, в ячейке \dm{Номер} ввести <<K123/10>>, в ячейке \dm{Тип документа} активировать список двойным щелчком левой кнопки мыши и выбрать <<Приказ (распоряжение) о приеме работника на работу>>, в ячейке \dm{Действителен с} дважды щелкнуть левой кнопкой мыши, в ячейке \dm{Должность} активировать список двойным щелчком левой кнопки мыши и выбрать <<Врач-офтальмолог>>, в поле \dm{Подразделение} выбрать из дерева структуры ЛПУ <<Отделение амбулаторной хирургии и травмотологоии>>. & Данные внесены в первую строку таблицы. В ячейках \dm{Дата} и \dm{Действителен с} указана текущая дата. \\ \hline
\nn & Перейти на вкладку \kw{График}. & Осуществлен переход на вкладку \kw{График}. \\ \hline
\nn & В нижней части формы заполнить поля следующим образом: \newline \dm{Амбулаторный прием}: <<10>>, \newline \dm{Первичная квота}: <<40>>, \newline \dm{Врачебная квота}: <<40>>, \newline \dm{Консультативная квота}: <<20>>, \newline \dm{Расписание видимо до}: указать последний день текущего месяца. & Данные внесены в соответствующие поля. \\ \hline
\nn & Нажать кнопку \kw{ОК}. & Текущая форма закрывается. Данные о сотруднике сохраняются в БД. \\ \hline
\nn &Нажать кнопку \kw{Закрыть}. & Форма \kw{Сотрудники} закрывается. \\ \hline
\end{longtable}

\subsubsection{Создание расписания по шаблону} \label{new_ttb}

\textbf{Необходимые начальные условия:} Должен быть выполнен п. \ref{new_sotr}

\textbf{Роли пользователя:} Старший регистратор, Администратор.

\setcounter{nnn}{0}
\begin{longtable}{|p{1cm}|p{7.5cm}|p{8cm}|}
\caption{Создание расписания по шаблону \label{new_ttb_tbl}}\\
\hline \rule{0pt}{15pt}  \centering \textbf{№ пп} & \centering \textbf{Действие} & \hfil \textbf{Ожидаемый результат} \\ \hline
\endfirsthead
\hline \rule{0pt}{15pt} \centering \textbf{№ пп} & \centering \textbf{Действие} & \hfil \textbf{Ожидаемый результат} \\ \hline
\endhead
\nn & В главном меню выбрать \mm{Работа \str Учет рабочего времени}. & Открывается форма \kw{График}. \\ \hline
\nn & В списке \kw{Структура ЛПУ} выбрать из дерева структуры ЛПУ <<Отделение амбулаторной хирургии и травмотологии>>, раскрывая соответствующие ветви дерева нажатием на знак <<$+$>> слева от ее названия. & В списке справа от текущего отображается список сотрудников выбранного подразделения. \\ \hline
\nn & В списке сотрудников в правом верхнем углу формы выбрать сотрудника <<2299 -- Дорофеев Руслан Олегович>>.& Курсор установлен на записи об указанном сотруднике. \\ \hline
\nn & Нажать кнопку \kw{Заполнить (F9)} в левом нижнем углу окна и в раскрывшемся списке выбрать <<По шаблону>>. & Открывается форма \kw{Шаблон планировщика}. \\ \hline
\nn & Убедиться, что в полях \dm{В период с (по)} указаны начальная и конечная даты текущего месяца, в левой части окна переключатель установлен на значение <<Нечет$\slash$чет>>. & Требуемые значения установлены.\\ \hline
\nn & В верхней строке основной части формы указать сдедующие значения: \newline \dm{Амбулаторный прием, Часы}: <<08:00 - 12:00>>, \newline \dm{Амбулаторный прием, Кабинет}: <<302>>, \newline \dm{Амбулаторный прием, План}: <<10>>.  & Данные внесены в соответствующие поля. \\ \hline
\nn & В следующей активной строке основной части формы (секция \kw{Четный день}) указать сдедующие значения: \newline \dm{Амбулаторный прием, Часы}: <<12:00 - 15:00>>, \newline \dm{Амбулаторный прием, Кабинет}: <<302>>, \newline \dm{Амбулаторный прием, План}: <<10>>.  & Данные внесены в соответствующие поля. \\ \hline
\nn & Установить флажок \dm{Округлять время приема врача до минут}. & Флажок установлен. \\ \hline
\nn & Нажать кнопку \kw{ОК}. & Текущая форма закрывается. Осуществляется возврат на форму \kw{График}. В средней части формы отображается сформированный график работы врача. \\ \hline
\nn &Нажать кнопку \kw{Закрыть}. & Форма \kw{График} закрывается. \\ \hline
\end{longtable}

\subsubsection{Создание расписания по шаблону <<скользящий график>>} \label{new_ttb2}

\textbf{Необходимые начальные условия:} Должен быть зарегистрирован сотрудник (см. п. \ref{new_sotr}).

\textbf{Роли пользователя:} Старший регистратор, Администратор.

\setcounter{nnn}{0}
\begin{longtable}{|p{1cm}|p{7.5cm}|p{8cm}|}
\caption{Создание расписания по шаблону <<скользящий график>> \label{new_ttb2_tbl}}\\
\hline \rule{0pt}{15pt}  \centering \textbf{№ пп} & \centering \textbf{Действие} & \hfil \textbf{Ожидаемый результат} \\ \hline
\endfirsthead
\hline \rule{0pt}{15pt} \centering \textbf{№ пп} & \centering \textbf{Действие} & \hfil \textbf{Ожидаемый результат} \\ \hline
\endhead
\nn & В главном меню выбрать \mm{Работа \str Учет рабочего времени}. & Открывается форма \kw{График}. \\ \hline
\nn & В списке \kw{Структура ЛПУ} выбрать из дерева структуры ЛПУ <<Отделение амбулаторной хирургии и травмотологии>>, раскрывая соответствующие ветви дерева нажатием на знак <<$+$>> слева от ее названия. & В списке справа от текущего отображается список сотрудников выбранного подразделения. \\ \hline
\nn & В списке сотрудников в правом верхнем углу формы выбрать сотрудника <<2299 -- Дорофеев Руслан Олегович>>.& Курсор установлен на записи об указанном сотруднике. \\ \hline
\nn & Нажать кнопку \kw{Заполнить (F9)} в левом нижнем углу окна и в раскрывшемся списке выбрать <<По шаблону<<скользящий график>>. & Открывается форма \kw{Шаблон планировщика}. \\ \hline
\nn & Последовательно щелкнуть левой кнопкой мыши по числам 1, 3, 5, 6, 8, 11, 15 в календаре, расположенном в центральной части формы. & В списке \kw{Выбранные дни} справа от календаря указано <<1, 3, 5, 6, 8, 11, 15>>, в поле <<Всего выбрано дней>> -- <<7>>. \\ \hline
\nn & В верхней строке основной части формы указать сдедующие значения: \newline \dm{Амбулаторный прием, Часы}: <<09:00 - 11:00>>, \newline \dm{Амбулаторный прием, Кабинет}: <<308>>, \newline \dm{Амбулаторный прием, План}: <<9>>.  & Данные внесены в соответствующие поля. \\ \hline
\nn & Установить флажки \dm{Заполнять выходные дни} и  \dm{Округлять время приема врача до минут}. & Флажки установлены. \\ \hline
\nn & Нажать кнопку \kw{ОК}. & Текущая форма закрывается. Осуществляется возврат на форму \kw{График}. В графике работы врача в средней части формы изменен график работы на 1, 3, 5, 6, 8, 11 и 15 числа месяца. \\ \hline
\nn &Нажать кнопку \kw{Закрыть}. & Форма \kw{График} закрывается. \\ \hline
\end{longtable}

\subsubsection{Создание расписания по индивидуальному графику} \label{new_ttb3}

\textbf{Необходимые начальные условия:} Должен быть зарегистрирован сотрудник (см. п. \ref{new_sotr}).

\textbf{Роли пользователя:} Старший регистратор, Администратор.

\setcounter{nnn}{0}
\begin{longtable}{|p{1cm}|p{7.5cm}|p{8cm}|}
\caption{Создание расписания по индивидуальному графику \label{new_ttb3_tbl}}\\
\hline \rule{0pt}{15pt}  \centering \textbf{№ пп} & \centering \textbf{Действие} & \hfil \textbf{Ожидаемый результат} \\ \hline
\endfirsthead
\hline \rule{0pt}{15pt} \centering \textbf{№ пп} & \centering \textbf{Действие} & \hfil \textbf{Ожидаемый результат} \\ \hline
\endhead
\nn & В главном меню выбрать пункт \mm{Справочники \str Персонал \str Сотрудники}. & Открывается форма \kw{Сотрудники}. \\ \hline
\nn & Найти сотрудника <<2299 -- Дорофеев Руслан Олегович>> и щелкнуть по записи левой кнопкой мыши. & Курсор установлен на записи об указанном сотруднике. \\ \hline
\nn & Нажать кнопку \kw{Правка F4} или клавишу F4 на клавиатуре. & Открывается форма \kw{Сотрудник}, содержащая данные выбранного сотрудника. Данные доступны для редактирования.\\ \hline
\nn & Перейти на вкладку  \kw{График}. & Осуществлен переход на вкладку \kw{График}.\\ \hline
\nn & В верхней строке выбрать <<Нечет \slash чет>>. & В таблице ниже появилось 2 строки с заголовками: <<Нечетный день>> и <<Четный день>>.\\ \hline
\nn & В строке <<Нечетный день>> указать сдедующие значения: \newline \dm{Амбулаторно}: <<09:00 - 11:00>>, \newline \dm{Каб.}: <<302>>, \newline \dm{План}: <<8>>, \newline \dm{Амбулаторно2}: <<12:00 - 14:00>>, \newline \dm{Каб.2}: <<302>>, \newline \dm{План}: <<8>>. \newline Остальные поля оставить незаполненными. & Данные внесены в соответствующие поля. \\ \hline
\nn & В строке <<Четный день>> указать сдедующие значения: \newline \dm{Амбулаторно}: <<08:00 - 11:00>>, \newline \dm{Каб.}: <<302>>, \newline \dm{План}: <<8>>, \newline \dm{Вызовы}: <<14:00 - 16:00>>, \newline \dm{План}: <<6>>. \newline Остальные поля оставить незаполненными. & Данные внесены в соответствующие поля. \\ \hline
\nn & Нажать кнопку \kw{OK}.& Текущая форма закрывается. Изменения данных сотрудника сохраняется в БД. \\ \hline
\nn &Нажать кнопку \kw{Закрыть}. & Форма \kw{Сотрудники} закрывается. \\ \hline
\nn & В главном меню выбрать \mm{Работа \str Учет рабочего времени}. & Открывается форма \kw{График}. \\ \hline
\nn & В списке \kw{Структура ЛПУ} выбрать из дерева структуры ЛПУ <<Отделение амбулаторной хирургии и травмотологии>>, раскрывая соответствующие ветви дерева нажатием на знак <<$+$>> слева от ее названия. & В списке справа от текущего отображается список сотрудников выбранного подразделения. \\ \hline
\nn & В списке сотрудников в правом верхнем углу формы выбрать сотрудника <<2299 -- Дорофеев Руслан Олегович>>.& Курсор установлен на записи об указанном сотруднике. В средней части формы отображается расписание выбранного сотрудника. \\ \hline
\nn & Нажать кнопку \kw{Заполнить (F9)} в левом нижнем углу окна и в раскрывшемся списке выбрать <<По персональному графику>>. & Открывается форма \kw{Персональный шаблон}. \\ \hline
\nn & Убедиться, что в полях \dm{В период с (по)} указаны первое и последнее число текущего месяца. & Поля заполнены верно. \\ \hline
\nn & Установить флажки \dm{Ввод второго периода приема} и  \dm{Округлять время приема врача до минут}. & Флажки установлены. \\ \hline
\nn & Нажать кнопку \kw{ОК}. & Текущая форма закрывается. Осуществляется возврат на форму \kw{График}. В графике работы врача в средней части формы \kw{График} график работы заменен. График работы соответствует ранее сделанным настройкам на форме \kw{Сотрудник}. \\ \hline
\nn &  Дважды щелкнуть левой кнопкой мыши в ячейке \dm{Амбулаторно} нечетного рабочего дня. & Открывается форма \kw{Редактор периодов}, содержащая список из двух периодов приема. \\ \hline
\nn & Нажать кнопку \kw{Oтмена}. & Текущая форма закрывается. \\ \hline
\nn &Нажать кнопку \kw{Закрыть}. & Форма \kw{График} закрывается. \\ \hline
\end{longtable}

\subsubsection{Квотирование времени приема по видам записи} \label{new_kvot}

\textbf{Необходимые начальные условия:} Должен быть зарегистрирован сотрудник согласно п. \ref{new_sotr} Должно быть создано расписание сотрудника согласно п. \ref{new_ttb3}

\textbf{Роли пользователя:} Старший регистратор, Администратор.

\setcounter{nnn}{0}
\begin{longtable}{|p{1cm}|p{7.5cm}|p{8cm}|}
\caption{Квотирование времени приема \label{new_kvot_tbl}}\\
\hline \rule{0pt}{15pt}  \centering \textbf{№ пп} & \centering \textbf{Действие} & \hfil \textbf{Ожидаемый результат} \\ \hline
\endfirsthead
\hline \rule{0pt}{15pt} \centering \textbf{№ пп} & \centering \textbf{Действие} & \hfil \textbf{Ожидаемый результат} \\ \hline
\endhead
\nn & В главном меню выбрать \mm{Работа \str Учет рабочего времени}. & Открывается форма \kw{График}. \\ \hline
\nn & В списке \kw{Структура ЛПУ} выбрать из дерева структуры ЛПУ <<Отделение амбулаторной хирургии и травмотологии>>, раскрывая соответствующие ветви дерева нажатием на знак <<$+$>> слева от ее названия. & В списке справа от текущего отображается список сотрудников выбранного подразделения. \\ \hline
\nn & В списке сотрудников в правом верхнем углу формы выбрать сотрудника <<2299 -- Дорофеев Руслан Олегович>>.& Курсор установлен на записи об указанном сотруднике. В средней части формы отображается расписание выбранного сотрудника. \\ \hline
\nn & Удерживая клавишу Ctrl на клавиатуре последовательно нажать на все номера нечетных рабочих дней в первом столбце таблицы расписания сотрудника, расположенного в средней части формы. & Выделены все рабочие нечетные дни месяца в расписании сотрудника.\\ \hline
\nn & В правом нижнем углу формы в таблице \kw{Распреление времени для предварительной записи}, двойным щелчком левой кнопки мыши активировать список в ячейке \dm{Вид записи} и выбрать значение <<Запись из регистратуры>>, в ячейке \dm{Начало периода} ввести <<12:00>>, \dm{Конец периода} -- <<14:00>>. & В таблицу \kw{Распреление времени для предварительной записи} добавлена 1 строка.\\ \hline
\nn & В следующей строке таблицы \kw{Распреление времени для предварительной записи}, двойным щелчком левой кнопки мыши активировать список в ячейке \dm{Вид записи} и выбрать значение <<Запись врачом на повторный прием>>, в ячейке \dm{Начало периода} ввести <<12:00>>, \dm{Конец периода} -- <<14:00>>. & В таблицу \kw{Распреление времени для предварительной записи} добавлена еще одна строка.\\ \hline
\nn & В следующей строке таблицы \kw{Распреление времени для предварительной записи}, двойным щелчком левой кнопки мыши активировать список в ячейке \dm{Вид записи} и выбрать значение <<Межкабинетная запись>>, в ячейке \dm{Начало периода} ввести <<09:00>>, \dm{Конец периода} -- <<11:00>>. & В таблице \kw{Распреление времени для предварительной записи} содержится 3 строки.\\ \hline
\nn & Нажать кнопку \kw{Заполнить}. & Выполняется заполнение квот для выбранных дней. \\ \hline
\nn & Последовательно нажать левой кнопкой мыши по различным дням расписания, обращая внимание на данные в таблице \kw{Распреление времени для предварительной записи}. & Для нечетных рабочих дней заполнено квотирование. \\ \hline
\nn &Нажать кнопку \kw{Закрыть}. & Форма \kw{График} закрывается. \\ \hline
\end{longtable}
