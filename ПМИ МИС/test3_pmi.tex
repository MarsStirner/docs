\subsection{Запись пациентов на прием и обслуживание в поликлинике}
\subsubsection{Запись на прием по графику} \label{reg_pat1}

\textbf{Необходимые начальные условия:} Должно быть заполнено расписание сотрудника согласно п. \ref{new_ttb3} Должен быть зарегистрирован пациент согласно п. \ref{new_client}

\textbf{Роли пользователя:} Регистратор, Старший регистратор, Администратор.

\setcounter{nnn}{0}
\begin{longtable}{|p{1cm}|p{7.5cm}|p{8cm}|}
\caption{Запись пациента на прием \label{reg_pat1_tbl}}\\
\hline \rule{0pt}{15pt}  \centering \textbf{№ пп} & \centering \textbf{Действие} & \hfil \textbf{Ожидаемый результат} \\ \hline
\endfirsthead
\hline \rule{0pt}{15pt} \centering \textbf{№ пп} & \centering \textbf{Действие} & \hfil \textbf{Ожидаемый результат} \\ \hline
\endhead
\nn & В главном меню выбрать пункт \mm{Работа \str Обслуживание пациентов}. & Открывается форма, содержащая картотеку пациентов. \\ \hline
\nn & Если с левой стороны экрана отсутствует панель \kw{График}, выбрать в главном меню пункт \mm{Настройки \str График}. & С левой стороны экрана находится панель \kw{График}. \\ \hline
\nn & На панели \kw{График} в списке \kw{Структура ЛПУ} выбрать из дерева <<Отделение амбулаторной хирургии и травмотологии>>, раскрывая нужные ветви дерева нажатием знака <<$+$>> слева от названия подразделения. & Выбрано <<Отделение амбулаторной хирургии и травмотологии>> из дерева структуры ЛПУ. В списке \kw{Персонал} отображаются сотрудники только выбранного подразделения. \\ \hline
\nn & В списке \kw{Персонал} раскрыть группу <<Лечебное дело>>, щелкнув левой кнопкой мыши по знаку <<$+$>> слева от названия и из раскрывшегося списка выбрать <<Дорофеев Р.О.>>. & В средней части панели \kw{График} отображается расписание работы выбранного сотрудника на текущий месяц. \\ \hline
\nn & В календаре на панели \kw{График} выбрать текущую дату. & В нижней части панели \kw{График} отображается список свободных и занятых интервалов для записи к выбранному врачу на текущий день. \\ \hline
\nn & На панели \kw{Фильтр} в правой части экрана установить флажок \dm{Фамилия} и в ставшее активным поле ввести <<Иванов>>, установить флажок \dm{Д.рожд.} и в ставшее активным поле ввести <<12.12.1961>>. & Заданы параметры поиска пациента. \\ \hline
\nn & Нажать кнопку \kw{Применить} на панели \kw{Фильтр}. & Отображается список пациентов, найденных по заданным параметрам поиска. \\ \hline
\nn & Установить курсор на пациенте <<Иванов Иван Васильевич, 12.12.1961>>. & Курсор установлен на указанном пациенте. В верхней части отображается краткая информация о выбранном пациенте. \\ \hline
\nn & В нижней части панели \kw{График} выбрать свободный интервал и щелкнуть по нему правой кнопкой мыши. & Появляется контекстное меню для интервала времени. \\ \hline
\nn & В контекстном меню выбрать пункт \kw{Поставить в очередь}. & Открывается форма \kw{Вид приема}. \\ \hline
\nn &  Установить переключатель на <<Амбулаторный пациент>> и нажать кнопку \kw{ОК}. & Открывается окно предварительного просмотра печатной формы \kw{Талон на прием к врачу}. \\ \hline
\nn & Необязательный шаг: Нажать кнопку \kw{Печатать} и направить печать на нужный принтер. & Выполнена печать талона. \\ \hline
\nn & Нажать кнопку \kw{Закрыть}. & Пациент записан  на прием. В выбранный интервал для записи в нижней части панели \kw{График} внесена фамилия пациента <<Иванов Иван Васильевич, 12.12.1961 М>>. \\ \hline
\end{longtable}

\subsubsection{Отмена записи на прием} \label{unreg_pat1}

\textbf{Необходимые начальные условия:} Должна быть выполнена запись пациента на прием согласно п. \ref{reg_pat1} 

\textbf{Роли пользователя:} Регистратор, Старший регистратор, Администратор.

\setcounter{nnn}{0}
\begin{longtable}{|p{1cm}|p{7.5cm}|p{8cm}|}
\caption{Отмена записи на прием \label{unreg_pat_tbl}}\\
\hline \rule{0pt}{15pt}  \centering \textbf{№ пп} & \centering \textbf{Действие} & \hfil \textbf{Ожидаемый результат} \\ \hline
\endfirsthead
\hline \rule{0pt}{15pt} \centering \textbf{№ пп} & \centering \textbf{Действие} & \hfil \textbf{Ожидаемый результат} \\ \hline
\endhead
\nn & В главном меню выбрать пункт \mm{Работа \str Обслуживание пациентов}. & Открывается форма, содержащая картотеку пациентов. \\ \hline
\nn & Если с левой стороны экрана отсутствует панель \kw{График}, выбрать в главном меню пункт \mm{Настройки \str График}. & С левой стороны экрана находится панель \kw{График}. \\ \hline
\nn & На панели \kw{График} в списке \kw{Структура ЛПУ} выбрать из дерева <<Отделение амбулаторной хирургии и травмотологии>>, раскрывая нужные ветви дерева нажатием знака <<$+$>> слева от названия подразделения. & Выбрано <<Отделение амбулаторной хирургии и травмотологии>> из дерева структуры ЛПУ. В списке \kw{Персонал} отображаются сотрудники только выбранного подразделения. \\ \hline
\nn & В списке \kw{Персонал} раскрыть группу <<Лечебное дело>>, щелкнув левой кнопкой мыши по знаку <<$+$>> слева от названия и из раскрывшегося списка выбрать <<Дорофеев Р.О.>>. & В средней части панели \kw{График} отображается расписание работы выбранного сотрудника на текущий месяц. \\ \hline
\nn & В календаре на панели \kw{График} выбрать текущую дату. & В нижней части панели \kw{График} отображается список свободных и занятых интервалов для записи к выбранному врачу на текущий день. \\ \hline
\nn & В нижней части панели \kw{График} выбрать интервал времени, на который записан пациент <<Иванов Иван Васильевич, 12.12.1961 М>> и щелкнуть по нему правой кнопкой мыши. & Раскрывается контекстное меню для интевала записи на прием. \\ \hline
\nn & В контекстном меню выбрать пункт \kw{Удалить из очереди}. & Появляется диалоговое окно <<Подтвердите удаление записи к врачу>>. \\ \hline
\nn & В диалоговом окне нажать кнопку \kw{ОК}. & Выбранный интервал времени освобожден. \\ \hline
\end{longtable}

\subsubsection{Выдача пациенту номерков} \label{reg_pat2}

\textbf{Необходимые начальные условия:} Должно быть заполнено расписание сотрудника согласно п. \ref{new_ttb3} Должен быть зарегистрирован пациент согласно п. \ref{new_client}

\textbf{Роли пользователя:} Регистратор, Старший регистратор, Администратор.

\setcounter{nnn}{0}
\begin{longtable}{|p{1cm}|p{7.5cm}|p{8cm}|}
\caption{Выдача пациенту номерков \label{reg_pat2_tbl}}\\
\hline \rule{0pt}{15pt}  \centering \textbf{№ пп} & \centering \textbf{Действие} & \hfil \textbf{Ожидаемый результат} \\ \hline
\endfirsthead
\hline \rule{0pt}{15pt} \centering \textbf{№ пп} & \centering \textbf{Действие} & \hfil \textbf{Ожидаемый результат} \\ \hline
\endhead
\nn & В главном меню выбрать пункт \mm{Работа \str Обслуживание пациентов}. & Открывается форма, содержащая картотеку пациентов. \\ \hline
\nn & Если с левой стороны экрана отсутствует вкладка \kw{Номерки}, выбрать в главном меню пункт \mm{Настройки \str Номерки}. & С левой стороны экрана присутствует вкладка \kw{Номерки}. \\ \hline
\nn & На вкладке \kw{Номерки} в поле \dm{Дата} указать текущую дату, установить флажок в поле \dm{Время с (по)} и в ставших активными полях указать <<08:00>> и <<13:00>>. & Указаны желаемые дата и время приема. \\ \hline
\nn & В списке \kw{Структура ЛПУ} выбрать из дерева <<Отделение амбулаторной хирургии и травмотологии>>, раскрывая нужные ветви дерева нажатием знака <<$+$>> слева от названия подразделения. & Выбрано <<Отделение амбулаторной хирургии и травмотологии>> из дерева структуры ЛПУ. В списке \kw{Персонал} отображаются специальности сотрудников только выбранного подразделения. \\ \hline
\nn & В списке \kw{Персонал} выбрать группу <<Лечебное дело>>. & В нижней части панели \kw{Номерки} отображается список свободных номерков врачей выбранной специальности, начиная с указанной даты, на выбранный период времени. \\ \hline
\nn & На панели \kw{Фильтр} в правой части экрана установить флажок \dm{Фамилия} и в ставшее активным поле ввести <<Иванов>>, установить флажок \dm{Д.рожд.} и в ставшее активным поле ввести <<12.12.1961>>. & Заданы параметры поиска пациента. \\ \hline
\nn & Нажать кнопку \kw{Применить} на панели \kw{Фильтр}. & Отображается список пациентов, найденных по заданным параметрам поиска. \\ \hline
\nn & Установить курсор на пациенте <<Иванов Иван Васильевич, 12.12.1961>>. & Курсор установлен на указанном пациенте. В верхней части отображается краткая информация о выбранном пациенте. \\ \hline
\nn \label{nn123} & В нижней части вкладки \kw{Номерки} выбрать свободный номерок на ближайшую дату  и дважды щелкнуть по нему левой кнопкой мыши. & Выполнена выдача пациенту номерка. \\ \hline
\nn & В списке номерков переместить курсор на другую строку. & Номерок, который был выдан пациенту, исчез из списка. \\ \hline
\nn & Перейти на вкладку \kw{Графики}. & Осуществлен переход на вкладку \kw{Графики}. \\ \hline
\nn & На панели \kw{График} в списке \kw{Структура ЛПУ} выбрать из дерева <<Отделение амбулаторной хирургии и травмотологии>>, раскрывая нужные ветви дерева нажатием знака <<$+$>> слева от названия подразделения. & Выбрано <<Отделение амбулаторной хирургии и травмотологии>> из дерева структуры ЛПУ. В списке \kw{Персонал} отображаются сотрудники только выбранного подразделения. \\ \hline
\nn & В списке \kw{Персонал} раскрыть группу <<Лечебное дело>>, щелкнув левой кнопкой мыши по знаку <<$+$>> слева от названия и из раскрывшегося списка выбрать <<Дорофеев Р.О.>>. & В средней части панели \kw{График} отображается расписание работы выбранного сотрудника на текущий месяц. \\ \hline
\nn & В календаре на панели \kw{График} выбрать дату, на которую в пп. \ref{nn123} был выдан номерок. & В нижней части панели \kw{График} отображается список свободных и занятых интервалов для записи к выбранному врачу на текущий день. Пациент <<Иванов Иван Васильевич, 12.12.1961 М>> присутствует в списке. \\ \hline
\end{longtable}

\subsubsection{Регистрация вызова врача на дом} \label{reg_home}

\textbf{Необходимые начальные условия:} Должно быть заполнено расписание сотрудника согласно п. \ref{new_ttb3} Должен быть зарегистрирован пациент согласно п. \ref{new_client}

\textbf{Роли пользователя:} Регистратор, Старший регистратор, Администратор.

\setcounter{nnn}{0}
\begin{longtable}{|p{1cm}|p{7.5cm}|p{8cm}|}
\caption{Регистрация вызова врача на дом \label{reg_home_tbl}}\\
\hline \rule{0pt}{15pt}  \centering \textbf{№ пп} & \centering \textbf{Действие} & \hfil \textbf{Ожидаемый результат} \\ \hline
\endfirsthead
\hline \rule{0pt}{15pt} \centering \textbf{№ пп} & \centering \textbf{Действие} & \hfil \textbf{Ожидаемый результат} \\ \hline
\endhead
\nn & В главном меню выбрать пункт \mm{Работа \str Обслуживание пациентов}. & Открывается форма, содержащая картотеку пациентов. \\ \hline
\nn & В левом нижнем углу экрана выбрать вкладку \kw{График}. Если с левой стороны экрана отсутствует панель \kw{График}, выбрать в главном меню пункт \mm{Настройки \str График}. & С левой стороны экрана находится панель \kw{График}. \\ \hline
\nn & На панели \kw{График} в списке \kw{Структура ЛПУ} выбрать из дерева <<Отделение амбулаторной хирургии и травмотологии>>, раскрывая нужные ветви дерева нажатием знака <<$+$>> слева от названия подразделения. & Выбрано <<Отделение амбулаторной хирургии и травмотологии>> из дерева структуры ЛПУ. В списке \kw{Персонал} отображаются сотрудники только выбранного подразделения. \\ \hline
\nn & В списке \kw{Персонал} раскрыть группу <<Лечебное дело>>, щелкнув левой кнопкой мыши по знаку <<$+$>> слева от названия и из раскрывшегося списка выбрать <<Дорофеев Р.О.>>. & В средней части панели \kw{График} отображается расписание работы выбранного сотрудника на текущий месяц. \\ \hline
\nn & В таблице под календарем выбрать вкладку \kw{На дому}. & В таблице отображается график обслуживания вызовов на дом выбранным врачом. \\ \hline
\nn & В таблице \kw{На дому} выбрать ближайшую дату, на которую есть обслуживание вызовов на дом и щелкнуть по ней левой кнопкой мыши. & В нижней части панели \kw{График} отображается список свободных и занятых интервалов для вызова врача на выбранный день. \\ \hline
\nn & На панели \kw{Фильтр} в правой части экрана установить флажок \dm{Фамилия} и в ставшее активным поле ввести <<Иванов>>, установить флажок \dm{Д.рожд.} и в ставшее активным поле ввести <<12.12.1961>>. & Заданы параметры поиска пациента. \\ \hline
\nn & Нажать кнопку \kw{Применить} на панели \kw{Фильтр}. & Отображается список пациентов, найденных по заданным параметрам поиска. \\ \hline
\nn & Установить курсор на пациенте <<Иванов Иван Васильевич, 12.12.1961>>. & Курсор установлен на указанном пациенте. В верхней части отображается краткая информация о выбранном пациенте. \\ \hline
\nn & В нижней части панели \kw{График} выбрать свободный интервал и дважды щелкнуть по нему левой кнопкой мыши. & Появляется диалоговое окно <<Этот пациент уже записан к врачу данной специальности. Подтвердите повторную запись>>. \\ \hline
\nn & Нажать кнопку \kw{ОК}. & Открывается форма \kw{Жалобы}. \\ \hline
\nn & В группе <<Давление>> дважды щелкнуть левой кнопкой мыши по записи <<высокое>>. & В правой части формы \kw{Жалобы} появилась строка <<Давление: высокое>>.\\ \hline
\nn & В группе <<Боль>> дважды щелкнуть левой кнопкой мыши по записи <<головная>>. & В правой части формы \kw{Жалобы} отображается <<Давление: Высокое, Боль: головная>>.\\ \hline
\nn & Нажать кнопку \kw{ОК}. & Вызов пациента зарегистрирован. В выбранный интервал для записи в нижней части панели \kw{График} внесена фамилия пациента <<Иванов Иван Васильевич, 12.12.1961 М>>. \\ \hline
\end{longtable}

\subsubsection{Обслуживание пациента по предварительной записи} \label{obsl1_pol}

\textbf{Необходимые начальные условия:} Должна быть выполнена предварительная запись пациента на прием согласно п. \ref{reg_pat2}

\textbf{Роли пользователя:} Регистратор, Старший регистратор, Врач поликлиники, Администратор.

\setcounter{nnn}{0}
\begin{longtable}{|p{1cm}|p{7.5cm}|p{8cm}|}
\caption{Обслуживание пациента по предварительной записи \label{obsl1_ pol_tbl}}\\
\hline \rule{0pt}{15pt}  \centering \textbf{№ пп} & \centering \textbf{Действие} & \hfil \textbf{Ожидаемый результат} \\ \hline
\endfirsthead
\hline \rule{0pt}{15pt} \centering \textbf{№ пп} & \centering \textbf{Действие} & \hfil \textbf{Ожидаемый результат} \\ \hline
\endhead
\nn & В главном меню выбрать пункт \mm{Работа \str Обслуживание пациентов}. & Открывается форма, содержащая картотеку пациентов. \\ \hline
\nn & На панели \kw{Фильтр} в правой части экрана установить флажок \dm{Фамилия} и в ставшее активным поле ввести <<Иванов>>, установить флажок \dm{Д.рожд.} и в ставшее активным поле ввести <<12.12.1961>>. & Заданы параметры поиска пациента. \\ \hline
\nn & Нажать кнопку \kw{Применить} на панели \kw{Фильтр}. & Отображается список пациентов, найденных по заданным параметрам поиска. \\ \hline
\nn & Установить курсор на пациенте <<Иванов Иван Васильевич, 12.12.1961>>. & Курсор установлен на указанном пациенте. В верхней части отображается краткая информация о выбранном пациенте. \\ \hline
\nn & В низу основной части формы в таблице \kw{Предварительная запись} выбрать запись к врачу Дорофееву Р.О., в которой ячейка \dm{Кабинет} заполнена и дважды щелкнуть по ней левой кнопкой мыши либо щелкнуть правой кнопкой мыщи и в появившемся контекстном меню выбрать пункт \kw{Новое обращение}. & Открывается форма \kw{Новое обращение} для пациента <<Иванов Иван Васильевич, 12.12.1961>>. В перечисленных полях указаны следющие значения: \newline \dm{Организация}: <<ГКП на ПХВ <<СГП>> >>, \newline \dm{Тип обращения}: <<Поликлиника>>, \newline \dm{Лечащий врач}: <<Дорофеев Р.О., офтальмолог>>, \newline \dm{Отделение}: <<Отделение амбулаторной хирургии и травмотологии>>, \newline \dm{Дата начала}: текущие дата и время, \newline \dm{Дата выполнения}: не заполнено. \\ \hline
\nn & В поле  \dm{Источник финансирования} выбрать из списка <<Платные услуги>>. & В перечисленные ниже поля автоматически подставились следующие значения: \newline \dm{Договор}: номер договора на платные услуги, \newline \dm{Тип события}: <<Поликлиника(платный)>>. \\ \hline
\nn & Нажать кнопку \kw{Создать}. & Текущая форма закрывается. Открывается форма \kw{Поликлиника (платный)}. Все поля, заполненные на предыдущем шаге, содержат соответствующие значения. \\ \hline
\nn & Нажать кнопку \kw{Сохранить}. & Появляется диалоговое окно, содержащее сообщение <<Данные успешно сохранены>>. \\ \hline
\nn & Нажать кнопку \kw{OK}. & Диалоговое окно закрывается. \\ \hline
\nn & Нажать кнопку \kw{Закрыть без сохранения}. & Текущая форма закрывается. Осуществляется возврат на форму, содержащую картотеку пациентов. \\ \hline
\end{longtable}

\subsubsection{Регистрация обращения без предварительной записи} \label{obsl2_pol}

\textbf{Необходимые начальные условия:} Должен быть зарегистрирован пациент согласно п. \ref{new_client}

\textbf{Роли пользователя:} Регистратор, Старший регистратор, Врач поликлиники, Администратор.

\setcounter{nnn}{0}
\begin{longtable}{|p{1cm}|p{7.5cm}|p{8cm}|}
\caption{Регистрация обращения без предварительной записи \label{obsl2_ pol_tbl}}\\
\hline \rule{0pt}{15pt}  \centering \textbf{№ пп} & \centering \textbf{Действие} & \hfil \textbf{Ожидаемый результат} \\ \hline
\endfirsthead
\hline \rule{0pt}{15pt} \centering \textbf{№ пп} & \centering \textbf{Действие} & \hfil \textbf{Ожидаемый результат} \\ \hline
\endhead
\nn & В главном меню выбрать пункт \mm{Работа \str Обслуживание пациентов}. & Открывается форма, содержащая картотеку пациентов. \\ \hline
\nn & На панели \kw{Фильтр} в правой части экрана установить флажок \dm{Фамилия} и в ставшее активным поле ввести <<Иванов>>, установить флажок \dm{Д.рожд.} и в ставшее активным поле ввести <<12.12.1961>>. & Заданы параметры поиска пациента. \\ \hline
\nn & Нажать кнопку \kw{Применить} на панели \kw{Фильтр}. & Отображается список пациентов, найденных по заданным параметрам поиска. \\ \hline
\nn & Установить курсор на пациенте <<Иванов Иван Васильевич, 12.12.1961>>. & Курсор установлен на указанном пациенте. В верхней части отображается краткая информация о выбранном пациенте. \\ \hline
\nn & Перейти на вкладку \kw{Обращение}. & Осуществляется переход на вкладку \kw{Обращение}. \\ \hline
\nn & Нажать клавишу F9 на клавиатуре либо кнопку \kw{Новый (F9)} в нижней части формы. & Открывается форма \kw{Новое обращение} для пациента <<Иванов Иван Васильевич, 12.12.1961>>. В перечисленных полях указаны следющие значения: \newline \dm{Организация}: <<ГКП на ПХВ <<СГП>>>>, \newline \dm{Тип обращения}: <<Поликлиника>>,  \newline \dm{Дата начала}: текущие дата и время, \newline \dm{Дата выполнения}: не заполнено. \\ \hline
\nn & В перечисленных полях выбрать из списка следующие значения: \newline \dm{Лечащий врач}: <<Молоткова С.В., Лечебное дело>>, \newline \dm{Источник финансирования}: <<Платные услуги>>. & В перечисленные ниже поля автоматически подставлены следующие значения: \newline \dm{Договор}: номер договора на платные услуги, \newline \dm{Тип события}: <<Поликлиника(платный)>>, \newline \dm{Отделение}: <<Терапевтическое отделение>>. \\ \hline
\nn & Нажать кнопку \kw{Создать}. & Текущая форма закрывается. Открывается форма \kw{Поликлиника (платный)}. Все поля, заполненные на предыдущем шаге, содержат соответствующие значения. \\ \hline
\nn & Нажать кнопку \kw{Сохранить}. & Появляется диалоговое окно, содержащее сообщение <<Данные успешно сохранены>>. \\ \hline
\nn & Нажать кнопку \kw{OK}. & Диалоговое окно закрывается. \\ \hline
\nn & Нажать кнопку \kw{Закрыть без сохранения}. & Текущая форма закрывается. Осуществляется возврат на форму, содержащую картотеку пациентов. \\ \hline
\end{longtable}