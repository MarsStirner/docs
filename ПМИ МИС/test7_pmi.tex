\subsection{Лечебный процесс в стационаре} \label{lecp_st}
\subsubsection{Регистрация поступления пациента в клиническое отделение} \label{reg_otd_st}

\textbf{Необходимые начальные условия:} Должны быть выполнены пп. \ref{reg_po_st} и \ref{osmotr_po_st}

\textbf{Роли пользователя:} Медсестра отделения, Администратор.

\setcounter{nnn}{0}
\begin{longtable}{|p{1cm}|p{7.5cm}|p{8cm}|}
\caption{Регистрация поступления пациента в клиническое отделение \label{lecp_st_tbl}}\\
\hline \rule{0pt}{15pt}  \centering \textbf{№ пп} & \centering \textbf{Действие} & \hfil \textbf{Ожидаемый результат} \\ \hline
\endfirsthead
\hline \rule{0pt}{15pt} \centering \textbf{№ пп} & \centering \textbf{Действие} & \hfil \textbf{Ожидаемый результат} \\ \hline
\endhead
\nn & В главном меню выбрать пункт \mm{Работа \str Стационарный монитор}. & Открывается форма \kw{Стационарный монитор}. \\ \hline
\nn & Перейти на вкладку \kw{Поступили}. & Осуществлен переход на вкладку \kw{Поступили}. \\ \hline
\nn & В правом верхнем углу формы в дереве структуры ЛПУ установить курсор на строке <<Акушерско-гинекологическое отделение>>, раскрывая необходимые ветви дерева нажатим кнопки <<$+$>> слева от ее названия. & Курсор установлен на указанном отделении. \\ \hline
\nn & В левой части формы на панели \kw{Фильтр} в поле \dm{Начало} указать текущую дату (либо дату выполнения п. \ref{reg_po_st}, если он выполнялся ранее). & Установлены параметры фильтрации обращений. \\ \hline
\nn & Нажать кнопку \kw{Применить} в правом нижнем углу панели \kw{Фильтр}. & Выполнена фильтрация обращений. В списке в правой части формы присутствует запись <<Калинина Динара Павловна, Ж, 20.06.1980>>. \\ \hline
\nn & Дважды щелкнуть левой кнопкой мыши по записи <<Калинина Динара Павловна, Ж, 20.06.1980>> либо щелкнуть по ней правой кнопкой мыши и в появившемся контекстном меню выбрать пункт \kw{Открыть обращение}. & Открывается форма \kw{Стационарное лечение (платные услуги)} для выбранного пациента. \\ \hline
\nn & В левой части формы выбрать раздел \kw{Движение пациента}, щелкнув по нему левой кнопкой мыши. & Осуществляется переход в раздел \kw{Движение пациента} стационарного обращения. \\ \hline
\nn & Нажать кнопку \kw{Движение}. & Открывается форма \kw{Калинина Динара Павловна - Движение}. В поле \dm{Назначено} верхней части формы указаны текущие дата и время, в поле \dm{Состояние} установлено <<Начато>>, в поле \dm{Исполнитель} указан текущий пользователь. В ячейке \dm{Значение} строки \dm{Переведен из отделения} табличной части формы установлено <<Приемное отделение>>, в строке \dm{Отделение пребывания} --  <<Акушерско-гинекологическое отделение>>. \\ \hline
\nn & В ячейках \dm{Значение} табличной части формы указать следующие значения:  \newline \dm{Время поступления}: указать текущее время, \newline \dm{Койка}: выбрать из списка коек отделения любую доступную койку, \newline \dm{Профиль койки}: <<Акушерство и гинекология>>. Остальные поля оставить незаполненными. & Данные о размещении пацента в отделении внесены. \\ \hline
\nn & Нажать кнопку \kw{Сохранить}. & Текущая форма закрывается. Данные о поступлении в отделение сохраняются в БД. Осуществляется возврат на форму \kw{Стационарное лечение (платные услуги)}. \\ \hline
\nn & В левой части формы выбрать раздел \kw{Основная информация}, щелкнув по нему левой кнопкой мыши. & Осуществляется переход в раздел \kw{Основная информация} стационарного обращения. \\ \hline
\nn & В поле \dm{Лечащий врач} выбрать из списка <<Бекбосынова Б.А., Акушер-гинеколог>>. & В поле установлено указанное значение. \\ \hline
\nn & Нажать кнопку \kw{Сохранить}. & Появляется диалоговое окно, содержащее сообщение <<Данные успешно сохранены>>. \\ \hline
\nn & Нажать кнопку \kw{OK}. & Диалоговое окно закрывается. В табличной части формы появилась новая запись о враче. \\ \hline
\end{longtable}

\subsubsection{Регистрация первичного осмотра в отделении} \label{osmotr_st}

\textbf{Необходимые начальные условия:} Должен быть выполнен п. \ref{reg_otd_st} 

\textbf{Роли пользователя:} Врач отделения, Администратор.

\setcounter{nnn}{0}
\begin{longtable}{|p{1cm}|p{7.5cm}|p{8cm}|}
\caption{Регистрация первичного осмотра пациента в отделении\label{osmotr_st_tbl}}\\
\hline \rule{0pt}{15pt}  \centering \textbf{№ пп} & \centering \textbf{Действие} & \hfil \textbf{Ожидаемый результат} \\ \hline
\endfirsthead
\hline \rule{0pt}{15pt} \centering \textbf{№ пп} & \centering \textbf{Действие} & \hfil \textbf{Ожидаемый результат} \\ \hline
\endhead
\nn & В главном меню выбрать пункт \mm{Работа \str Стационарный монитор}. & Открывается форма \kw{Стационарный монитор}. \\ \hline
\nn & Перейти на вкладку \kw{Присутствуют}. & Осуществлен переход на вкладку \kw{Присутствуют}. \\ \hline
\nn & В правом верхнем углу формы в дереве структуры ЛПУ установить курсор на строке <<Акушерско-гинекологическое отделение>>, раскрывая необходимые ветви дерева нажатим кнопки <<$+$>> слева от ее названия. & Курсор установлен на указанном отделении. В правой верхней части формы отображается список пациентов, находящихся в отделении. В списке присутствует <<Калинина Динара Павловна, Ж, 20.06.1980>>.\\ \hline
\nn & Дважды щелкнуть левой кнопкой мыши по записи <<Калинина Динара Павловна, Ж, 20.06.1980>> либо щелкнуть по ней правой кнопкой мыши и в появившемся контекстном меню выбрать пункт \kw{Открыть обращение}. & Открывается форма \kw{Стационарное лечение (платные услуги)} для выбранного пациента.  \\ \hline
\nn & В левой части формы выбрать раздел \kw{Медицинские документы}, щелкнув по нему левой кнопкой мыши. & Осуществляется переход в раздел \kw{Медицинские документы} обращения. Список документов пуст. \\ \hline
\nn & Нажать кнопку \kw{Создать} в правой верхней части формы. & Открывается форма \kw{Создание действий}. \\ \hline
\nn & Перейти на вкладку \kw{Дерево}. & Осуществлен переход на вкладку \kw{Дерево} на форме \kw{Создание действий}. \\ \hline
\nn & Последовательно раскрыть следующие ветви дерева, нажимая знак <<$+$>> слева от названия группы: <<Стационар>> -- <<1\_1: Первичный осмотр>>. & Раскрыт список видов документов при первичном осмотре. \\ \hline 
\nn & Выбрать из списка <<1\_1\_10 Первичный осмотр акушера>> и дважды щелкнуть по нему левой кнопкой мыши либо перетащить запись мышкой в таблицу \kw{Выбранные действия} в правой части формы. & Действие появилось в таблице \kw{Выбранные действия}. \\ \hline
\nn & Нажать кнопку \kw{ОК}. & Текущая форма закрывается. Открывается форма \kw{Калинина Динара Павловна - Первичный осмотр акушера}. В поле \dm{Состояние} указано <<Начато>>, в поле \dm{Начато} -- текущие дата и время, в поле \dm{Исполнитель} -- текущий пользователь. \\ \hline
\nn & Повторить пп. \ref{n1} -- \ref{n2} таблицы \ref{osmotr_ pol_tbl} для каждой строки табличной части текущей формы. & Заполнена таблица в левой части формы. \\ \hline 
\nn & В верхней части формы в поле \dm{Состояние} выбрать <<Закончено>>. & В поле \dm{Выполнено} автоматически появились текущие дта и время. \\ \hline
\nn & Нажать кнопку \kw{OK}. & Текущая форма закрывается. Осуществляется возврат на форму \kw{Стационарное лечение (платные услуги)}. В списке медицинских документов появляется новая запись.\\ \hline
\nn & Нажать кнопку \kw{Сохранить}. & Появляется диалоговое окно, содержащее сообщение <<Данные успешно сохранены>>. \\ \hline
\nn & Нажать кнопку \kw{OK}. & Диалоговое окно закрывается. \\ \hline
\end{longtable}

\subsubsection{Регистрация направлений на исследования} \label{lab_st}

\textbf{Необходимые начальные условия:} Должен быть зарегистрирован первичный осмотр пациента в отделении согласно п. \ref{osmotr_st} Должна быть открыта форма \kw{Стационарное лечение (платные услуги)} для данного обращения.

\textbf{Роли пользователя:} Врач отделения, Администратор.

\setcounter{nnn}{0}
\begin{longtable}{|p{1cm}|p{7.5cm}|p{8cm}|}
\caption{Регистрация направлений на исследования \label{lab_ st_tbl}}\\
\hline \rule{0pt}{15pt}  \centering \textbf{№ пп} & \centering \textbf{Действие} & \hfil \textbf{Ожидаемый результат} \\ \hline
\endfirsthead
\hline \rule{0pt}{15pt} \centering \textbf{№ пп} & \centering \textbf{Действие} & \hfil \textbf{Ожидаемый результат} \\ \hline
\endhead
\nn & В левой части формы выбрать раздел \kw{Диагностические и лабораторные исследования}, щелкнув по нему левой кнопкой мыши. & Осуществляется переход в раздел \kw{Диагностические и лабораторные исследования} обращения. Список медицинских записей пуст. \\ \hline
\nn & Нажать кнопку \kw{Создать} в правой верхней части формы. & Открывается форма \kw{Создание действий}. \\ \hline
\nn & Перейти на вкладку \kw{Дерево}. & Осуществляется переход на вкладку \kw{Дерево} на форме \kw{Создание действий}. \\ \hline
\nn & В левой части формы, нажимая на знак <<$+$>> слева от  названия группы, последовательно раскрыть следующие ветви: <<2\_1: Лабораторная -- 2\_1\_01: Общеклинические анализы>>. & Раскрыто дерево видов общеклинических исследований. \\ \hline 
\nn & Выбрать из списка <<2\_1\_01\_01: Общеклинический анализ крови>> и дважды щелкнуть по нему левой кнопкой мыши либо перетащить запись мышкой в таблицу \kw{Выбранные действия} в правой части формы. & Запись появилась в таблице \kw{Выбранные действия}. \\ \hline
\nn & Выбрать из списка <<2\_1\_01\_03: Общий анализ мочи>> и дважды щелкнуть по нему левой кнопкой мыши либо перетащить запись мышкой в таблицу \kw{Выбранные действия} в правой части формы. & Запись добавлена в таблицу \kw{Выбранные действия}. \\ \hline
\nn & Выбрать из списка <<2\_1\_01\_04: Анализ мочи по Нечипоренко>> и дважды щелкнуть по нему левой кнопкой мыши либо перетащить запись мышкой в таблицу \kw{Выбранные действия} в правой части формы. & Запись добавлена в таблицу \kw{Выбранные действия}. \\ \hline
\nn & Раскрыть группу <<2\_4: Ультразвуковая>>, нажав левой кнопкой мыши на знак <<$+$>> слева от названия. & Раскрыто дерево видов функциональных исследований. \\ \hline 
\nn & Выбрать из списка <<2\_4\_2\_34: Протокол УЗИ в I триместре беременности>> и дважды щелкнуть по нему левой кнопкой мыши либо перетащить запись мышкой в таблицу \kw{Выбранные действия} в правой части формы. & Запись добавлена в таблицу \kw{Выбранные действия}. \\ \hline
\nn & Нажать кнопку \kw{ОК}. & Текущая форма закрывается. Осуществляется возврат на форму \kw{Стационарное лечение (платные услуги)}. В списке медицинских записей появилось 4 записи. \\ \hline
\nn & Нажать кнопку \kw{Сохранить}. & Появляется диалоговое окно, содержащее сообщение <<Данные успешно сохранены>>. \\ \hline
\nn & Нажать кнопку \kw{OK}. & Диалоговое окно закрывается. \\ \hline
\end{longtable}


\subsubsection{Создание листа назначений} \label{ln_st}

\textbf{Необходимые начальные условия:} Должен быть зарегистрирован первичный осмотр пациента в отделении согласно п. \ref{osmotr_st} Должна быть открыта форма \kw{Стационарное лечение (платные услуги)} для данного обращения.

\textbf{Роли пользователя:} Врач отделения, Администратор.

\setcounter{nnn}{0}
\begin{longtable}{|p{1cm}|p{7.5cm}|p{8cm}|}
\caption{Создание листа назначений \label{ln_st_tbl}}\\
\hline \rule{0pt}{15pt}  \centering \textbf{№ пп} & \centering \textbf{Действие} & \hfil \textbf{Ожидаемый результат} \\ \hline
\endfirsthead
\hline \rule{0pt}{15pt} \centering \textbf{№ пп} & \centering \textbf{Действие} & \hfil \textbf{Ожидаемый результат} \\ \hline
\endhead
\nn & В левой части формы выбрать раздел \kw{Лечение}, щелкнув по нему левой кнопкой мыши. & Осуществляется переход в раздел \kw{Лечение} обращения. \\ \hline
\nn & Нажать кнопку \kw{Лист назначений} в правой верхней части формы. & Открывается форма \kw{Лист назначений}. \\ \hline
\nn \label{n88} & Нажать кнопку \kw{Создать назначение} в левом нижнем углу и в появившемся списке выбрать <<Инфузионная терапия>>. & Открывается форма \kw{Назначение}. \\ \hline
\nn & В верхней части формы ввести следующие данные: \newline \dm{Способ введения}: <<внутривенно>>, \newline установить флажок \dm{Срочно}, \newline \dm{Скорость введения}: <<100>>. & Данные внесены в соответствующие поля. \\ \hline
\nn & Нажать кнопку \kw{Добавить препарат}. & В левой нижней части формы появляется поле для поиска препарата. \\ \hline
\nn & В поле поиска ввести строки <<Магне>>.  & Появился список найденных по части наименования препаратов. \\ \hline
\nn & Выбрать в списке <<Магневист>>. & Препарат с выбранным названием появился в таблице ниже. Возможно наличие нескольких одноименных записей в таблице с различной формой или дозировкой. \\ \hline
\nn & Выбрать запись <<Магневист, р-р для внутривенного введения>> и дважды щелкнуть по нему левой кнопкой мыши. & Препарат появился в таблице \kw{Препараты} в левой нижней части формы. \\ \hline
\nn & В ячейке \dm{Доза} таблицы \kw{Препараты} ввести <<100>>, в ячейке \dm{Ед.Изм.} выбрать из списка <<мл>>. & Данные о дозировке внесены в таблицу. \\ \hline
\nn & Нажать кнопку \kw{Добавить время приема}. & В таблице \kw{Исполнения} в правой нижней части формы появляется первая запись. \\ \hline
\nn \label{n9} & В строке таблицы \kw{Исполнения} указать следующие данные: \newline \dm{Начало}: текущие дата и время, \newline \dm{Окончание}: текущие дата и время + 1 час. & Задано время выполнения назначения. \\ \hline
\nn & Нажать кнопку \kw{ОК}. & Текущая форма закрывается. Осуществляется возврат на форму \kw{Лист назначений}. В нижней части формы, в списке препаратов появилась 1 запись. \\ \hline
\nn & Нажать кнопку \kw{Создать назначение} в левом нижнем углу и в появившемся списке выбрать <<Терапия>>. & Открывается форма \kw{Назначение}. \\ \hline
\nn & В верхней части формы в поле \dm{Способ введения} выбрать из списка <<внутривенно>>. & Данные внесены в соответствующие поля. \\ \hline
\nn & Нажать кнопку \kw{Добавить препарат}. & В левой нижней части формы появляется поле для поиска препарата. \\ \hline
\nn & В поле поиска ввести строку <<Глюкоза>>.  & Появился список найденных по части наименования препаратов. \\ \hline
\nn & Выбрать в списке <<Глюкоза>>. & Препарат с выбранным названием появился в таблице ниже. Присутствуют нескольких одноименных записей в таблице с различной формой выпуска или дозировкой. \\ \hline
\nn & Выбрать запись <<Глюкоза, 5\%, р-р для внутривенного введения>> и дважды щелкнуть по нему левой кнопкой мыши. & Препарат появился в таблице \kw{Препараты} в левой нижней части формы. \\ \hline
\nn & В ячейке \dm{Доза} таблицы \kw{Препараты} ввести <<3>>, в ячейке \dm{Ед.Изм.} выбрать из списка <<мл>>. & Данные о дозировке внесены в таблицу. \\ \hline
\nn & Нажать кнопку \kw{Добавить препарат}. & В левой нижней части формы появляется поле для поиска препарата. \\ \hline
\nn & В поле поиска ввести часть строки <<Аскорб>>.  & Появился список найденных по части наименования препаратов. \\ \hline
\nn & Выбрать в списке <<Аскорбиновая кислота>>. & Препарат с выбранным названием появился в таблице ниже. Присутствуют нескольких одноименных записей в таблице с различной формой выпуска или дозировкой. \\ \hline
\nn & Выбрать запись <<Аскорбиновая кислота, 60 мкг, р-р для внутривенного и внутримышечного введения>> и дважды щелкнуть по нему левой кнопкой мыши. & Препарат появился в таблице \kw{Препараты} в левой нижней части формы. \\ \hline
\nn & В ячейке \dm{Доза} таблицы \kw{Препараты} ввести <<1>>, в ячейке \dm{Ед.Изм.} выбрать из списка <<мл>>. & В назначении зарегистрирована смесь из 2-х препаратов. \\ \hline
\nn & Нажать кнопку \kw{Добавить время приема}. & В таблице \kw{Исполнения} в правой нижней части формы появляется первая запись. \\ \hline
\nn & В первой строке таблицы \kw{Исполнения} указать следующие данные: \newline \dm{Начало}: текущие дата и время <<12:00>>, \newline \dm{Окончание}: оставить незаполненным. & Добавлены данные о времени выполнения назначения. \\ \hline
\nn & Нажать кнопку \kw{Добавить время приема}. & В таблице \kw{Исполнения} в правой нижней части формы появляется первая запись. \\ \hline
\nn & В новой строке таблицы \kw{Исполнения} указать следующие данные: \newline \dm{Начало}: текущие дата и время <<20:00>>, \newline \dm{Окончание}: оставить незаполненным. & Добавлен новый период времени для выполнения назначения. \\ \hline
\nn & Нажать кнопку \kw{ОК}. & Текущая форма закрывается. Осуществляется возврат на форму \kw{Лист назначений}. В нижней части формы, в списке препаратов появилась новая запись. Интервалы приема препаратов имеют полосатую окраску. \\ \hline
\nn & Нажать кнопку \kw{Сохранить}. & Данные листа назначений сохраняются. Интервалы приема препаратов окрашиваются в один цвет, соответствующий состоянию выполнения назначений. \\ \hline
\nn & Нажать кнопку \kw{Отмена}. & Текущая форма закрывается. Осуществляется возврат на форму \kw{Стационарное лечение (платные услуги)}. \\ \hline
\end{longtable}

\subsubsection{Корректировка листа назначений} \label{ln_edt_st}

\textbf{Необходимые начальные условия:} Должен быть создан лист назначений пацинета согласно п. \ref{ln_st} Должна быть открыта форма \kw{Стационарное лечение (платные услуги)} для данного обращения.

\textbf{Роли пользователя:} Врач отделения, Администратор.

\setcounter{nnn}{0}
\begin{longtable}{|p{1cm}|p{7.5cm}|p{8cm}|}
\caption{Редактирование листа назначений \label{ln_edt_st_tbl}}\\
\hline \rule{0pt}{15pt}  \centering \textbf{№ пп} & \centering \textbf{Действие} & \hfil \textbf{Ожидаемый результат} \\ \hline
\endfirsthead
\hline \rule{0pt}{15pt} \centering \textbf{№ пп} & \centering \textbf{Действие} & \hfil \textbf{Ожидаемый результат} \\ \hline
\endhead
\nn & В левой части формы выбрать раздел \kw{Лечение}, щелкнув по нему левой кнопкой мыши. & Осуществляется переход в раздел \kw{Лечение} обращения. \\ \hline
\nn & Нажать кнопку \kw{Лист назначений} в правой верхней части формы. & Открывается форма \kw{Лист назначений}. \\ \hline
\nn & Дважды щелкнуть левой кнопкой мыши по записи <<Глюкоза (5\%) 3.0 мл. Аскорбиновая кислота (60 мкг) 1.0 мл>>. & Открывается форма \kw{Назначение}. \\ \hline
\nn & В таблице \kw{Исполнения} в правой нижней части формы выбрать последнюю запись и щелкнуть левой кнопкой мыши по значку в ячейке \dm{Отменить} выбранной строки. & Появляется диалоговое окно с сообщением <<Вы действительно хотите отменить выполнение?>>. \\ \hline
\nn & Нажать кнопку \kw{Да}. &  Исчезла отметка в ячейке \dm{Исполнить} для выбранной строки. \\ \hline
\nn & Нажать кнопку \kw{Добавить время приема}. & В таблице \kw{Исполнения} в правой нижней части формы появляется новая запись. \\ \hline
\nn & В появившейся строке таблицы \kw{Исполнения} указать следующие данные: \newline \dm{Начало}: текущие дата и время <<22:00>>, \newline \dm{Окончание}: оставить незаполненным. & Данные внесены в соответствующие поля. \\ \hline
\nn & Нажать кнопку \kw{ОК}. & Текущая форма закрывается. Осуществляется возврат на форму \kw{Лист назначений}. В нижней части формы отредактированные и новые интервалы приема препарата имеют полосатую окраску. \\ \hline
\nn & Добавить назначение препарата <<Дисоль, 100 мл., раствор для инфузий>> выполнив шаги, аналогичные пп. \ref{n88}--\ref{n9} таблицы \ref{ln_st_tbl}. & На форме \kw{Лист назначений} в нижней части формы добавлена новая запись.\\ \hline
\nn & Нажать кнопку \kw{Сохранить}. & Данные листа назначений сохраняются. Интервалы приема препаратов окрашиваются в один цвет, соответствующий состоянию выполнения назначений. Интервал <<20:00>> препарата <<Глюкоза (5\%) 3.0 мл. Аскорбиновая кислота (60 мкг) 1.0 мл>> окрашен в серый цвет. \\ \hline
\nn & Нажать кнопку \kw{Отмена}. & Текущая форма закрывается. Осуществляется возврат на форму \kw{Стационарное лечение (платные услуги)}. \\ \hline
\end{longtable}

\subsubsection{Исполнение медикаментозных назначений} \label{ln_acc_st}

\textbf{Необходимые начальные условия:} Должен быть скорректирован лист назначений пацинета согласно п. \ref{ln_edt_st} 

\textbf{Роли пользователя:} Медсестра отделения, Администратор.

\setcounter{nnn}{0}
\begin{longtable}{|p{1cm}|p{7.5cm}|p{8cm}|}
\caption{Исполнение назначений \label{ln_acc_st_tbl}}\\
\hline \rule{0pt}{15pt}  \centering \textbf{№ пп} & \centering \textbf{Действие} & \hfil \textbf{Ожидаемый результат} \\ \hline
\endfirsthead
\hline \rule{0pt}{15pt} \centering \textbf{№ пп} & \centering \textbf{Действие} & \hfil \textbf{Ожидаемый результат} \\ \hline
\endhead
\nn & В главном меню выбрать пункт \mm{Работа \str Лист назначений (исполнения)}. & Открывается форма \kw{Лист назначений (исполнения)}. \\ \hline
\nn & В левом верхнем углу в поле \dm{Дата} установить дату создания листа назначений согласно п. \ref{ln_st} & Выбрана дата просмотра листа назначений. \\ \hline
\nn & В поле \dm{Отделение} в верхней части формы выбрать из дерева структуры ЛПУ <<Акушерско-гинекологическое отделение>>. & Выбрано отделение для просмотра назначений. \\ \hline
\nn & В раскрывающемся списке \dm{Пациенты} установить флажок для записи <<Калинина Динара Павловна>>. & В нижней части формы отображается список назначений выбранного пациента. \\ \hline
\nn & Установить флажки в ячейках \dm{Исполнить} первых двух строк таблицы в нижней части формы. & Для выбранных строк все интервалы приема, для которых может быть установлена отметка об исполнении на текущий момент, приняли полосатую окраску. \\ \hline
\nn & Нажать кнопку \kw{Сохранить}. & Данные об исполнении сохраняются. Исполненные интервалы окрашиваются в зеленый цвет. \\ \hline
\nn & Нажать кнопку \kw{Отмена}. & Текущая форма закрывается. \\ \hline
\end{longtable}

\subsubsection{Назначение физиотерапевтического лечения} \label{fiz_st}

\textbf{Необходимые начальные условия:} Должен быть зарегистрирован первичный осмотр пациента в отделении согласно п. \ref{osmotr_st} Должна быть открыта форма \kw{Стационарное лечение (платные услуги)} для данного обращения.

\textbf{Роли пользователя:} Врач отделения, Администратор.

\setcounter{nnn}{0}
\begin{longtable}{|p{1cm}|p{7.5cm}|p{8cm}|}
\caption{Назначение физиотерапевтического лечения \label{fiz_ st_tbl}}\\
\hline \rule{0pt}{15pt}  \centering \textbf{№ пп} & \centering \textbf{Действие} & \hfil \textbf{Ожидаемый результат} \\ \hline
\endfirsthead
\hline \rule{0pt}{15pt} \centering \textbf{№ пп} & \centering \textbf{Действие} & \hfil \textbf{Ожидаемый результат} \\ \hline
\endhead
\nn & В левой части формы выбрать раздел \kw{Лечение}, щелкнув по нему левой кнопкой мыши. & Осуществляется переход в раздел \kw{Лечение} обращения. \\ \hline
\nn & Нажать кнопку \kw{Создать} в правой верхней части формы. & Открывается форма \kw{Создание действий}. \\ \hline
\nn & Перейти на вкладку \kw{Дерево}. & Осуществлен переход на вкладку \kw{Дерево} на форме \kw{Создание действий}. \\ \hline
\nn & Раскрыть ветвь дерева <<3\_3: Физиотерапевтическое>>, щелкнув левой кнопкой мыши по знаку <<$+$>> слева от названия. & Раскрывается список видов физиотерапевтических процедур. \\ \hline 
\nn & Выбрать из списка <<3\_33: Массаж лечебный нижних конечностей>> и дважды щелкнуть по нему левой кнопкой мыши либо перетащить запись мышкой в таблицу \kw{Выбранные действия} в правой части формы. & Запись появилась в таблице \kw{Выбранные действия}. \\ \hline
\nn & Нажать кнопку \kw{ОК}. & Текущая форма закрывается. Открывается форма \kw{Калинина Динара Павловна - Массаж лечебный нижних конечностей}. \\ \hline
\nn & В поле \dm{Количество} в верхней части формы ввести значение <<10>>. & Указано количество процедур. \\ \hline
\nn & Нажать кнопку \kw{Сохранить}. & Текущая форма закрывается. Данные о назначении сохраняются в БД. Осуществляется возврат на форму \kw{Стационарное лечение (платные услуги)}. В списке лечебных назначений появилась новая запись <<ЛФК>>.\\ \hline
\nn & Нажать кнопку \kw{Сохранить}. & Появляется диалоговое окно, содержащее сообщение <<Данные успешно сохранены>>. \\ \hline
\nn & Нажать кнопку \kw{OK}. & Диалоговое окно закрывается. \\ \hline
\end{longtable}

\subsubsection{Регистрация предварительных диагнозов пациента} \label{dz_st}

\textbf{Необходимые начальные условия:} Должен быть зарегистрирован первичный осмотр пациента в отделении согласно п. \ref{osmotr_st} Должна быть открыта форма \kw{Стационарное лечение (платные услуги)} для данного обращения.

\textbf{Роли пользователя:} Врач отделения, Администратор.

\setcounter{nnn}{0}
\begin{longtable}{|p{1cm}|p{7.5cm}|p{8cm}|}
\caption{Регистрация предварительных диагнозов пациента \label{dz_ st_tbl}}\\
\hline \rule{0pt}{15pt}  \centering \textbf{№ пп} & \centering \textbf{Действие} & \hfil \textbf{Ожидаемый результат} \\ \hline
\endfirsthead
\hline \rule{0pt}{15pt} \centering \textbf{№ пп} & \centering \textbf{Действие} & \hfil \textbf{Ожидаемый результат} \\ \hline
\endhead
\nn & В левой части формы выбрать раздел \kw{Основная информация}, щелкнув по нему левой кнопкой мыши. & Осуществляется переход в раздел \kw{Основная информация} обращения. \\ \hline
\nn & Нажать кнопку \kw{Предварительные диагнозы}. & Открывается форма \kw{Таблица диаг\-нозов}. \\ \hline
\nn & В первой строке таблицы дважды щелкнуть левой кнопкой мыши в ячейке \dm{Тип} и выбрать из списка <<предварительный основной>>, в ячейке \dm{МКБ} ввести <<О13.0>>, в ячейке \dm{Хар} дважды щелкнуть левой кнопкой мыши и выбрать значение <<острое>>. & Данные внесены в соответствующие поля. В поле \dm{Врач} автоматически подставлена фамилия текущего лечащего врача <<Бекбосынова Б.А.>>. \\ \hline
\nn & Нажать кнопку \kw{Сохранить}. & Текущая форма закрывается. Осуществляется возврат на форму \kw{Стационарное лечение (платные услуги)}. \\ \hline
\nn & Нажать кнопку \kw{Сохранить}. & Появляется диалоговое окно, содержащее сообщение <<Данные успешно сохранены>>. \\ \hline
\nn & Нажать кнопку \kw{OK}. & Диалоговое окно закрывается. \\ \hline
\end{longtable}

\subsubsection{Открытие листа временной нетрудоспособности} \label{vutopen_st}

\textbf{Необходимые начальные условия:} Должен быть зарегистрирован первичный осмотр пациента в отделении согласно п. \ref{osmotr_st} Должна быть открыта форма \kw{Стационарное лечение (платные услуги)} для данного обращения.

\textbf{Роли пользователя:} Врач отделения, Администратор.

\setcounter{nnn}{0}
\begin{longtable}{|p{1cm}|p{7.5cm}|p{8cm}|}
\caption{Открытие листа временной нетрудоспособности \label{vutopen_ st_tbl}}\\
\hline \rule{0pt}{15pt}  \centering \textbf{№ пп} & \centering \textbf{Действие} & \hfil \textbf{Ожидаемый результат} \\ \hline
\endfirsthead
\hline \rule{0pt}{15pt} \centering \textbf{№ пп} & \centering \textbf{Действие} & \hfil \textbf{Ожидаемый результат} \\ \hline
\endhead
\nn & В левой части формы выбрать раздел \kw{ВУТ}, щелкнув по нему левой кнопкой мыши. & Осуществляется переход в раздел \kw{ВУТ} обращения. \\ \hline
\nn & На вкладке \kw{Листок нетрудоспособности} установить флажок \dm{ВУТ}. & Становятся доступными поля для регистрации листа временной нетрудоспособности. \\ \hline
\nn & В поле \dm{Причина} выбрать из списка <<заболевание>>, в поля \dm{Серия и номер} ввести значения <<B4>> и <<887456>> соответственно, в поле \dm{Место работы} ввести <<Хлебозавод №1>>. & Данные внесены в соответствующие поля. \\ \hline
\nn & В первой строке табличной части формы дважды щелкнуть левой кнопкой мыши в ячейке \dm{Начало}, в ячейке \dm{Окончание} выбрать дату на 7 дней больше текущей, в ячейке \dm{Результат} двойным щелчком мыши активировать список и выбрать значение <<нетрудоспособен>>, в ячейке \dm{Режим} двойным щелчком левой кнопки мыши активировать список и выбрать значение <<домашний>>. & Данные внесены в соответствующие поля. В ячейке \dm{Начало} установлена текущая дата, в ячейке \dm{Длительность} и поле  \dm{Длительность} в верхней части формы автоматически рассчитано значение <<7>>.\\ \hline
\nn & Нажать кнопку \kw{Сохранить}. & Появляется диалоговое окно, содержащее сообщение <<Данные успешно сохранены>>. \\ \hline
\nn & Нажать кнопку \kw{OK}. & Диалоговое окно закрывается. \\ \hline
\end{longtable}

\subsubsection{Регистрация ежедневного осмотра пациента лечащим врачом} \label{osmotr2_st}

\textbf{Необходимые начальные условия:} Должен быть зарегистрирован первичный осмотр пациента в отделении согласно п. \ref{osmotr_st} Должна быть открыта форма \kw{Стационарное лечение (платные услуги)} для данного обращения.

\textbf{Роли пользователя:} Врач отделения, Администратор.

\setcounter{nnn}{0}
\begin{longtable}{|p{1cm}|p{7.5cm}|p{8cm}|}
\caption{Регистрация ежедневного осмотра пациента \label{osmotr2_ st_tbl}}\\
\hline \rule{0pt}{15pt}  \centering \textbf{№ пп} & \centering \textbf{Действие} & \hfil \textbf{Ожидаемый результат} \\ \hline
\endfirsthead
\hline \rule{0pt}{15pt} \centering \textbf{№ пп} & \centering \textbf{Действие} & \hfil \textbf{Ожидаемый результат} \\ \hline
\endhead
\nn & В левой части формы выбрать раздел \kw{Медицинские документы}, щелкнув по нему левой кнопкой мыши. & Осуществляется переход в раздел \kw{Медицинские документы} обращения. \\ \hline
\nn & Нажать кнопку \kw{Создать} в правой верхней части формы. & Открывается форма \kw{Создание действий}. \\ \hline
\nn & Перейти на вкладку \kw{Дерево}. & Осуществлен переход на вкладку \kw{Дерево} на форме \kw{Создание действий}. \\ \hline
\nn & Последовательно раскрыть следующие ветви дерева, нажимая знак <<$+$>> слева от названия группы: <<Стационар>> -- <<1\_2: Дневниковый осмотр>>. & Раскрыт список видов дневниковых осмотров. \\ \hline 
\nn & Выбрать из списка <<1\_2\_01 Дневниковый осмотр акушера>> и дважды щелкнуть по нему левой кнопкой мыши либо перетащить запись мышкой в таблицу \kw{Выбранные действия} в правой части формы. & Действие появилось в таблице \kw{Выбранные действия}. \\ \hline
\nn & Нажать кнопку \kw{ОК}. & Текущая форма закрывается. Открывается форма \kw{Калинина Динара Павловна - Дневниковый осмотр акушера}. \\ \hline
\nn & Повторить пп. \ref{n1} -- \ref{n2} таблицы \ref{osmotr_ pol_tbl} для каждой строки табличной части текущей формы. & Заполнена таблица в левой части формы. \\ \hline 
\nn & В поле \dm{Состояние} выбрать из списка <<Закончено>>. & В поле \dm{Выполнено} автоматически установлены текущие дата и время. \\ \hline
\nn & Нажать кнопку \kw{Сохранить}. & Текущая форма закрывается. Осуществляется возврат на форму\kw{Стационарное лечение (платные услуги)}. В списке медицинских документов появляется новая запись.\\ \hline
\nn & Нажать кнопку \kw{Сохранить}. & Появляется диалоговое окно, содержащее сообщение <<Данные успешно сохранены>>. \\ \hline
\nn & Нажать кнопку \kw{OK}. & Диалоговое окно закрывается. \\ \hline
\end{longtable}

\subsubsection{Планирование операции} \label{operplan_st}

\textbf{Необходимые начальные условия:} Должен быть зарегистрирован первичный осмотр пациента в отделении согласно п. \ref{osmotr_st} Должна быть открыта форма \kw{Стационарное лечение (платные услуги)} для данного обращения.

\textbf{Роли пользователя:} Врач отделения, Администратор.

\setcounter{nnn}{0}
\begin{longtable}{|p{1cm}|p{7.5cm}|p{8cm}|}
\caption{Планирование операции \label{operplan_st_tbl}}\\
\hline \rule{0pt}{15pt}  \centering \textbf{№ пп} & \centering \textbf{Действие} & \hfil \textbf{Ожидаемый результат} \\ \hline
\endfirsthead
\hline \rule{0pt}{15pt} \centering \textbf{№ пп} & \centering \textbf{Действие} & \hfil \textbf{Ожидаемый результат} \\ \hline
\endhead
\nn & В левой части формы выбрать раздел \kw{Лечение}, щелкнув по нему левой кнопкой мыши. & Осуществляется переход в раздел \kw{Лечение} обращения. \\ \hline
\nn & Нажать кнопку \kw{Создать} в правой верхней части формы. & Открывается форма \kw{Создание действий}. \\ \hline
\nn & Перейти на вкладку \kw{Дерево}. & Осуществлен переход на вкладку \kw{Дерево} на форме \kw{Создание действий}. \\ \hline
\nn & Раскрыть группу <<3\_2: Оперативное>> нажав знак <<$+$>> слева от названия группы. & Раскрыт список видов операций. \\ \hline 
\nn & Выбрать из списка <<3\_2\_01\_20 Операция (акушерство)>> и дважды щелкнуть по ней левой кнопкой мыши либо перетащить запись мышкой в таблицу \kw{Выбранные действия} в правой части формы. & Действие появилось в таблице \kw{Выбранные действия}. \\ \hline
\nn & Нажать кнопку \kw{ОК}. & Текущая форма закрывается. Открывается форма \kw{Калинина Динара Павловна - Операция (акушерство)}. \\ \hline
\nn & В верхней части формы в поле \dm{План} ввести следующую дату время <<09:00>>. & Указана планируемая дата операции. \\ \hline
\nn & Нажать кнопку \kw{Сохранить}. & Текущая форма закрывается. Осуществляется возврат на форму\kw{Стационарное лечение (платные услуги)}. В списке появилась новая запись.\\ \hline
\nn & Нажать кнопку \kw{Сохранить}. & Появляется диалоговое окно, содержащее сообщение <<Данные успешно сохранены>>. \\ \hline
\nn & Нажать кнопку \kw{OK}. & Диалоговое окно закрывается. \\ \hline
\end{longtable}

\subsubsection{Формирование протокола операции} \label{oper_st}

\textbf{Необходимые начальные условия:} Должна быть запланирована операция согласно п. \ref{operplan_st} Должна быть открыта форма \kw{Стационарное лечение (платные услуги)} для данного обращения.

\textbf{Роли пользователя:} Врач отделения, Администратор.

\setcounter{nnn}{0}
\begin{longtable}{|p{1cm}|p{7.5cm}|p{8cm}|}
\caption{Формирование протокола операции \label{oper_st_tbl}}\\
\hline \rule{0pt}{15pt}  \centering \textbf{№ пп} & \centering \textbf{Действие} & \hfil \textbf{Ожидаемый результат} \\ \hline
\endfirsthead
\hline \rule{0pt}{15pt} \centering \textbf{№ пп} & \centering \textbf{Действие} & \hfil \textbf{Ожидаемый результат} \\ \hline
\endhead
\nn & В левой части формы выбрать раздел \kw{Лечение}, щелкнув по нему левой кнопкой мыши. & Осуществляется переход в раздел \kw{Лечение} обращения. \\ \hline
\nn & Дважды щелкнуть левой кнопкой мыши по записи <<Операция (акушерство)>>. & Открывается форма \kw{Калинина Динара Павловна - Операция (акушерство)}. \\ \hline
\nn & Ввести данные в ячейки \dm{Значение} каждой строки таблицы, расположенной в левой части формы. & Заполнена таблица в левой части формы. \\ \hline
\nn & В поле \dm{Состояние} в верхней части формы выбрать значение <<Закончено>>. & Установлено состояние <<Закончено>> для операции.\\ \hline 
\nn & Нажать кнопку \kw{Сохранить}. & Текущая форма закрывается. Данные протокола операции сохраняются в БД. Осуществляется возврат на форму  \kw{Стационарное лечение (платные услуги)}. В списке появилась новая запись.\\ \hline
\nn & Нажать кнопку \kw{Сохранить}. & Появляется диалоговое окно, содержащее сообщение <<Данные успешно сохранены>>. \\ \hline
\nn & Нажать кнопку \kw{OK}. & Диалоговое окно закрывается. \\ \hline
\end{longtable}

\subsubsection{Регистрация осмотра пациента дежурным врачом} \label{osmotrd_st}

\textbf{Необходимые начальные условия:} Должен быть зарегистрирован первичный осмотр пациента в отделении согласно п. \ref{osmotr_st} Должна быть открыта форма \kw{Стационарное лечение (платные услуги)} для данного обращения.

\textbf{Роли пользователя:} Дежурный врач, Администратор.

\setcounter{nnn}{0}
\begin{longtable}{|p{1cm}|p{7.5cm}|p{8cm}|}
\caption{Регистрация осмотра пациента дежурным врачом \label{osmotrd_ st_tbl}}\\
\hline \rule{0pt}{15pt}  \centering \textbf{№ пп} & \centering \textbf{Действие} & \hfil \textbf{Ожидаемый результат} \\ \hline
\endfirsthead
\hline \rule{0pt}{15pt} \centering \textbf{№ пп} & \centering \textbf{Действие} & \hfil \textbf{Ожидаемый результат} \\ \hline
\endhead
\nn & В левой части формы выбрать раздел \kw{Медицинские документы}, щелкнув по нему левой кнопкой мыши. & Осуществляется переход в раздел \kw{Медицинские документы} обращения. \\ \hline
\nn & Нажать кнопку \kw{Дежурный врач} в правой верхней части формы. & Открывается форма \kw{Калинина Динара Павловна - Осмотр дежурного врача}. \\ \hline
\nn & Ввести с клавиатуры данные в ячейки \dm{Значение} для каждой строки таблицы, расположенной в левой части формы. & Заполнена таблица в левой части формы. \\ \hline 
\nn & Нажать кнопку \kw{Сохранить}. & Текущая форма закрывается. Осуществляется возврат на форму \kw{Стационарное лечение (платные услуги)}. В списке медицинских документов появляется новая запись.\\ \hline
\nn & Нажать кнопку \kw{Сохранить}. & Появляется диалоговое окно, содержащее сообщение <<Данные успешно сохранены>>. \\ \hline
\nn & Нажать кнопку \kw{OK}. & Диалоговое окно закрывается. \\ \hline
\end{longtable}

\subsubsection{Ввод результатов лаборторных исследований} \label{labrez_st}

\textbf{Необходимые начальные условия:} Должна быть выполнена рагистрация направлений на лабораторные ииследования согласно  п. \ref{lab_st} Должна быть открыта форма \kw{Стационарное лечение (платные услуги)} для данного обращения.

\textbf{Роли пользователя:} Лаборант, Врач отделения, Администратор.

\setcounter{nnn}{0}
\begin{longtable}{|p{1cm}|p{7.5cm}|p{8cm}|}
\caption{Ввод результатов лабораторных исследований \label{labrez_ st_tbl}}\\
\hline \rule{0pt}{15pt}  \centering \textbf{№ пп} & \centering \textbf{Действие} & \hfil \textbf{Ожидаемый результат} \\ \hline
\endfirsthead
\hline \rule{0pt}{15pt} \centering \textbf{№ пп} & \centering \textbf{Действие} & \hfil \textbf{Ожидаемый результат} \\ \hline
\endhead
\nn & В левой части формы выбрать раздел \kw{Диагностические и лабораторные исследования}, щелкнув по нему левой кнопкой мыши. & Осуществляется переход в раздел \kw{Диагностические и лабораторные исследования} обращения. \\ \hline
\nn \label{n10} & Дважды щелкнуть левой кнопкой мыши по записи <<Общий анализ крови>>. & Открывается форма \kw{Калинина Динара Павловна - Общий анализ крови}. \\ \hline
\nn & В ячейки \dm{Значение} всех строк таблицы в левой части формы ввести с клавиатуры соответствующие числовые значения показателей.  & Введены результаты лабораторного исследования. \\ \hline
\nn & В поле \dm{Состояние} в верхней части формы выбрать значение <<Закончено>>, в поле \dm{Выполнено} установить текущую дату и время выполнения <<9:30>>, в поле \dm{Исполнитель} выбрать из списка <<Аскарова К.К., Лаборант>>. & Данные внесены в соответствующие поля. \\ \hline
\nn \label{n11}& Нажать кнопку \kw{Сохранить}. & Теущая форма закрывается. Данные о выполнени лабораторного исследования сохраняются в БД. Осуществляется возврат на форму \kw{Стационарное лечение (платные услуги)}.\\ \hline
\nn & Выполнить пп. \ref{n10}--\ref{n11} для исследования <<Общий анализ мочи>>. & Заполнены результаты анализа <<Общий анализ мочи>>.\\ \hline
\nn & Выполнить пп. \ref{n10}--\ref{n11} для исследования <<Анализ мочи по Нечипоренко>>. & Заполнены результаты анализа <<Анализ мочи по Нечипоренко>>. \\ \hline
\nn & Нажать кнопку \kw{Сохранить}. & Появляется диалоговое окно, содержащее сообщение <<Данные успешно сохранены>>. \\ \hline
\nn & Нажать кнопку \kw{OK}. & Диалоговое окно закрывается. \\ \hline
\end{longtable}

\subsubsection{Ввод результатов инструментальной диагностики} \label{iirez_st}

\textbf{Необходимые начальные условия:} Должна быть выполнена рагистрация направления на инструментальные иcследования согласно  п. \ref{lab_st} Должна быть открыта форма \kw{Стационарное лечение (платные услуги)} для данного обращения.

\textbf{Роли пользователя:} Врач функциональной диагностики, Врач отделения, Администратор.

\setcounter{nnn}{0}
\begin{longtable}{|p{1cm}|p{7.5cm}|p{8cm}|}
\caption{Ввод результатов инструментальной диагностики \label{iirez_ st_tbl}}\\
\hline \rule{0pt}{15pt}  \centering \textbf{№ пп} & \centering \textbf{Действие} & \hfil \textbf{Ожидаемый результат} \\ \hline
\endfirsthead
\hline \rule{0pt}{15pt} \centering \textbf{№ пп} & \centering \textbf{Действие} & \hfil \textbf{Ожидаемый результат} \\ \hline
\endhead
\nn & В левой части формы выбрать раздел \kw{Диагностические и лабораторные исследования}, щелкнув по нему левой кнопкой мыши. & Осуществляется переход в раздел \kw{Диагностические и лабораторные исследования} обращения. \\ \hline
\nn & Дважды щелкнуть левой кнопкой мыши по записи <<Протокол УЗИ в I триместре беременности>>. & Открывается форма \kw{Калинина Динара Павловна - Протокол УЗИ в I триместре беременности}. \\ \hline
\nn & Заполнить ячейки \dm{Значение} всех строк таблицы в левой части формы способом, аналогичным описанном в пп. \ref{n1}--\ref{n2} таблицы \ref{osmotr_pol_tbl}.  & Введены результаты УЗИ. \\ \hline
\nn & В поле \dm{Состояние} в верхней части формы выбрать значение <<Закончено>>, в поле \dm{Выполнено} установить текущую дату, в поле \dm{Исполнитель} выбрать из списка <<Бибаев В.В., Врач функциональной диагностики>>. & Данные внесены в соответствующие поля. \\ \hline
\nn & Нажать кнопку \kw{Сохранить}. & Теущая форма закрывается. Данные о выполнени инструментального исследования сохраняются в БД. Осуществляется возврат на форму \kw{Стационарное лечение (платные услуги)}.\\ \hline
\nn & Нажать кнопку \kw{Сохранить}. & Появляется диалоговое окно, содержащее сообщение <<Данные успешно сохранены>>. \\ \hline
\nn & Нажать кнопку \kw{OK}. & Диалоговое окно закрывается. \\ \hline
\end{longtable}

\subsubsection{Регистрация перевода пациента в другое отделение (передающее отделеление)} \label{reg_per_st}

\textbf{Необходимые начальные условия:} Должен быть выполнен п. \ref{reg_otd_st} 

\textbf{Роли пользователя:} Медсестра отделения, Администратор.

\setcounter{nnn}{0}
\begin{longtable}{|p{1cm}|p{7.5cm}|p{8cm}|}
\caption{Регистрация перевода пациента в другое отделение (1) \label{reg_per_st_tbl}}\\
\hline \rule{0pt}{15pt}  \centering \textbf{№ пп} & \centering \textbf{Действие} & \hfil \textbf{Ожидаемый результат} \\ \hline
\endfirsthead
\hline \rule{0pt}{15pt} \centering \textbf{№ пп} & \centering \textbf{Действие} & \hfil \textbf{Ожидаемый результат} \\ \hline
\endhead
\nn & В главном меню выбрать пункт \mm{Работа \str Стационарный монитор}. & Открывается форма \kw{Стационарный монитор}. \\ \hline
\nn & Перейти на вкладку \kw{Присутствуют}. & Осуществлен переход на вкладку \kw{Присутствуют}. \\ \hline
\nn & В правом верхнем углу формы в дереве структуры ЛПУ установить курсор на строке <<Акушерско-гинекологическое отделение>>, раскрывая необходимые ветви дерева нажатим кнопки <<$+$>> слева от ее названия. & Курсор установлен на указанном отделении. В списке в правой верхней части формы отображаются записи о пациентах, находящихся в данном отделении. В списке присутствует запись <<Калинина Динара Павловна, Ж, 20.06.1980>>. \\ \hline
\nn & Дважды щелкнуть левой кнопкой мыши по записи <<Калинина Динара Павловна, Ж, 20.06.1980>> либо щелкнуть по ней правой кнопкой мыши и в появившемся контекстном меню выбрать пункт \kw{Открыть обращение}. & Открывается форма \kw{Стационарное лечение (платные услуги)} для выбранного пациента. \\ \hline
\nn & В левой части формы выбрать раздел \kw{Движение пациента}, щелкнув по нему левой кнопкой мыши. & Осуществляется переход в раздел \kw{Движение пациента} стационарного обращения. \\ \hline
\nn & Нажать кнопку \kw{Движение}. & Открывается форма \kw{Калинина Динара Павловна - Движение}. \\ \hline
\nn & В ячейах \dm{Значение} в строке <<Время выбытия>> табличной части формы указать текущее время, в строке <<Переведен в отделение>> из дерева структуры ЛПУ выбрать значение <<Хирургическое отделение>>. В остальных незаполненных строках таблицы выбрать или ввести произвольные значения.  & Указаны данные перевода пациента. \\ \hline
\nn & В поле \dm{Состояние} в верхней части формы выбрать значение <<Закончено>>. & В поле \dm{Выполнено} автоматически подставлена текущая дата. \\ \hline
\nn & Нажать кнопку \kw{Сохранить}. & Текущая форма закрывается. Данные о переводе в другое отделение сохраняются в БД. Осуществляется возврат на форму \kw{Стационарное лечение (платные услуги)}. \\ \hline
\nn & Нажать кнопку \kw{Сохранить}. & Появляется диалоговое окно, содержащее сообщение <<Данные успешно сохранены>>. \\ \hline
\nn & Нажать кнопку \kw{OK}. & Диалоговое окно закрывается. В табличной части формы появилась новая запись о враче. \\ \hline
\end{longtable}

\subsubsection{Регистрация перевода пациента в другое отделение (принимающее отделеление)} \label{reg_otd2_st}

\textbf{Необходимые начальные условия:} Должен быть выполнены п. \ref{reg_per_st}

\textbf{Роли пользователя:} Медсестра отделения, Администратор.

\setcounter{nnn}{0}
\begin{longtable}{|p{1cm}|p{7.5cm}|p{8cm}|}
\caption{Регистрация перевода пациента в другое отделение (2) \label{reg_otd2_st_tbl}}\\
\hline \rule{0pt}{15pt}  \centering \textbf{№ пп} & \centering \textbf{Действие} & \hfil \textbf{Ожидаемый результат} \\ \hline
\endfirsthead
\hline \rule{0pt}{15pt} \centering \textbf{№ пп} & \centering \textbf{Действие} & \hfil \textbf{Ожидаемый результат} \\ \hline
\endhead
\nn & В главном меню выбрать пункт \mm{Работа \str Стационарный монитор}. & Открывается форма \kw{Стационарный монитор}. \\ \hline
\nn & Перейти на вкладку \kw{Переведены (в отделение)}. & Осуществлен переход на вкладку \kw{Переведены (в отделение)}. \\ \hline
\nn & В правом верхнем углу формы в дереве структуры ЛПУ установить курсор на строке <<Хирургическое отделение>>, раскрывая необходимые ветви дерева нажатим кнопки <<$+$>> слева от ее названия. & Курсор установлен на указанном отделении. В списке в правой части формы присутствует запись <<Калинина Динара Павловна, Ж, 20.06.1980>>. \\ \hline
\nn & Дважды щелкнуть левой кнопкой мыши по записи <<Калинина Динара Павловна, Ж, 20.06.1980>> либо щелкнуть по ней правой кнопкой мыши и в появившемся контекстном меню выбрать пункт \kw{Открыть обращение}. & Открывается форма \kw{Стационарное лечение (платные услуги)} для выбранного пациента. \\ \hline
\nn & В левой части формы выбрать раздел \kw{Движение пациента}, щелкнув по нему левой кнопкой мыши. & Осуществляется переход в раздел \kw{Движение пациента} стационарного обращения. \\ \hline
\nn & Нажать кнопку \kw{Движение}. & Открывается форма \kw{Калинина Динара Павловна - Движение}. В поле \dm{Назначено} верхней части формы указаны текущие дата и время, поле \dm{Состояние} установлено <<Начато>>, в поле \dm{Исполнитель} указан текущий пользователь. В ячейке \dm{Значение} строки \dm{Переведен из отделения} табличной части формы установлено <<Акушерско-гинекологическое отделение>>, в строке \dm{Отделение пребывания} --  <<Хирургическое отделение>>. \\ \hline
\nn & В ячейках \dm{Значение} табличной части формы указать следующие значения:  \newline \dm{Время поступления}: указать текущее время, \newline \dm{Койка}: выбрать из списка коек хирургического отделения любую доступную койку, \newline \dm{Профиль койки}: <<Хирургический взрослые>>. Остальные поля оставить незаполненными. & Данные о размещении пацента в отделении внесены. \\ \hline
\nn & Нажать кнопку \kw{Сохранить}. & Текущая форма закрывается. Данные о размещении в новом  отделение сохраняются в БД. Осуществляется возврат на форму \kw{Стационарное лечение (платные услуги)}. \\ \hline
\nn & В левой части формы выбрать раздел \kw{Основная информация}, щелкнув по нему левой кнопкой мыши. & Осуществляется переход в раздел \kw{Основная информация} стационарного обращения. \\ \hline
\nn & В поле \dm{Лечащий врач} выбрать из списка <<Абдрахманов К.К., Хирург>>. & В поле установлено указанное значение. \\ \hline
\nn & Нажать кнопку \kw{Сохранить}. & Появляется диалоговое окно, содержащее сообщение <<Данные успешно сохранены>>. \\ \hline
\nn & Нажать кнопку \kw{OK}. & Диалоговое окно закрывается. В табличной части формы появилась новая запись о враче. \\ \hline
\end{longtable}