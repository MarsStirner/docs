\newpage
\sect{Введение}

 Настоящий документ предназначен для персонала медицинского учреждения, ответственного за выгрузку данных в ТФОМС, работников отделов информатизации и автоматизации.
 
 Данное руководство пользователя содержит все необходимые сведения для организации работы по выгрузке и загрузке данных в ТФОМС.
 
 Для работы в Системе администрирования ЛПУ и понимания материала настоящего документа сотрудник должен владеть основной терминологией пользователя ПК, иметь навыки работы на компьютере не ниже уровня пользователя ПК.
 
 Система администрирования ЛПУ – это информационная система 
 

\newpage
\sect{Назначение и условия применения}

 Руководство пользователя является основным справочным документом пользователя по работе с системой. Оно может быть использовано как для обучения работе в системе новых пользователей, так и для расширения и закрепления знаний и навыков пользователей, имеющих опыт работы в системе администрирования ЛПУ.
 
 При возникновении проблем во время работы с Системой администрирования ЛПУ, пользователь должен, в первую очередь, прибегнуть к настоящему документу для поиска решения проблемы, и только в случае, если с помощью данного документа проблему разрешить не удалось, обратиться в службу технической поддержки.
 
 В документе будут использоваться следующие условные обозначения:  \vspace*{0.5em}
 
 \dm{Название} -- так в тексте будут выделяться название полей и пунктов меню приложения.
 
 \btn{OK} -- так будут обозначаться кнопки экранных форм.
 
 \keys{F1} -- так будут обозначаться клавиши на клавиатуре.
 
 \begin{vnim}
  Так будут обозначаться важные предупреждения. Их необходимо прочесть перед выполнением дальнейших инструкций!
 \end{vnim}
 
 \begin{prim}
 Так будут обозначаться полезные замечания, которые не являются обязательными для изучения, однако могут значительно повысить эффективность работы. Продвинутым пользователям рекомендуется обратить на них внимание.
 \end{prim}
