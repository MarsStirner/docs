\newpage
\section{Картотека пациентов}

В \tmisp ~для каждого пациента заводится регистрационная карточка, которая содержит всю персональную информацию о пациенте. Она регистрируется в системе при первом обращении пациента в МУ и выполняет функцию амбулаторной карты пациента. При всех последующих обращениях осуществляется поиск ранее зарегистрированной карточки пациента и привязка к ней очередного случая обращения. В случае изменения каких-либо персональных данных пациента, регистрационную карточку можно отредактировать. 

Таким образом, при работе с \tmist~персональные данные пациента вносятся в систему один раз, а медицинские записи добавляются при каждом обращении. Благодаря данному механизму значительно упрощается процесс получение врачом медицинской информации о предыдущих случаях обращения пациента.

Совокупность регистрационных карточек пациентов, случаев их обращений в ЛПУ и результатов обращений называется картотекой пациентов. Для доступа к картотеке пациентов необходимо перейти по ссылке \dm{Обслуживание пациентов} на панели навигации, расположенной по левой границе страницы, либо щелкнуть по плитке \dm{Обслуживание пациентов} на главной странице системы (Рисунок \ref{img_gen_main}). Будет осуществлен переход на страницу \dm{Обслуживание пациентов} (Рисунок \ref{img_cl_find}). Здесь можно найти карточку пациента или зарегистрировать нового пациента, просмотреть предварительные записи на прием и данные обращений пациента, а так же зарегистрировать новые.

\begin{figure}[ht]\centering
 \includegraphics[width = 1\textwidth ,keepaspectratio]{cl_find}
 \caption{Страница поиска и обслуживания пациентов}
 \label{img_cl_find}
\end{figure} 

\subsection{Поиск регистрационной карточки пациента} \label{cl_find}

В верхней части страницы  \dm{Обслуживание пациентов} (Рисунок \ref{img_cl_find}) находится поле поиска пациентов. В него в произвольном порядке можно ввести фамилию, имя, отчество, дату рождения, код пациента, номера его полиса ОМС или документа, удостоверяющего личность. Дату рождения в поле поиска нужно вводить в формате <<ДД.ММ.ГГГГ>>. % либо <<ДДММГГГГ>> без разделителей. 

После ввода всех необходимых критериев поиска, следует нажать клавишу \keys{Enter} на клавиатуре либо кнопку \btn{Найти} в правой части поля поиска. Через несколько мгновений на экране появится список найденных согласно заданным условиям пациентов (Рисунок \ref{img_cl_findrez}). 
 
\begin{figure}[ht]\centering
 \includegraphics[width = 1\textwidth ,keepaspectratio]{cl_findrez}
 \caption{Результаты поиска пациентов}
 \label{img_cl_findrez}
\end{figure} 

Если по заданным параметрам не будет найдено ни одного пациента, по под полем поиска появится сообщение <<Пациент не найден в базе данных>>.

\ifthenelse{\isnamedefined{fullversion} \OR \isnamedefined{regversion}}
{
\subsection{Регистрационная карточка пациента} \label{cl_card}

Регистрационная карточка пациента содержит всю персональную информацию о пациенте: его фамилию, имя, отчество, дату рождения, адрес, сведения о документах пациента, его страховых полисах, льготах, занятости и т.д. Вся эта информация должна вводиться регистратором с первичных документов пациента при его первом обращении в ЛПУ. В момент регистрации пациента ему присваивается уникальный код, по которому его регистрационную карточку можно быстро найти в картотеке пациентов.

Для регистрации нового пациента необходимо нажать  кнопку \btn{Зарегистрировать пациента} на странице \dm{Обслуживание пациентов} (Рисунок \ref{img_cl_find}). Откроется страница регистрационной карточки пациента (Рисунок \ref{img_cl_card}). 

\begin{figure}[!ht]\centering
 \includegraphics[width = 1\textwidth ,keepaspectratio]{cl_card}
 \caption{Регистрационная карточка пациента}
 \label{img_cl_card}
\end{figure} 

Карточка содержит большое количество информации. Для удобства пользователя, информация разбита на логические блоки. Для перехода к определенному блоку можно щелкнуть левой кнопкой мыши по его названию в правой части страницы либо воспользоваться полосой прокрутки и найти раздел самостоятельно.  

\subsubsection{Блок <<Основная информация>>}

Поля \dm{Фамилия}, \dm{Имя}, \dm{Дата рождения} и \dm{Пол} в регистрационной карточке пациента являются обязательными для заполнения. Они помечены символом <<*>> красного цвета.

Дата рождения пациента может вводиться  с клавиатуры или выбираться из календаря (подрбнее см. раздел \ref{gen_cal}). При заполнении с клавиатуры нужно вводить дату в формате <<ДДММГГГГ>> или <<ДД.ММ.ГГГГ>>. Недостающие разделители вставляются автоматически. Аналогичным образом в \tmisp~заполняются все поля, содержащие даты.

Значение поля \dm{Пол} может вводиться с клавиатуры или выбираться из раскрывающегося списка. 

В поле \dm{СНИЛС} достаточно ввести только цифры, разделительные тире будут вставлены автоматически. При вводе СНИЛС автоматически проводится проверка корректности введенного значения. В случае, если введенное значение не прошло проверку, поле будет выделено красным цветом и справа от поля появится подсказка <<Введен невалидный СНИЛС>>. Сохранение карточки пациента с неверным СНИЛС невозможно.

В поле \dm{Примечание} можно внести дополнительные сведения о пациенте, для которых не предусмотрено отдельных полей в карточке пациента. 
 
\subsubsection{Блок <<Документ удостоверяющий личность>>}

В данном блоке (Рисунок \ref{img_cl_card}) указываются данные документа, удостоверяющего личность пациента. Как правило, для детей таким документом является свидетельство о рождении, а для взрослых – паспорт. Однако, возможны и другие варианты. 
\begin{itemize}
 \item \dm{Тип документа} выбирается из раскрывающегося списка. В зависимости от выбранного типа, определяется набор реквизитов документа, доступных и обязательных для  заполнения. Обязательные для заполнения поля подсвечиваются красным цветом. Для большинства типов документов обязательно требуется указать серию, номер, дату выдачи и кем был выдан документ. 
 \item \dm{Серия, Номер}. Для каждого типа документов задается собственный формат серии и номера: определяется их длина,  возможность ввода в поле буквенных значений, а так же других специальных символов. 
 \item В поле \dm{Дата выдачи} дата может вводиться с клавиатуры или выбираться из календаря (подробнее см. раздел \ref{gen_cal}) 
 \item Поле \dm{Действителен до} заполняется аналогично предыдущему. Если документ не имеет срока действия, то поле может оставаться пустым. Данное поле необходимо заполнить обязательно, если в \tmisp~происходит изменение сведений о действующем документе, удостоверяющем личность (т.е. если пациент предоставил новый документ того же типа, что и предыдущий). 
 \item В поле \dm{Выдан} можно выбрать значение из справочника. %или ввести произвольный текст с клавиатуры. 
 По мере ввода текста в данное поле, в его  раскрывающемся списке будет производиться фильтрация значений в соответствии с введенным текстом. Для выбора значения из справочника необходимо щелкнуть по требуемому значению в списке левой кнопкой мыши или переместить на него курсор с помощью стрелок на клавиатуре, а затем нажать клавишу \keys{Enter}. %Для сохранения значения не из справочника необходимо обязательно нажать клавишу \keys{Enter} в данном поле после завершения ввода. Для очистки поля нужно нажать кнопку \includegraphics[scale=0.7]{clear} в правой части поля.
\end{itemize}

После сохранения информации о документе, удостоверяющем личность пациента появляется возможность прикрепления скан-копии указанного документа (Рисунок \ref{img_cl_docs}). Для этого необходимо нажать кнопку \btn{Добавить копию документа} в нижней части данного блока и в появившемся всплывающем окне выполнить прикрепление документа со сканера или из файла (см. раздел \ref{cl_copydocs}).  

\begin{figure}[ht]\centering
 \includegraphics[width = 1\textwidth ,keepaspectratio]{cl_docs}
 \caption{Регистрационная карточка пациента. Блок ввода документа, удостоверяющего личность}
 \label{img_cl_docs}
\end{figure} 

\subsubsection{Блок <<Адрес регистрации и проживания>>}

В регистрационной карточке существует возможность указания двух адресов для каждого пациента: адреса регистрации и адреса фактического проживания (Рисунок \ref{img_cl_adres}). 

\begin{figure}[ht]\centering
 \includegraphics[width = 1\textwidth ,keepaspectratio]{cl_adres}
 \caption{Регистрационная карточка пациента. Блок ввода адреса регистрации.}
 \label{img_cl_adres}
\end{figure} 

Заполнение адреса рекомендуется выполнять с помощью всероссийского классификатора адресов -- КЛАДР. Для заполнения полей \dm{Населенный пункт} или \dm{Улица} нужно ввести его(ее) наименование или часть наименования в соответствующее поле и нажать клавишу \keys{Enter} на клавиатуре. В раскрывающемся списке поля появится перечень наименований, найденных в соответствии с условиями поиска. Необходимо выбрать из раскрывающегося списка требуемое значение, щелкнув по нему левой кнопкой мыши или установив на него курсор с помощью стрелок на клавиатуре и нажав клавишу \keys{Enter}. Поля \dm{Дом}, \dm{Корпус} и \dm{Квартира} могут заполняться произвольными значеними.

Щелкнув по пиктограмме \includegraphics[scale=0.7]{list} в конце поля \dm{Населенный пункт} или \dm{Улица}, можно повторно раскрыть список наименований, сформированный при поиске значений в справочнике КЛАДР, и выбрать другое значение. Нажатие на пиктограмму \includegraphics[scale=0.7]{clearx} в правой части полей \dm{Населенный пункт} или \dm{Улица} позволяет очистить соответствующее поле. 

Если для данных, внесенных в поля \dm{Населенный пункт} и \dm{Улица}, не найдены соответствия в справочнике КЛАДР, то раскрывающийся список не будет отображен, а при перемещении курсора в другое поле, введенный текст в текущем поле будет скрыт, а поле будет считаться заполненным некорректно (обведено красным цветом). Сохранение карточки пациента при этом будет невозможно. 

В такой ситуации следует еще раз проверить правильность написания названий населенного пункта и улицы, попытаться использовать другое написание или альтернативные названия. Если и после этого найти соответствие в справочнике не удалось, можно щелкнуть левой кнопкой мыши по ссылке <<Ввести вручную>> над соответствующим полем. При выборе ручного ввода в поле \dm{Населенный пункт}, данное поле, а так же поля \dm{Улица}, \dm{Дом}, \dm{Корпус} и \dm{Квартира} исчезнут со страницы, а на их месте появится новое поле \dm{В свободном виде}, куда можно ввести адрес в произвольной форме. Проверка на соответствие справочнику КЛАДР при этом производиться не будет (Рисунок \ref{img_cl_adrfree}). При выборе ручного ввода в поле \dm{Улица}, оно заменяется на поле \dm{Улица в свободном виде}. Введенное в это поле наименование не проверяется на соответствие КЛАДР. Остальные поля блока адреса при вводе улицы в свободном виде заполняются стандартным образом.

\begin{vnim}
Ввод адреса в свободной форме можно применять только в крайних случаях, если выбор из справочника КЛАДР абсолютно невозможен. 
\end{vnim}

\begin{figure}[ht]\centering
 \includegraphics[width = 1\textwidth ,keepaspectratio]{cl_adrfree}
 \caption{Регистрационная карточка пациента. Блок ввода адреса регистрации. Ввод адреса в свободной форме}
 \label{img_cl_adrfree}
\end{figure}   

\begin{prim}
Для того, чтобы от свободного ввода снова вернуться к вводу адреса по справочнику КЛАДР, нужно щелкнуть по ссылке <<Выбрать из Кладр>> над полем свободного ввода улицы или адреса в целом.
\end{prim}

В полях ниже, аналогичным способом можно ввести адрес фактического проживания пациента. Если адрес фактического проживания совпадает с адресом регистрации, то в подразделе \dm{Адрес проживания} можно установить флажок \dm{Совпадает с адресом регистрации}. Тогда данные из подраздела \dm{Адрес регистрации} будут перенесены в соответствующие поля подраздела \dm{Адрес проживания}. %Редактирование адреса проживания при этом становится недоступно. 

Для удаления адреса пациента следует нажать кнопку \includegraphics[scale=1]{del}, расположенную справа от наименования блока.
 
\subsubsection{Блок <<Медицинские полисы>>}
  
В подразделе \dm{Полис ОМС} (Рисунок \ref{img_cl_police}) указываются данные действующего полиса обязательного медицинского страхования. Необходимо указать тип полиса, его серию и номер, дату выдачи, срок действия и название СМО, выдавшей полис. Поле \dm{Действителен до} может оставаться незаполненным, если срок действия полиса не ограничен. Остальные поля являются обязательными для заполнения. 

В поле \dm{Страховая медицинская организация} можно выбрать наименование из справочника или ввести произвольное значение с клавиатуры. По мере ввода текста в данное поле, в его  раскрывающемся списке будет производиться фильтрация значений в соответствии с введенным текстом. Для выбора значения из справочника необходимо щелкнуть по нему левой кнопкой мыши или установить на него курсор с помощью стрелок на клавиатуре, а затем нажать клавишу \keys{Enter}. Для сохранения значения, отсутствуюшего в справочнике требуется нажать клавишу \keys{Enter} в данном поле после завершения ввода. Для очистки поля нужно нажать кнопку \includegraphics[scale=0.7]{clear} в правой части поля.

\begin{figure}[ht]\centering
 \includegraphics[width = 1\textwidth ,keepaspectratio]{cl_police}
 \caption{Регистрационная карточка пациента. Блок <<Медицинские полисы>> пациента}
 \label{img_cl_police}
\end{figure}  

По умолчанию полисы ДМС в карточке пациента отсутствуют. Для регистрации данных о полисе ДМС необходимо щелкнуть по ссылке \dm{<<$+$Добавить новый ДМС>>} в подразделе \dm{Полисы ДМС}. Появятся поля, аналогичные полям в подразделе \dm{Полис ОМС}. Заполнение данных полиса ДМС осуществляется аналогично заполнению раздела \dm{Полисы ОМС}.

Пациент может иметь только один действующий полис ОМС и любое количество действующих полисов ДМС. Поэтому при добавлении нового полиса ОМС, он заменяет ранее зарегистрированный. При добавлении нового полиса ДМС, он добавляется к списку существующих полисов ДМС. 

\begin{prim}
При изменении данных полиса или документа, удостоверяющего личность, ранее введенные данные не теряются. Все ранее зарегистрированные для пациента полисы  и документы, удостоверяющие личность, можно найти в блоке \dm{История изменений документов} (см.раздел \ref{cl_docshistory})
\end{prim}

После сохранения данных полиса, можно прикрепить фотокопию данного документа к регистрационной карточке. Для этого необходимо нажать кнопку \btn{Добавить копию документа} в конце соответствуюшего подраздела и выполнить прикрепление документа со сканера либо из файла (см. раздел \ref{cl_copydocs})


\subsubsection{Блок <<Особенности пациента>>} 

Блок \dm{Особенности пациента} (Рисунок \ref{img_cl_osob}) содержит жизненно-важные параметры, необходимые для оказания медицинской помощи (витальную информацию). 

\begin{figure}[!ht]\centering
 \includegraphics[width = 1\textwidth ,keepaspectratio]{cl_osob}
 \caption{Регистрационная карточка пациента. Блок <<Особенности пациента>>}
 \label{img_cl_osob}
\end{figure} 

Блок содержит 3 подраздела: 
\begin{itemize}
 \item Группа крови и резус фактор;
 \item Аллергия;
 \item Медикаментозная непереносимость.
\end{itemize}

По умолчанию поля для каждого из подразделов скрыты. Для добавления информации в какой-либо из подразделов необходимо щелкнуть по ссылке \dm{<<$+$Добавить новую группу крови>>}, \dm{<<$+$Добавить аллергию>>} или \dm{<<$+$Добавить медикаментозную непереносимость>>} соответственно. Тогда на странице появятся дополнительные поля для ввода информации соответствующего подраздела.

В подразделе \dm{Группа крови и резус-фактор} хранится информация о группе крови  и резусе пациента. Если группа крови пациента известна, то ее необходимо внести в регистрационную карточку. Для этого следует щелкнуть по ссылке \dm{<<$+$Добавить новую группу крови>>} и заполнить появившиеся поля. В поле \dm{Дата установления} нужно указать дату установления группы крови, в поле \dm{Тип} следует выбрать группу крови и резус-фактор из справочника, в поле \dm{Врач, установивший группу крови} выбрать фамилию врача из справочника сотрудников (по умолчанию указывается фамилия текущего пользователя). Все поля обязательны для заполнения. 

Если группа крови была установлена в другом МУ, то в качестве врача, установившего группу крови, следует указывать сотрудника, зарегистрировавшего ее в карточке пациента.

%\begin{vnim}
%Редактирование группы крови и резус-фактора после сохранения регистрационной карточки пациента становится невозможным. Проявите особую аккуратность при внесении этих данных.
%\end{vnim}

Подразделы \dm{Аллергия} и \dm{Медикаментозная непереносимость} заполняются по одному и тому же принципу. В поле  \dm{Вещество (Препарат)} необходимо ввести описание аллергена, например <<пыль>> или <<ампициллин>>, в ячейке \dm{Степень} выбрать из списка степень аллергической реакции, в ячейке \dm{Дата установления} ввести дату установления аллергии (медикаментозной непереносимости), в поле \dm{Примечание}, можно указать дополнительные сведения относительно реакции. В каждом подразделе может содержаться любое количество записей. Добавление новых записей производится щелчком левой кнопки мыши по ссылке \dm{<<$+$Добавить аллергию>>} или \dm{<<$+$Добавить медикаментозную непереносимость>>} соответственно. Поля \dm{Вещество (Препарат)}, \dm{Степень} и \dm{Дата установления} являются обязательными для заполнения. В случае их незаполнения, поля будут выделены красным цветом, сохранение регистрационной карточки пациента при этом будет невозможно.

\subsubsection{Блок <<Социальные статусы>>} 

В этом блоке можно внести ряд дополнительных сведений о пациенте, которые используются, в первую очередь для статистического учета (Рисунок \ref{img_cl_socst}): 

\begin{itemize}
 \item \dm{Инвалидность} -- данные о виде инвалидности пациента и документах, подтверждающих ее.
 \item \dm{Занятость} -- сведения о занятости пациента.
\end{itemize}

По умолчанию поля для каждого из подразделов скрыты. Для добавления информации в какой-либо из подразделов необходимо щелкнуть по ссылке \dm{<<$+$Добавить инвалидность>>} или \dm{<<$+$Добавить занятость>>} соответственно.

\begin{figure}[!ht]\centering
 \includegraphics[width = 1\textwidth ,keepaspectratio]{cl_socst}
 \caption{Регистрационная карточка пациента. Блок <<Социальные статусы>>}
 \label{img_cl_socst}
\end{figure} 

При заполнении данных об инвалидности необходимо выбрать категорию (группу) инвалидности в поле \dm{Тип}, указать дату начала и окончания инвалидности, а так же заполнить данные документа, подтверждающего инвалидность. Поля \dm{Тип}, \dm{Дата начала}, \dm{Тип}(документа), \dm{Серия}, \dm{Номер}, \dm{Дата выдачи}, \dm{Выдан} являются обязательными для заполнения.  В случае их незаполнения, поля будут выделены красным цветом, сохранение регистрационной карточки пациента при этом будет невозможно.
В поле \dm{Выдан} можно выбирать значения из списка или вводить собственные значения. Для очистки поля можно нажать кнопку \includegraphics[scale=0.7]{clear} в правой части поля.

\begin{prim}
Данные обо всех документах, подтверждающих инвалидность можно так же просмотреть в блоке \dm{История изменений документов} (см. раздел \ref{cl_docshistory}).
\end{prim}

При заполнении данных о занятости, достаточно выбрать значение в поле \dm{Тип}. Остальные поля являются необязательными. Для заполнения данных документа о занятости следует щелкнуть по ссылке \dm{<<заполнить документ>>} справа от названия секции \dm{Документ}, тогда поля для ввода данных документа станут доступными для ввода и редактирования.

Для удаления подраздела \dm{Инвалидность} или \dm{Занятость} из карточки следует нажать кнопку \includegraphics[scale=1]{del}, расположенную справа от наименования подраздела.

После сохранения документов, подтверждающих инвалидность или занятость, можно прикрепить фотокопию данных документов к регистрационной карточке. Для этого необходимо нажать кнопку \btn{Добавить копию документа} в конце соответствуюшего подраздела и выполнить прикрепление документа со сканера либо из файла (см. раздел \ref{cl_copydocs})

\subsubsection{Блок <<Гражданство>>}

В данном блоке указывается гражданство пациента. Для каждого пациента можно создать одну или несколько записей о гражданстве.

По умолчанию поля этого блока скрыты. Для добавления сведений о гражданстве необходимо щелкнуть левой кнопкой мыши по ссылке \dm{<<$+$Добавить гражданство>>} (Рисунок \ref{img_cl_state}).

\begin{figure}[!ht]\centering
 \includegraphics[width = 1\textwidth ,keepaspectratio]{cl_state}
 \caption{Регистрационная карточка пациента. Блок <<Гражданство>>}
 \label{img_cl_state}
\end{figure} 

Для заполнения сведений о гражданстве пациента достаточно выбрать его из раскрывающегося списка в поле \dm{Тип}. Поля \dm{Дата начала} и \dm{Дата окончания} в данном блоке не являются обязательными для заполнения.
 
\subsubsection{Блок <<Контактная информация и родственники>>}

В подразделе \dm{Связи с другими пациентами} содержатся указатели на регистрационные карточки родственников пациента (Рисунок \ref{img_cl_contact}). 

\begin{figure}[ht!]\centering
 \includegraphics[width = 1\textwidth ,keepaspectratio]{cl_contact}
 \caption{Регистрационная карточка пациента. Блок <<Контактная информация и родственники>>}
 \label{img_cl_contact}
\end{figure} 

По умолчанию поля этого блока скрыты. Для добавления связи необходимо щелкнуть левой кнопкой мыши по ссылке \dm{<<$+$Добавить связь>>}. В появившихся полях необходимо выбрать тип связи из раскрывающегося списка в поле \dm{Тип} и соответствующего пациента из картотеки в поле \dm{Родственник}. Список доступных значений в поле \dm{Тип} зависит от пола текущего пациента, а так же направления стрелки в поле \dm{Связь}. Для изменения направления связи необходимо щелкнуть по стрелке, в результате, она изменит свое направление, при этом значение в поле \dm{Тип} также изменится. 

\begin{vnim}
Для создания связи родственник должен быть зарегистрирован в \tmisp.
\end{vnim}

Для поиска родственника необходимо ввести его фамилию в поле \dm{Родственник}. По мере набора фамилии будет осуществляться поиск пациентов в базе данных. Результат поиска будет отображаться в раскрывающемся списке поля. Следует выбрать из предложенного списка нужного пациента. Если не будет найдено соответствия введенных данных с пациентами в картотеке, поле \dm{Родственник} будет выделено красным цветом. Сохранение регистрационной карточки пациента при этом будет невозможно. 

Возможна регистрация нескольких родственников пациента. Для добавления нового родственника следует еще раз щелкнуть левой кнопкой мыши по ссылке \dm{<<$+$Добавить связь>>} и внести данные в новые поля.

В подразделе \dm{Контакты пациента} можно хранить номера телефонов, факсов, адреса электронной почты, данные контактных лиц и т.п. 

Для добавления контактной информации следует щелкнуть левой кнопкой мыши по ссылке \dm{<<$+$Добавить контакт>>} и заполнить появившиеся на экране поля следующим образом:
\begin{itemize}
 \item \dm{Тип} -- выбирается тип контактной информации из раскрывающегося списка;
 \item \dm{Номер} -- вводится номер телефона, факса, адрес электронной почты и т.п.
 \item В поле \dm{Примечание} можно указать любую дополнительную информацию относительно контакта, например, имена родственников, рекомендуемое время звонка и т.д.
\end{itemize} 

Возможен ввод нескольких контактов в карточку пациента. Для добавления нового контакта следует повторно нажать на ссылку \dm{<<$+$Добавить контакт>>}, а затем заполнить появившиеся поля.


\subsubsection{Копии документов} \label{cl_copydocs}

В данном блоке можно добавить или просмотреть фотокопии документов пациента, прикрепленные к карточке (Рисунок \ref{img_cl_copydocs}). 

\begin{figure}[ht!]\centering
 \includegraphics[width = 1\textwidth ,keepaspectratio]{cl_copydocs}
 \caption{Регистрационная карточка пациента. Блок <<Копии документов>>}
 \label{img_cl_copydocs}
\end{figure} 

Добавление копий документов может производиться как в данном блоке, так и в блоке, соответствующем типу сохраняемого документа:
\begin{itemize}
 \item Документы, удостоверяющме личность;
 \item Полисы ОМС;
 \item Полисы ДМС;
 \item Документы, подверждающие инвалидность;
 \item Документы, подтверждающие занятость.
\end{itemize}

Для добавления копии документа из данного блока нужно щелкнуть по ссылке \dm{<<$+$Прикрепить файл>>}, расположенной в правом верхнем углу блока. Для добавления копии документа из блока, соответствующего типу добавляемого документа,  необходимо нажать кнопку \btn{Добавить копию документа} в конце соответствующего типу документа раздела или блока. Данная кнопка доступна только после сохранения информации о документе пациента. После нажатия на ссылку или кнопку соответственно появляется  всплывающее окно(Рисунок \ref{img_cl_addcopydoc}), где имеется возможность прикрепления копии документа непосредственно со сканера или из ранее сохраненного файла.

\begin{figure}[ht]\centering
 \includegraphics[width = 1\textwidth ,keepaspectratio]{cl_addcopydoc}
 \caption{Добавление копии документа со сканера}
 \label{img_cl_addcopydoc}
\end{figure} 

Для получения копии документа со сканера необходимо выбрать вкладку \dm{Сканировать} в левом верхнем углу окна (открывается по умолчанию) (Рисунок \ref{img_cl_addcopydoc}). При отсутствии сканера в списке устройств нужно щелкнуть левой кнопкой мыши по ссылке \dm{<<Получить список доступных устройств>>}. После непродолжительной обработки под кнопкой появится список устройств, подключения к которым обнаруженны на данном компьютере. Необходимо установить переключатель на устройство-сканер, выбрать качество сканирования в раскрывающемся списке \dm{Качество изображения} и нажать кнопку \btn{Получить избражение}. Начнется процесс сканирования, который может занять несколько секунд. После его завершения скан-копия документа появится в правой части всплывающего окна.

Для сохранения копии документа из файла необходимо перейти на вкладку  \dm{Выбрать существующий} в левом верхнем углу окна. Далее следует нажать кнопку \btn{Обзор} и в появившемся окне указать путь к ранее сохраненной фотокопии документа. После загрузки файла в систему, он так же отобразится в правой части окна (Рисунок \ref{img_cl_addcopydoc2}).   

\begin{figure}[ht]\centering
 \includegraphics[width = 1\textwidth ,keepaspectratio]{cl_addcopydoc2}
 \caption{Добавление копии документа из файла}
 \label{img_cl_addcopydoc2}
\end{figure} 

Далее следует заполнить секцию \dm{Информация о документе} в левой части окна:
\begin{itemize}
 \item \dm{Тип документа} -- выбирается из справочника типов документов и должен соответствовать типу прикрепляемого документа. При вызове окна добавления копии документа из секции, содержащей сведения об этом документе, тип устанавливается автоматически в соответствии с типом зарегистрированного документа. 
 \item \dm{Наименование} -- наименование документа, которое будет отображаться в регистрационной карточке пациента. По умолчанию наименование совпадает с выбранным типом документа, но его можно изменить на произвольное.
 \item Принадлежность прикрепляемого документа выбирается переключателем: <<документ пациента>> либо <<документ родственника>>.
\end{itemize}

Над областью просмотра изображения располагаются инструменты, позволяющие выполнять его простейшую графическую обработку:
\begin{itemize}
 \item Кнопки \includegraphics[scale=0.7]{cpdrotate} позволяют повернуть изображение против и по часовой стрелке соответственно.
 \item Кнопка \includegraphics[scale=0.7]{addw} увеличивает размер изображения.
 \item Кнопка \includegraphics[scale=0.7]{delw} уменьшает размер изображения.
 \item Кнопка \includegraphics[scale=0.7]{cpdsize} изменяет размер изображения так, чтобы оно занимало всю доступную область.
 \item Кнопка \includegraphics[scale=0.7]{cpdcrop} дает возможность обрезать изображение. При нажатии на нее появляется возможность выделения прямоугольной области изображения с помощью мыши. Для обрезки изображения по границе выделенной области следует нажать кнопку \includegraphics[scale=0.7]{ok}. Для выхода из режима обрезки изображения следует нажать кнопку \includegraphics[scale=0.7]{xdel}.
 \item Кнопка \includegraphics[scale=0.7]{refresh} позволяет отменить все действия по редактированию и вернуться к исходному изображению.  
\end{itemize}

В \tmisp~возможно создание и хранение многостраничных изображений. Для добавления новой страницы в документ нужно нажать кнопку 
\includegraphics[scale=0.7]{addw}, расположенную в правом нижнем углу всплывающего окна, после чего отсканировать или добавить их существующего файла новое изображение.  На каждой странице возможно размещение только одного изображения. При попытке добавить новое изображение на ту же страницу, предыдущее будет удаляться. Для удаления текущей страницы необходимо нажать кнопку \includegraphics[scale=0.7]{delw}. Удаление страницы приводит к удалению размещенного на ней изображения.

Перемещение между страницами документа осуществляется с помощью кнопок, расположенных слева, под областью просмотра и редактирования изображения. Если кнопки управления страницами не видны на экране, рекомендуется воспользоваться полосой прокрутки, расположенной по правой границе окна и переместиться в нижнюю его часть. 

После того как все страницы добавлены, нужно сохранить копию документа, нажав кнопку \btn{Сохранить} в правом нижнем углу окна. В поле \dm{Копия документа} соответствующего блока регистрационной карточки пациента, а так же в текущем разделе, появится ссылка на прикрепленное изображение (Рисунок \ref{img_cl_addcopydoc3}). Для просмотра и редактирования прикрепленной фотокопии следует щелкнуть по наименованию документа левой кнопкой мыши. 

\begin{figure}[ht]\centering
 \includegraphics[width = 1\textwidth ,keepaspectratio]{cl_addcopydoc3}
 \caption{Информация о прикреплении копии документа в регистрационной карточке пациента}
 \label{img_cl_addcopydoc3}
\end{figure} 

Для удаления фотокопии документа нужно нажать кнопку \btn{Удалить документ} в левом нижнем углу всплывающего окна, а затем подтвердить удаление в появившемся диалоговом окне, нажав кнопку \btn{Да}.  

\subsubsection{Блок <<История изменения документов>>} \label{cl_docshistory}

В данном блоке содержится информация обо всех документах, которые когда-либо регистрировались для данного пациента (Рисунок \ref{img_cl_docshistory}). В истории учитываются документы, удостоверяющие личность пациента, полиса ОМС и ДМС, документы, подтверждающие социальные статусы и инвалидности пациента. Для каждого документа указывается его тип, серия, номер, дата начала и окончания действия. Список документов доступен только для просмотра. Редактирование и удаление документов невозможно. 

\begin{figure}[ht]\centering
 \includegraphics[width = 1\textwidth ,keepaspectratio]{cl_docshistory}
 \caption{Регистрационная карточка пациента. Блок <<История изменений документов>>}
 \label{img_cl_docshistory}
\end{figure} 


\subsection{Регистрация нового пациента} \label{cl_new}

Для регистрации нового пациента необходимо перейти к на страницу \dm{Обслуживание пациентов}, воспользовавшись пиктограммой на панели навигации либо осуществив переход с главной страницы системы, а затем нажать кнопку \btn{Зарегистрировать пациента}  в правой верхней части страницы.

\begin{vnim}
Перед началом регистрации нового пациента необходимо убедиться, что данный пациент не был зарегистрирован ранее. Для этого рекомендуется воспользоваться поиском (см. раздел \ref{cl_find})
\end{vnim}

После нажатия кнопки \btn{Зарегистрировать пациента} откроется страница регистрационной карточки пациента,  содержащая незаполненные поля. Следует ввести все данные пациента в пустые поля в соответствии с разделом \ref{cl_card} Для сохранения введенных данных требуется нажать кнопку \btn{Сохранить} в правой части страницы.

Если какие-либо поля регистрационной карточки были заполнены некорректно или обязательные для заполнения поля остались пустыми, сохранение будет невозможно. В этом случае, при наведении указателя мыши на кнопку \btn{Сохранить}, появится соответствующая подсказка, а ошибочные поля будут обведены красным.

Если сохранение карточки не требуется, нужно нажать кнопку \btn{Отмена} в правой части страницы. Страница регистрации пациента будет закрыта без сохранения данных в БД.

Перемещение между блоками регистрационной карточки можно выполнять как с помощью полос прокрутки или колесика мыши, так и щелкнув левой кнопкой мыши по наименованию соответствующего блока в содержании карточки, расположенном в правой части страницы (Рисунок \ref{img_cl_content}). 

\begin{figure}[ht]\centering
 \includegraphics[width = 1\textwidth ,keepaspectratio]{cl_content}
 \caption{Содержание регистрационной карточки}
 \label{img_cl_content}
\end{figure} 

\subsection{Редактирование регистрационной карточки пациента}

Данные, введенные в регистрационную карточку пациента, не являются статичными, их можно динамически изменять в соответствии с изменениями и уточнениями персональных данных пациента. При изменении каких-либо документов у пациента, его социального статуса, выявлении новых особенностей и т.п., требуется открыть регистрационную карточку пациента на редактирование и внести в нее соответствующие изменения. 

Для редактирования регистрационной карточки пациента, следует найти пациента в БД (см. раздел \ref{cl_find}), щелкнуть по записи о нем левой кнопкой мыши (Рисунок \ref{img_cl_findrez}) и в появившемся всплывающем окне (Рисунок \ref{img_cl_contrwin}) нажать кнопку \btn{Редактировать данные пациента} или кнопку \includegraphics[scale=0.7]{edtb} в правом верхнем углу окна. 
}{}

\begin{figure}[ht]\centering
 \includegraphics[width = 1\textwidth ,keepaspectratio]{cl_contrwin}
 \caption{Окно управления обслуживанием пациента}
 \label{img_cl_contrwin}
\end{figure} 

\ifthenelse{\isnamedefined{fullversion} \OR \isnamedefined{regversion}}{
Откроется страница, содержащая заполненную регистрационную карточку пациента. Требуется внести изменения в соответствующие поля и сохранить их, нажав кнопку \btn{Сохранить}. Состав полей и методы их заполнения подробно рассмотрены в разделе \ref{cl_card}

Блок основной информации о пациенте доступен для редактирования постоянно. Для того, чтобы отредактировать любой другой блок или подраздел, необходимо щелкнуть по значку \includegraphics[scale=1]{edt} в правом верхнем углу соответствующего подраздела, после чего поля выбранного блока станут доступными для редактирования.

Для удаления информации из какого-либо блока или подраздела нужно щелкнуть левой кнопкой мыши по значку  \includegraphics[scale=1]{del} в его правом верхнем углу. Появится диалоговое окно с запросом подтверждения удаления. Следует нажать кнопку \btn{Да} в этом окне, после чего информация будет удалена. В блоке \dm{Документ удостоверяющий личность} при удалении все поля очищаются, но не удаляются со страницы. При удалении информации в других блоках, поля записи удаляются со страницы.

Сыылки \dm{<<$+$Добавить адрес проживания>>} и \dm{<<$+$Добавить адрес регистрации>>} в блоке \dm{Адреса регистрации и проживания}, \dm{<<$+$Добавить новый ОМС>>} и \dm{<<$+$Добавить новый ДМС>>} в блоке \dm{Медицинские полисы} очищают поля соответствующего подраздела для ввода новых данных. Данные предыдущих полисов при этом сохраняются в истории изменений документов. В остальных разделах нажатие на аналогичную ссылку позволяет добавить еще одну запись в выбранном подразделе.

\subsection{Вывод на печать медицинских документов пациента}

Из регистрационной карточки пациента можно распечатать ряд медицинских документов. Для этого требуется нажать кнопку \includegraphics[scale=0.7]{print} в правом верхнем углу страницы регистрационной карточки пациента. Откроется окно \dm{Печать документов}, содержащее список доступных печатных форм (Рисунок \ref{img_cl_printm}). Флажками слева от названия отмечаются документы, выбранные для отправки на печать. Нужно установить флажки рядом с названиями документов, которые  требуется распечатать, и нажать кнопку \btn{Печать}. Отмеченные флажками документы будут выведены на экран для предварительного просмотра, а затем отправлены на принтер. Установка флажка в заголовочной строке позволяет отметить для печати все документы списка.

\begin{figure}[ht]\centering
 \includegraphics[width = 1\textwidth ,keepaspectratio]{cl_printm}
 \caption{Печать медицинских документов пациента}
 \label{img_cl_printm}
\end{figure} 

Кнопка \btn{Печать компактно} тоже выводит выбранные документы на печать, но не делает переход на новую страницу для каждого нового документа.

Если для одного из выбранных для печати документов требуется указане параметров, то в правом нижнем углу всплывающего окна появится кнопка \btn{Далее} (кнопки \btn{Печать} и \btn{Печать компактно} при этом будут недоступны), при натии на которую осуществляется переход в окно задания параметров формирования печатной формы. После заполнения параметров в данном окне можно нажать кнопку \btn{Печать} и \btn{Печать компактно}.

Печать документов пациента так же можно вызвать со страницы обслуживания пациентов (Рисунок \ref{img_cl_findrez}). Для этого необходимо найти пациента и щелкнуть по нему левой кнопкой мыши. В открывшемся окне (Рисунок \ref{img_cl_contrwin}) нужно нажать кнопку \includegraphics[scale=0.7]{print} в нижней части либо в правом верхнем углу всплывающего окна. Откроется список документов для печати (Рисунок \ref{img_cl_printm}), где можно выбрать и вывести на печать документы вышеописанным способом. 
   
Печать медицинских документов, создаваемых в процессе обследования и лечения пациента, будет рассмотрена в следующих разделах.
}{}