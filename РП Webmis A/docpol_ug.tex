\newpage
\section{Обслуживание пациентов}

\subsection {Регистрация обращений} \label{pol_obr}

Каждый раз при обращении пациента в ЛПУ за амбулаторной помощью, в картотеке пациентов для него регистрируется новое обращение. Обращение содержит цель, установленные диагнозы пациента, результаты осмотров и обследований, информацию о назначенных мероприятиях и их выполнении, результат обращения. На основании обращения можно распечатать <<Талон амбулаторного пациента>> (Ф. 025\slash У-12).

Для регистрации обращения на основе предварительной записи следует найти данные пациента в БД (см. п. \ref{cl_find}) и щелкнуть по соответствующей записи левой кнопкой мыши. В появившемся всплывающем окне (Рисунок \ref{img_cl_contrwin}) нужно нажать кнопку \includegraphics[scale=0.6]{addg}, справа от соответствующей записи на вкладке \dm{Предварительная запись}. Будет открыта страница \dm{Создание обращения}.

Кнопки \btn{Создать обращение} или \includegraphics[scale=0.6]{addb} позволяют создавать обращение без предварительной записи. Кнопки доступны:
\begin{itemize}
 \item Из высплывающего окна картотеки пациентов (Рисунок \ref{img_cl_contrwin});
 \item Со страницы создания и редактирования регистрационной карточки пациента. 
\end{itemize}

Дле регистрации обращения без предварительной записи нужно нажать кнопку \btn{Создать обращение} или кнопку \includegraphics[scale=0.6]{addb}. Если на момент регистрации обращения у пациента имеются действующие предварительные записи к другим специалистам, то появится предупреждение во всплывающем окне <<У пациента есть предварительные записи>>.Необходимо убедиться, что предварительные записи были зарегистрированы к другому врачу и только после этого продолжить регистрацию текущего обращения нажатием кнопки \btn{Все равно продолжить}. Отменить создание обращения без предварительной записи можно, нажав кнопку  \btn{Отмена} во всплывающем окне.

После подтверждения создания обращения будет открыта страница \dm{Создание обращения}. На этой странице, прежде всего, необходимо убедиться, что обращение создано для нужного пациента, проверив его данные в правой верхней части окна. После этого нужно заполнить пустые и изменить неверно заполненные поля в блоке \dm{Основная информация}. Часть полей может быть заполнена на основе данных предварительной записи (если обращение создавалось на основе нее) или значениями по умолчанию.

\begin{vnim}
 Поля \dm{Дата выполнения} и \dm{Время выполнения} на данном этапе заполнять не нужно!
\end{vnim}

Все поля, кроме полей \dm{Дата выполнения} и \dm{Время выполнения} являются обязательными для заполнения.
\begin{itemize}
 \item \dm{Тип обращения} выбирается из списка значение <<Поликлиника>>.
 \item \dm{Источник финансирования} – канал оплаты обращения, выбирается из списка.
 \item \dm{Договор} – номер договора об оплате выбирается из списка. Состав списка зависит от выбранного источника финансирования.
 \item \dm{Тип события} – выбирается из списка. Состав списка изменяется в зависимости от выбранного типа обращения и источника финансирования.
 \item \dm{Лечащий врач} – врач, к которому направляется пациент в поликлинике.
 \item \dm{Подразделение} – отделение поликлиники, куда направляется пациент.
 \item \dm{Дата начала} – по умолчанию устанавливается дата предварительной записи либо текущая дата. При необходимости дату можно изменить.
 \item \dm{Время начала} – по умолчанию устанавливается время предварительной записи либо текущее время. При необходимости время можно изменить.
 \item \dm{Дата выполнения} – дата завершения обслуживания по данному обращению. Должна заполняться врачом.
 \item \dm{Время выполнения} – время закрытия обращения. Должно заполняться врачом.
\end{itemize}

После того как все поля заполнены верно, нужно нажать кнопку \btn{Создать} в правом нижнем углу страницы. Будет осуществлен переход к следующему этапу оформления обращения. Как правило, работа регистратуры ограничивается первым этапом. Последующее заполнение обращения выполняется врачом.

Если была предпринята попытка создания обращения не для того пациента или создание обращения вообще не требуется, следует нажать кнопку \btn{Отменить}. Обращение создано не будет.

\subsubsection{Регистрация платных услуг}

В случае обращения пациента за медицинскими услугами на платной основе, необходимо зарегистрировать список услуг, которые будут оказаны пациенту, согласовать стоимость и оформить договор на оказание соответствующих услуг.
\subsection{Создание расписания работы врачей} \label{pol_ttbl_new}