\newpage
\section{Работа поликлиники}

\subsection{Создание расписания работы врачей} \label{pol_ttbl_new}

Если в ЛПУ организован прием пациентов по предварительной записи или по талонам, необходимо регулярно создавать расписание работы врачей, ведущих амбулаторный прием на последующий период работы ЛПУ. В зависимости от организации работы в конкретном ЛПУ, период, на который составляется расписание может быть различным. Рекомендуется создавать расписание на календарный месяц. Однако, возможно создание расписания и на более короткий или более продолжительный период.

Для доступа к созданию и редактированию расписания приема врачей необходимо нажать кнопку \btn{Формирование графика врача} в верхней части страницы на панели управления, либо нажать на блок \dm{Графики работы врачей} на главной странице системы (Рисунок \ref{img_gen_main}). Будет осуществлен переход на страницу \dm{График врача} (Рисунок \ref{img_pol_ttbl1}).

\begin{figure}[ht]\centering
 \includegraphics[width = 1\textwidth ,keepaspectratio]{pol_ttbl1}
 \caption{Страница выбора графика работы}
 \label{img_pol_ttbl1}
\end{figure}

На открывшейся странице необходимо выполнить следующие действия:
\begin{enumerate}
 \item В первом поле выбрать номер года для составления расписания.
 \item Во втором поле выбрать месяц, на который планируется составить расписание.
 \item В третьем поле выбрать сотрудника, для которого планируется составление расписания. По мере ввода текста в данное поле, осуществляется фильтрация списка сотрудников по введенным буквам. Отбираются сотрудника, фамилия или специальность которых начинается с введенного буквосочетания. Когда запись об искомом сотруднике появится на экране, следует выбрать ее из выпадающего списка, щелкнув по ней левой кнопкой мыши. На экране появится расписание выбранного сотрудника за указанный месяц (Рисунок \ref{img_pol_ttbl2}). 
 \item Нажать кнопку \btn{Редактировать} в левой части страницы. 
\end{enumerate}

 \begin{prim}
 Кнопка \includegraphics[scale=0.6]{xdel}, расположенная в правой части поля поиска сотрудника, очищает это поле и снимает фильтрацию со списка сотрудников соответственно.
 \end{prim}

\begin{figure}[ht]\centering
 \includegraphics[width = 1\textwidth ,keepaspectratio]{pol_ttbl2}
 \caption{График работы сотрудника}
 \label{img_pol_ttbl2}
\end{figure}

В результате выполнения вышеперечисленных действий расписание станет доступно для редактирования, на экране появятся дополнительные кнопки управления расписанием (Рисунок \ref{img_pol_ttbl3}):

\begin{figure}[ht]\centering
 \includegraphics[width = 1\textwidth ,keepaspectratio]{pol_ttbl3}
 \caption{Редактирование расписания сотрудника}
 \label{img_pol_ttbl3}
\end{figure}

По назначению кнопок управления расписанием можно выделить следующие группы:
\begin{enumerate}
 \item Кнопки управления сохранением расписания (группа 1, рисунок \ref{img_pol_ttbl3}). 
 \begin{itemize}
  \item Кнопка \btn{Сохранить} позволяет сохранить созданное расписание сотрудника.
  \item Кнопка \btn{Отменить} осуществляет выход из режима редактирования расписания. Все несохраненные изменения будут потеряны.
 \end{itemize}
 \item Кнопка \btn{Заполнить} открывает окно создания расписание работы сотрудника на выделенные дни. Кнопка доступна, если выделен хотя бы один день расписания.
 \item Кнопки группового выделения (группа 3, рисунок \ref{img_pol_ttbl3}) позволяют выделять или снимать выделение для группы дней по условию:
 \begin{itemize}
  \item Кнопка \btn{Нечет дни} выделяет все нечетные дни месяца. Если флажок \dm{Выбирать выходные} установлен, то выделяются все нечетные дни, включая выходные. В противном случае - только нечетные рабочие дни.
 \item Кнопка \dm{Инверсия} инвертирует выделение дней расписания, т.е. снимает выделение с ранее выделенных дней месяца и выделяет все дни, которые были не выделены. C помощью данной кнопки можно быстро выделить все четные дни месяца. Для этого следует последовательно нажать кнопки \btn{Нечет дни} и \btn{Инверсия}. Если флажок \dm{Выбирать выходные} НЕ установлен, то выходные дни не будут выделяться вне зависимости от того, были ли они выделены до нажатия кнопки \btn{Инверсия}.
 \item Кнопка \btn{Все} позволяет выделить все дни месяца. Если флажок \dm{Выбирать выходные} НЕ установлен, то выделяются все рабочие дни месяца.
 \item Кнопка \btn{Снять} снимает все выделения.
 \end{itemize}
 \item Установленный флажок \dm{Выбирать выходные} позволяет включать в группу выбора выходные дни и составлять на них расписание.
 \item Кнопки \includegraphics[scale=0.6]{arr1} (группа 6, рисунок \ref{img_pol_ttbl3}),   позволяют выделить все дни недели, напротив которой расположена кнопка. Если флажок \dm{Выбирать выходные} установлен, то выделяются все дни, в противном случае - только рабочие дни.
\end{enumerate}

Можно так же выделять дни в произвольном порядке, щелкая по ним левой кнопкой мыши. Для снятия выделения с дня следует щелкнуть по нему левой кнопкой мыши повторно.


При заполнении расписания последовательность действий следующая: 
\begin{enumerate}
 \item \label{n2} Необходимо выделить один или несколько дней, расписание на которые совпадает, с помощью кнопок, перечисленных выше, либо щелчком левой кнопки мыши и нажать кнопку \btn{Заполнить} на странице \dm{График врача}. Появится всплывающее окно \dm{Заполнение расписания} (Рисунок \ref{img_pol_ttbl4}). В верхней части открывшегося окна будут перечислены дни, на которые формируется расписание в результате текущей операции. 

 \begin{figure}[ht]\centering
  \includegraphics[width = 1\textwidth ,keepaspectratio]{pol_ttbl4}
  \caption{Окно <<Заполнение расписания>>}
  \label{img_pol_ttbl4}
 \end{figure}

 \item \label{n1} Следует нажать кнопку \btn{Добавить интервал} и заполнить появившиеся поля:
 \begin{itemize}
  \item \dm{Тип} -- тип приема выбирается из списка (<<Амбулаторно>> или <<Hа дому>>);
  \item \dm{Начало приема} -- время начала работы врача по обслуживанию обращений выбранного типа;
  \item \dm{Окончание приема} -- время окончания работы врача по обслуживанию обращений выбранного типа;
  \item \dm{План приема} -- плановое количество пациентов, которых должен принять врач за день. Соответствует количеству талонов, которые будут созданы на текущий день для выбранного типа приема. Является обязательным для заполнения;
  \item \dm{Сверх плана} -- допустимое количество пациентов, которые могут быть записаны дополнительно, сверх планового числа талонов;
  \item \dm{Вне очереди} -- допустимое количество экстренных пациентов, которые могут быть приняты данным врачом без талона в течении дня.
  \item \dm{Кабинет} -- кабинет, в котором ведется прием. Поле доступно только, если выбран амбулаторный тип приема.
 \end{itemize}
 Поля \dm{Тип}, \dm{Начало приема}, \dm{Окончание приема}, \dm{План приема} являются обязательными для заполнения.  Для амбулаторного типа приема обязательным так же является поле \dm{Кабинет}.

\begin{figure}[ht]\centering
 \includegraphics[width = 1\textwidth ,keepaspectratio]{pol_ttbl5}
 \caption{Окно <<Заполнение расписания>> после внесения данных}
 \label{img_pol_ttbl5}
\end{figure}

 \item Повторить п. \ref{n1} для добавления нужного числа интервалов приема. Допускается добавление нескольких интервалов в рамках одного дня приема, в том числе нескольких интервалов одного типа (например, можно добавить 2 интервала одного типа, если прием ведется с перерывом). 

 \item \label{n3} После заполнения расписания окно будет выглядеть следующим образом (Рисунок \ref{img_pol_ttbl5}). Следует нажать кнопку \btn{Подтвердить} в правом нижнем углу окна, после чего оно будет закрыто, а в окне \dm{График врача} расписание на выбранные дни будет заполнено в соответствии с заданным шаблоном. 
 \item При необходимости повторить шаги \ref{n2} -- \ref{n3} нужное число раз для заполнения расписания на весь требуемый период. По окончании заполнения, страница формирования графика врача примет следующий вид (Рисунок \ref{img_pol_ttbl6}). 
 \item Следует нажать кнопку \btn{Сохранить} для внесения расписания в БД. 
\end{enumerate}

\begin{figure}[ht]\centering
 \includegraphics[width = 1\textwidth ,keepaspectratio]{pol_ttbl6}
 \caption{Заполненное расписание сотрудника}
 \label{img_pol_ttbl6}
\end{figure}

\subsubsection{Регистрация отсутствий сотрудников}

Если врач не может вести прием по какой-либо причине, необходимо отметить его отсутствие в расписании. Для этого следует открыть расписание сотрудника на соответствующий период на редактирование, выбрать дни отсутствия и нажать кнопку \btn{Заполнить}. Откроется всплывающее окно \dm{Заполнение расписания}. В правом верхнем углу окна нужно нажать кнопку \btn{Установить причину отсутствия} и в появившемся под кнопкой поле выбрать из раскрывающегося списка причину отсутствия. Далее следует нажать кнопку \btn{Подтвердить} в правом нижнем углу окна. Расписание сотрудника на выбранные дни будет удалено, а в полях дней отсутствия будет указана выбранная причина (Рисунок \ref{img_pol_ttbl7}). После сохранения  расписание сотрудника станет недоступным для записи на дни отсутствия.

\begin{vnim}
 Не забудьте сохранить расписание после внесения информации об отсутствиях кнопкой \btn{Сохранить}.
\end{vnim} 

\begin{figure}[ht]\centering
 \includegraphics[width = 1\textwidth ,keepaspectratio]{pol_ttbl7}
 \caption{Отсутствия сотрудника}
 \label{img_pol_ttbl7}
\end{figure}

Если необходимо отменить запись об отсутствии сотрудника, следует выделить дни, в которые нужно отменить ранее установленное отсутствие и нажать кнопку \btn{Заполнить}. Появится всплывающее окно \dm{Заполнение расписания}. В правом верхнем углу появившегося окна следует нажать кнопку \btn{Убрать причину отсутствия} и нажать кнопку \btn{Подтвердить}. Запись об отсутствии сотрудника будет удалена, однако расписание сотрудника на эти дни нужно будет создать заново.

\subsubsection{Просмотр расписания} \label{pol_ttbl_view}

Просмотреть расписание работы сотрудников можно, нажав кнопку \btn{Просмотр графика работы} в верхней части любой страницы. Будет осуществлен переход на страницу \dm{График врача} (Рисунок \ref{img_pol_ttbl1}).

\begin{prim}
 Просмотр расписания выбранного сотрудника доступен так же при записи пациента на прием.
\end{prim}

В верхней части страницы необходимо выбрать год и месяц, на которые нужно посмотреть расписание, а так же фамилию сотрудника, расписание которого требуется просмотреть (см. раздел \ref{pol_ttbl_new}) После этого ниже, на текущей  странице отобразится расписание выбранного сотрудника (Рисунок \ref{img_pol_ttblview}). По умолчанию будет открыто расписание на текущую неделю выбранного месяца. 

\begin{figure}[ht]\centering
 \includegraphics[width = 1\textwidth ,keepaspectratio]{pol_ttblview}
 \caption{Просмотр расписания сотрудника}
 \label{img_pol_ttblview}
\end{figure}

Для просмотра расписания амбулаторного приема должна быть активирована вкладка \dm{Амбулаторно}. Для просмотра графика обслуживания квартирных вызовов следует осуществить переход на вкладку \dm{На дому} (Рисунок \ref{img_pol_ttblview}, [1]). Ниже, под названием вкладки, отображается список недель выбранного месяца (Рисунок \ref{img_pol_ttblview}, [2]). Для просмотра расписания на другую неделю, следует выбрать ее, щелкнув по ней левой кнопкой мыши.

В основной части страницы отображается список интервалов приема выбранного врача на выбранную неделю. В зависимости от доступности и назначения интервалы имеют следующие цвета и обозначения:
\begin{itemize}
 \item \dm{Зеленый цвет} -- обозначает свободные интервалы приема, на которые могут быть записаны пациенты. На каждом интервале указано время начала приема;
 \item \dm{Оранжевый цвет} с надписью <<CITO>> обозначает интервалы, предназначены для записи экстренных пациентов;
 \item \dm{Серый цвет} с надписью <<Сверх плана>> -- на данный интервал допустимо записать пациента сверх плановой нормы приема;
 \item \dm{Красный цвет} обозначает интервалы, на которые уже записаны пациенты. На каждом интервале указывается время начала приема и фамилия записанного пациента;
 \item \dm{Бежевый цвет} -- обозначает выходные и нерабочие дни.
 \item Интервалы, имеющие \dm{бледно-зеленый}, \dm{бледно-оранжевый} и \dm{бледно-серый} цвета, недоступны для записи пациентов, т.к. относятся к прошлым датам.
\end{itemize}


\subsection{Предварительная запись на прием и создание обращений}

\subsubsection{Предварительная запись на прием} \label{pol_predvz}
Предварительная запись пациентов на прием осуществляется на странице обслуживания пациентов. Для перехода на эту страницу необходимо нажать кнопку \btn{Обслуживание пациентов} вверху любой страницы на панели управления, либо нажать на блок \dm{Обслуживание пациентов} на главной странице системы (Рисунок \ref{img_gen_main}).

Последовательность действий при записи пациента на прием должна быть следующая:
\begin{enumerate}
 \item \label{n4} Необходимо найти пациента в картотеке (см. раздел \ref{cl_find}) Если пациент не был зарегистрирован ранее, его следует зарегистрировать (см. раздел \ref{cl_new}) в  БД.
 \item Если пациент найден в БД, нужно щелкнуть левой кнопкой мыши по записи о нем в списке найденных пациентов и в появившемся всплывающем окне (Рисунок \ref{img_cl_contrwin}) нажать кнопку \btn{Записать на прием} или кнопку \includegraphics[scale=0.6]{rec} в правом верхнем углу окна. Для вновь зарегистрированного пациента можно нажать кнопку \btn{Записать на прием} в правом верхнем углу регистрационной карточки пациента. Откроется страница \dm{Запись пациента на прием}.  
 \item В правой верхней части страницы, в поле поиска врача, нужно ввести фамилию или специальность врача. По мере ввода данных в поле поиска, список врачей, расположенный ниже, будет фильтроваться согласно условиям поиска.
 
 \begin{prim}
   Кнопка  \includegraphics[scale=0.6]{xdel} в поле поиска сотрудника позволяет очистить его, в результате чего, отображается полный список сотрудников ЛПУ. Кнопка  \includegraphics[scale=0.6]{xdel} напротив фамилии пациента, закрывает страницу предварительной записи для данного пациента.
  \end{prim}
  
 \item \label{n5} Далее следует установить флажок напротив одной или нескольких фамилий врачей, к которым требуется записать пациента. На экране появится расписание выбранного сотрудника (или сотрудников) на текущую неделю (Рисунок \ref{img_pol_zapttbl}). Цветовые обозначения интервалов здесь аналогичны описанным в разделе \ref{pol_ttbl_view} Единственное отличие состоит в том, что при записи на прием красным цветом обозначены интервалы, на которые записан текущий пациент. Если интервал занят другим пациентом, то он вовсе не отображается на данной странице.
 
 \begin{figure}[ht]\centering
  \includegraphics[width = 1\textwidth ,keepaspectratio]{pol_zapttbl}
  \caption{Запись пациента на прием}
  \label{img_pol_zapttbl}
 \end{figure}
 
 \item В случае наличия свободных талонов, нужно щелкнуть по одному из них левой кнопкой мыши. Для смены недели просмотра, следует выбрать соответствующие год, месяц и неделю в верхней части страницы, под фамилией пациента.
 \item После выбора свободного интервала, нужно нажать кнопку \btn{Ok} в появившемся всплывающем окне, подтверждающую запись пациента на прием. Выбранный интервал окрасится в красный цвет.
\end{enumerate}

Если одновременно было выбрано несколько сотрудников в списке для просмотра расписания, то их расписания будут отображаться последовательными блоками. Для просмотра расписания следующего сотрудника можно воспользоваться полосой прокрутки либо скрыть расписание предыдущего сотрудника, нажав кнопку \includegraphics[scale=0.6]{rollup}, слева от его фамилии.

Если для выбранного пациента ранее были зарегистрированы предварительные записи к другим специалистам, то расписания этих врачей, будет всегда отображаться при последующих записях на прием вверху списка в свернутом виде (Рисунок \ref{img_pol_zap2ttbl}) таким образом, что будут видны дата и время предварительных записей только текущего пациента. Развернуть расписание можно, нажав кнопку \includegraphics[scale=0.6]{rolldown} слева от фамилии сотрудника.

 \begin{figure}[ht]\centering
  \includegraphics[width = 1\textwidth ,keepaspectratio]{pol_zap2ttbl}
  \caption{Отображение ранее выполненных предварительных записей}
  \label{img_pol_zap2ttbl}
 \end{figure}
     
В \tmis~реализована возможность экстренной записи и записи сверх нормы на выбранную дату. Для записи экстренного пациента, нужно выбрать в расписании талон оранжевого цвета с надписью <<CITO>>. Талоны <<CITO>> всегда располагаются самыми первыми в расписании врача. Если экстренные талоны отсутствую, то для данного врача не предусмотрен прием экстренных пациентов вне очередности приема.

Для записи пациентов сверх нормы нужно выбрать в расписании талон серого цвета с надписью <<Сверх плана>>. Талоны данного типа всегда распологаются в самом конце списка интервалов выбранного врача на соответствующий день. Если талоны <<Сверх нормы>> отсутствуют, то данный врач не осуществляет прием сверх плана либо все они уже заняты. 

\begin{prim}
Количество экстренных пациентов и пациентов сверх плана, которые могут быть записаны к данному врачу на текущий день, настраивается при создании расписания работы каждого врача индивидуально. 
\end{prim}

Для отмены предварительной записи нужно щелкнуть по соответствующему талону красного цвета на странице записи пациентов на прием (Рисунок \ref{img_pol_zapttbl}) и в появившемся всплывающем окне <<Отменить запись на прием?>> нажать кнопку \btn{Ok}. Запись пациента на прием будет отменена, выбранный интервал осободится и окрасится в соответствующий его состоянию цвет.

\subsection{Вызов врача на дом} \label{pol_home}

Механизм регистрации вызовов врача на дом в \tmis~полностью аналогичен предварительной записи на прием в поликлинике. Для регистрации вызова на дом необходимо выполнить шаги \ref{n4} -- \ref{n5}, описанные в п. \ref{pol_predvz} Далее следует перейти на вкладку \dm{На дому} (Рисунок \ref{img_pol_zapttbl}, [1]), а затем способом щелкнуть левой кнопкой мыши по любому свободному интервалу на требуемый день и нажать кнопку \btn{ОК} в окне подтверждения записи на прием. Вызов врача но дом будет зарегистрирован, а интервал окрасится в красный цвет.

\subsection {Регистрация обращений} \label{pol_obr}

Каждый раз при обращении пациента в ЛПУ за амбулаторной помощью, в картотеке пациентов для него регистрируется новое обращение. Обращение содержит цель, установленные диагнозы пациента, результаты осмотров и обследований, информацию о назначенных мероприятиях и их выполнении, результат обращения. На основании обращения можно распечатать <<Талон амбулаторного пациента>> (Ф. 025\slash У-12).

Для регистрации обращения на основе предварительной записи следует найти данные пациента в БД (см. п. \ref{cl_find}) и щелкнуть по соответствующей записи левой кнопкой мыши. В появившемся всплывающем окне (Рисунок \ref{img_cl_contrwin}) нужно нажать кнопку \includegraphics[scale=0.6]{addg}, справа от соответствующей записи на вкладке \dm{Предварительная запись}. Будет открыта страница \dm{Создание обращения}.

Кнопки \btn{Создать обращение} или \includegraphics[scale=0.6]{addb} позволяют создавать обращение без предварительной записи. Кнопки доступны:
\begin{itemize}
 \item Из высплывающего окна картотеки пациентов (Рисунок \ref{img_cl_contrwin});
 \item Со страницы создания и редактирования регистрационной карточки пациента. 
\end{itemize}

Дле регистрации обращения без предварительной записи нужно нажать кнопку \btn{Создать обращение} или кнопку \includegraphics[scale=0.6]{addb}. Если на момент регистрации обращения у пациента имеются действующие предварительные записи к другим специалистам, то появится предупреждение во всплывающем окне <<У пациента есть предварительные записи>>.Необходимо убедиться, что предварительные записи были зарегистрированы к другому врачу и только после этого продолжить регистрацию текущего обращения нажатием кнопки \btn{Все равно продолжить}. Отменить создание обращения без предварительной записи можно, нажав кнопку  \btn{Отмена} во всплывающем окне.

После подтверждения создания обращения будет открыта страница \dm{Создание обращения}. На этой странице, прежде всего, необходимо убедиться, что обращение создано для нужного пациента, проверив его данные в правой верхней части окна. После этого нужно заполнить пустые и изменить неверно заполненные поля в блоке \dm{Основная информация}. Часть полей может быть заполнена на основе данных предварительной записи (если обращение создавалось на основе нее) или значениями по умолчанию.

\begin{vnim}
 Поля \dm{Дата выполнения} и \dm{Время выполнения} на данном этапе заполнять не нужно!
\end{vnim}

Все поля, кроме полей \dm{Дата выполнения} и \dm{Время выполнения} являются обязательными для заполнения.
\begin{itemize}
 \item \dm{Тип обращения} выбирается из списка значение <<Поликлиника>>.
 \item \dm{Источник финансирования} – канал оплаты обращения, выбирается из списка.
 \item \dm{Договор} – номер договора об оплате выбирается из списка. Состав списка зависит от выбранного источника финансирования.
 \item \dm{Тип события} – выбирается из списка. Состав списка изменяется в зависимости от выбранного типа обращения и источника финансирования.
 \item \dm{Лечащий врач} – врач, к которому направляется пациент в поликлинике.
 \item \dm{Подразделение} – отделение поликлиники, куда направляется пациент.
 \item \dm{Дата начала} – по умолчанию устанавливается дата предварительной записи либо текущая дата. При необходимости дату можно изменить.
 \item \dm{Время начала} – по умолчанию устанавливается время предварительной записи либо текущее время. При необходимости время можно изменить.
 \item \dm{Дата выполнения} – дата завершения обслуживания по данному обращению. Должна заполняться врачом.
 \item \dm{Время выполнения} – время закрытия обращения. Должно заполняться врачом.
\end{itemize}

После того как все поля заполнены верно, нужно нажать кнопку \btn{Создать} в правом нижнем углу страницы. Будет осуществлен переход к следующему этапу оформления обращения. Как правило, работа регистратуры ограничивается первым этапом. Последующее заполнение обращения выполняется врачом.

Если была предпринята попытка создания обращения не для того пациента или создание обращения вообще не требуется, следует нажать кнопку \btn{Отменить}. Обращение создано не будет.

\subsubsection{Регистрация платных услуг}

В случае обращения пациента за медицинскими услугами на платной основе, необходимо зарегистрировать список услуг, которые будут оказаны пациенту, согласовать стоимость и оформить договор на оказание соответствующих услуг.